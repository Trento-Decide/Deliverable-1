\chapter{Il Progetto Trento Decide}

\section{Introduzione}
Il progetto si basa sull'assunto che il comune di Trento non sia in grado di determinare con precisione in maniera assolutamente ottimale la totalità dei problemi apprezzabili nel territorio comunale. Trento Decide si pone come piattaforma che mira a ridurre il gap tra l'efficienza comunale attuale, ed un utopistica efficienza assoluta. Nel software Trento Decide i cittadini, autenticatisi come tali, possono: pubblicare, modificare e votare iniziative di carattere pubblico, nei più svariati ambiti utili alla cittadinanza, come: urbanistica, ambiente, cultura... \\
I cittadini votando una proposta pubblicata, hanno la possibilità di portarla all'attenzione del comune, che si occuperà a quel punto di mettere in atto una breve analisi di fattibilità tecnica, al termine della quale verrà pubblicata insieme all'esito riguardo l'iniziativa, che può essere rifiutata o accettata e dunque concretizzata. \\
il progetto trattasi in pratica di una applicazione web, accessibile via browser e destinata a due tipologie di utenti principali: cittadini e tecnici comunali.
Gli utenti nello specifico potranno: pubblicare iniziative, motivandole attraverso titolo e descrizione, se tale iniziativa viene pubblicata in una categoria notevole (def 1.1) il sistema proporrà automaticamente delle preventive valutazioni tecnico-scientifiche. Un cittadino in disaccordo o parziale accordo con una iniziativa potrà proporre di modificarla, giustificandone il motivo. Un cittadino in accordo con un'iniziativa potrà votarla, per portarla all'attenzione del comune. Un tecnico del comune, una volta notificatagli un'iniziativa la quale somma dei voti abbia superato il limite previsto di voto (def 1.2), dovrà inoltrarla all'ufficio di competenza, e fornire informazioni in maniera trasparente di tutto il processo di valutazione al pubblico. Una volta completata l'analisi tecnica, il tecnico riporterà i dati alla cittadinanza, de facto notificando la rifiuta o l'accettazione della proposta. \\
La piattaforma sarà totalmente disponibile e usabile da browser, senza la necessità di installare alcun pacchetto aggiuntivo, progettato per garantire: trasparenza, chiarezza e sicurezza.

\section{Vantaggi}
