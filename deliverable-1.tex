% !TeX spellcheck = it_IT
\documentclass[a4paper]{report}

\usepackage[a4paper, margin=2.5cm]{geometry}

\usepackage[utf8]{inputenc}
\usepackage[T1]{fontenc}
\usepackage[italian]{babel}
\usepackage{fancyhdr}
\usepackage{xcolor}
\usepackage{tocloft}

\newlength{\originalsecnumwidth}
\setlength{\originalsecnumwidth}{\cftsecnumwidth}

\usepackage{tocloft}

\newlength{\originalsubsecnumwidth}
\setlength{\originalsubsecnumwidth}{\cftsubsecnumwidth}


\newcounter{rfstart}
\newenvironment{rfscope}{
    \setcounter{rfstart}{\value{section}}
    \renewcommand{\thesubsection}{\textbf{RF\the\numexpr\value{section}-\value{rfstart}\relax.\arabic{subsection}}}
    \renewcommand{\thesubsubsection}{\textbf{\thesubsection.\arabic{subsubsection}}}
      \addtocontents{toc}{\protect\setlength{\cftsubsecnumwidth}{3.9em}}
}{
  \addtocontents{toc}{\protect\setlength{\cftsecnumwidth}{\originalsecnumwidth}}
}

\newenvironment{rnfscope}{
    \setcounter{section}{0}    
    \renewcommand{\thesection}{\textbf{RNF\arabic{section}}}
    \addtocontents{toc}{\protect\setlength{\cftsecnumwidth}{4em}}
}{
    \addtocontents{toc}{\protect\setlength{\cftsecnumwidth}{\originalsecnumwidth}}
}

\newcounter{ucstart}
\newlength{\originalucsubsecwidth}
\setlength{\originalucsubsecwidth}{\cftsubsecnumwidth}
\newenvironment{ucscope}{
    \setcounter{ucstart}{\value{section}}
    \renewcommand{\thesubsection}{\textbf{UC\the\numexpr\value{section}-\value{ucstart}\relax.\arabic{subsection}}}
    \renewcommand{\thesubsubsection}{\textbf{\thesubsection.\arabic{subsubsection}}}
    \addtocontents{toc}{\protect\setlength{\cftsubsecnumwidth}{3.5em}}
}{
    \addtocontents{toc}{\protect\setlength{\cftsubsecnumwidth}{\originalucsubsecwidth}}
}

\newenvironment{usscope}{
    \setcounter{section}{0}    
    \renewcommand{\thesection}{\textbf{US\arabic{section}}}
    \addtocontents{toc}{\protect\setlength{\cftsecnumwidth}{4em}}
}{
    \addtocontents{toc}{\protect\setlength{\cftsecnumwidth}{\originalsecnumwidth}}
}

\definecolor{trentogreen}{RGB}{0, 122, 82}
\definecolor{grayline}{RGB}{200, 200, 200}

\setcounter{secnumdepth}{3}
\setcounter{tocdepth}{3}

\usepackage{hyperref}
\hypersetup{
  colorlinks=true,
  linkcolor=blue,
  citecolor=blue,
  urlcolor=blue
}
\pagestyle{fancy}
\fancyhf{}
\rhead{Documento di Analisi dei Requisiti}
\lhead{Trento Decide}
\cfoot{\thepage}

\usepackage{graphicx}

\newcommand{\defref}[1]{(def.~\ref{#1})}

\newcounter{definition}
\renewcommand{\thedefinition}{\arabic{definition}}

\newcounter{rfsection}
\setcounter{rfsection}{1}

\newcounter{ucsection}
\setcounter{ucsection}{1}

\renewcommand{\bibname}{Bibliografia e Sitografia}

\usepackage{etoc}

\begin{document}

\begin{titlepage}
  \centering

  {\Huge \bfseries Documento di Analisi dei Requisiti }\\[1.5em]
  \Large \emph{Progetto Trento Decide}\\[1em]
  \Large {Corso di Ingegneria del Software \\
Università di Trento}\\[1em]

  {\large Versione 2.3 }
  \vspace{0.5cm}

  {\large \today \par}
  \vspace{3.5cm}

  \includegraphics[width=0.33 \textwidth]{img/unitn.png}

\vfill

\begin{flushright}
  \textbf{Autori}\\[0.4em]
  Youssef Bouadoud (255343)\\
  Alessandro Duranti (251835)\\
  Tommaso Tricker (252029)
\end{flushright}

\vspace{1cm}

\hrule
\vspace{0.3em}
{\small
Dipartimento di Ingegneria e Scienza dell'Informazione
Università di Trento
}

\end{titlepage}

\tableofcontents
\newpage

\etocsettocstyle{}{}

\chapter{Il Progetto Trento Decide}

\section{Introduzione}

Il progetto si fonda sull’assunto che il Comune di Trento, pur disponendo di competenze e strumenti amministrativi avanzati, non sia in grado di individuare e affrontare con assoluta precisione e completezza l’insieme delle problematiche presenti sul territorio comunale. Tale limite non deriva da inefficienze specifiche, ma è una conseguenza fisiologica della complessità urbana e della distanza che spesso si crea tra amministrazione e cittadinanza.
Trento Decide nasce proprio con l’obiettivo di ridurre il divario tra l’efficienza amministrativa attuale e un’ideale efficienza assoluta, fungendo da ponte diretto tra cittadini e istituzioni. La piattaforma si propone di rendere il processo decisionale più inclusivo, trasparente e reattivo, offrendo ai cittadini uno spazio digitale in cui proporre, discutere e votare iniziative di pubblico interesse.
I cittadini possono pubblicare nuove iniziative, modificarle o proporre revisioni nel caso in cui non ne condividano pienamente i contenuti. Inoltre, hanno la possibilità di esprimere il proprio voto sulle proposte esistenti, contribuendo così a determinarne la priorità e la rilevanza all’interno del processo decisionale.
Quando una proposta supera la soglia minima di voti prevista (def. 1.2), essa viene automaticamente segnalata agli amministratori comunali competenti che hanno il compito di avviare una valutazione di fattibilità tecnico-scientifica. Nel caso di iniziative appartenenti a particolari categorie di rilievo (def. 1.1), il sistema fornisce già in fase preliminare delle analisi tecniche automatiche, così da agevolare e velocizzare sia il processo di esame, che la comprensione di quanto trattato.
Il personale comunale amministrativo, una volta ricevuta la notifica di una nuova proposta, provvede a inoltrarla all’ufficio competente e a garantire la massima trasparenza del processo di valutazione, rendendo disponibili alla cittadinanza tutte le informazioni relative alle fasi di analisi. Al termine di tale processo, l’amministrazione comunica pubblicamente l’esito, che può consistere nell’accettazione e conseguente attuazione dell’iniziativa oppure nel suo rifiuto motivato.
L’intero sistema è implementato come applicazione web accessibile direttamente tramite browser, senza la necessità di installare software aggiuntivo da parte degli utilizzatori. La piattaforma è progettata per garantire trasparenza, chiarezza e sicurezza, ponendosi come uno strumento al servizio sia della cittadinanza, che può contribuire attivamente al miglioramento del territorio, sia dell’amministrazione, che beneficia di un canale di comunicazione diretta, strutturata e partecipata.

\section{Glossario}

\begin{description}
  \item[Definizione 1.1 – Categoria di rilievo:] 
  Per \textit{categoria di rilievo} si intende un ambito tematico di particolare rilevanza pubblica, la cui trattazione comporta un impatto significativo sul territorio comunale o sulla qualità della vita collettiva.  
  Le iniziative pubblicate all’interno di tali categorie vengono sottoposte dal sistema a una prima analisi tecnico-scientifica automatica, utile a supportare l’attività decisionale dell’amministrazione comunale.  
  Le principali categorie di rilievo previste nella piattaforma sono:
  \begin{itemize}
    \item \textbf{Urbanistica} – Interventi su spazi pubblici, pianificazione urbana, mobilità e viabilità, arredo urbano e infrastrutture cittadine.
    \item \textbf{Ambiente} – Tutela del verde pubblico, riduzione dell’inquinamento, gestione dei rifiuti, risparmio energetico e promozione di pratiche sostenibili.
    \item \textbf{Sicurezza} – Installazione di sistemi di sorveglianza, illuminazione pubblica, segnaletica, percorsi pedonali e strategie di prevenzione del degrado urbano.
    \item \textbf{Cultura e istruzione} – Realizzazione di eventi culturali, potenziamento di biblioteche, spazi per attività educative, percorsi formativi e valorizzazione del patrimonio storico-artistico.
    \item \textbf{Innovazione e digitale} – Progetti di digitalizzazione dei servizi comunali, creazione di piattaforme online, installazione di sensori e infrastrutture per città intelligenti.
    \item \textbf{Sociale e inclusione} – Servizi di sostegno alle fasce vulnerabili, spazi comunitari, iniziative per l’inclusione sociale e il dialogo interculturale.
    \item \textbf{Mobilità sostenibile} – Piste ciclabili, trasporto pubblico ecologico, infrastrutture per veicoli elettrici e strategie per ridurre l’uso dell’auto privata.
  \end{itemize}
  L’elenco potrà essere esteso nel tempo per includere ulteriori categorie di interesse strategico per la cittadinanza.
  
  \vspace{1em}
  \item[Definizione 1.2 - Soglia di voto:] Numero minimo di voti necessario affinché una proposta venga inoltrata all’amministrazione per la valutazione di fattibilità.
  \item[Definizione 1.3 - Utente:] Insieme di tutti i ruoli predefiniti previsti dalla piattaforma, specificati nel Requisito Funzionale \ref{sec:ruoli}
\end{description}

\section{Vantaggi}

\subsection{Comune}

\begin{itemize}
  \item Prioritizzazione data-driven dei problemi: Il sistema di pubblicazione e 
  voto aiuta a individuare rapidamente le iniziative più rilevanti per la 
  comunità, riducendo il divario tra i bisogni percepiti sul territorio e l'agenda 
  amministrativa.

  \item Flusso operativo standardizzato e tracciabile: dall'arrivo della proposta 
  all'inoltro all'ufficio competente fino all'esito, con linea temporale pubblica e 
  regole chiare; meno email disperse e più ordine procedurale.

  \item Migliore allocazione delle risorse tecniche: per le categorie di rilievo 
  il sistema propone valutazioni tecnico-scientifiche preventive, riducendo 
  analisi ridondanti e concentrando i tecnici sulle pratiche più impattanti.

  \item Canale unico e tracciabile: Ogni proposta segue un flusso chiaro 
  con registri pubblici che agevolano gli audit interni e garantiscono la 
  trasparenza verso l'esterno.

  \item Riduzione del carico sugli sportelli: Centralizzando segnalazioni e 
  proposte, diminuiscono email, PEC e richieste frammentate agli uffici/URP, 
  con risparmi operativi e di tempo.
\end{itemize}

\subsection{Cittadini}

\begin{itemize}
  \item Partecipazione effettiva: Pubblicare, votare e proporre modifiche rende 
  la cittadinanza protagonista nella definizione delle priorità pubbliche.

  \item Chiarezza sul percorso: Ogni iniziativa ha uno stato visibile e aggiornato 
  (proposta, in valutazione, accettata, rifiutata), riducendo l'asimmetria informativa.

  \item Accesso semplice via web: Nessuna installazione richiesta; la piattaforma 
  è usabile da browser e orientata a trasparenza, chiarezza e sicurezza.

  \item Tracciabilità personale: Ogni cittadino può seguire le proprie iniziative, 
  le modifiche proposte e i voti espressi.
\end{itemize}

\section{Limiti}

\begin{itemize}
  \item Definizione della soglia di voto (def 1.2): Un limite troppo alto scoraggia; 
  troppo basso satura gli uffici. La taratura richiede monitoraggio e possibili 
  revisioni.

  \item Aspettative giuridiche ambigue: superare la soglia di voto può essere 
  interpretato come "diritto all'attuazione"; rischio di contenzioso se l'esito 
  è negativo.

  \item Competenza amministrativa: molte proposte ricadono su enti non comunali 
  (Provincia, Stato, gestori di servizi); serve un flusso di re-indirizzamento 
  chiaro.

  \item Costi ricorrenti: moderazione, comunicazione, supporto 
  utenti, osservabilità e test di sicurezza vanno finanziati in modo 
  continuativo.
\end{itemize}

\chapter{Requisiti Funzionali}

\renewcommand{\thesubsection}{\textbf{RF\arabic{section}.\arabic{subsection}}}

\setlength{\parskip}{0.4em}
\setlength{\parindent}{0pt}

\section{Introduzione}
I requisiti funzionali definiscono le funzioni che il sistema \textit{Trento Decide} deve offrire per supportare la partecipazione civica, la collaborazione tra cittadini e amministrazione comunale e la trasparenza del processo decisionale.  
Ogni requisito è formulato in modo chiaro e verificabile ed è identificato dal prefisso \textbf{RF}, seguito da un numero progressivo e, ove necessario, da sottosezioni tematiche.  

% ====================================================
% RF1
% ====================================================

\section{Autenticazione e gestione credenziali}

\subsection{Registrazione dei cittadini} \label{sec:registrazione}
L'utente che intende registrarsi come cittadino deve accedere inizialmente con SPID o CIE, il sistema richiede l'inserimento di email, password e nome utente (credenziali locali); inoltre l’utente può opzionalmente acconsentire a ricevere notifiche via email.  Al termine, il sistema invia all'utente una email di conferma.

\subsection{Registrazione di moderatori e associazioni}
Gli account dei moderatori e delle associazioni/comitati vengono creati dall’amministratore tramite l’interfaccia di gestione.
Il sistema invia ai destinatari le credenziali di accesso tramite email.

\subsection{Login} \label{sec:login}
Il sistema deve consentire il login tramite credenziali locali (email e password) oppure tramite SPID o CIE.

\subsection{Gestione credenziali utente}
L'utente deve poter:
\begin{itemize}
  \item visualizzare e aggiornare email, username e password;
  \item aggiornare la preferenza relativa alla ricezione delle notifiche via email;
   \item consultare lo storico di modifiche alle proprie credenziali;
  \item richiedere la cancellazione del proprio profilo;
\end{itemize}

% ====================================================
% RF2
% ====================================================

\section{Gestione proposte cittadine e sondaggi}

\subsection{Creazione proposta}
Il sistema deve consentire ai cittadini di creare nuove proposte in stato di \textit{bozza}, compilando un modulo composto dai seguenti campi obbligatori:
\begin{itemize}
    \item \textbf{Titolo} — testo breve che identifica la proposta;
    \item \textbf{Descrizione} — testo esteso che illustra il contenuto della proposta;
    \item \textbf{Luogo} — indicazione dell’area geografica interessata, selezionabile tramite indirizzo o mappa;
    \item \textbf{Categoria} — definizione \ref{sec:cat};
\end{itemize}

Inoltre il sistema deve salvare automaticamente, per ogni proposta, le seguenti informazioni:
\begin{itemize}
    \item \textbf{Autore} — nome utente del cittadino che ha creato la proposta;
    \item \textbf{Data} — timestamp di creazione generato dal sistema;
\end{itemize}

I valori e la presenza di ciascun campo possono variare in base alla categoria selezionata.  
Il sistema deve utilizzare per ogni categoria l’elenco dei campi obbligatori e facoltativi configurato dall’amministrazione comunale.

\subsection{Pubblicazione proposta}
Il sistema deve consentire al cittadino autore di pubblicare una proposta precedentemente salvata in stato di \textit{bozza}.  
La pubblicazione è possibile solo se tutti i campi obbligatori previsti per la categoria selezionata risultano compilati.  
Al momento della pubblicazione il sistema aggiorna lo stato della proposta a \textit{pubblicata}.

\subsection{Proposte collettive}
Il sistema deve consentire ad associazioni e comitati di presentare \textbf{proposte collettive} (def.~\ref{def:proposta-collettiva}).  
Il sistema deve etichettare tali proposte come “collettive” e applicare ad esse gli stessi processi di validazione e pubblicazione previsti per le proposte individuali.

\subsection{Sondaggi} \label{sec:sondaggi}
Il sistema deve consentire all’amministrazione di creare, pubblicare e gestire sondaggi tematici rivolti ai cittadini.  
Ogni sondaggio deve comprendere:  
(a) \textit{Titolo}, 
(b) \textit{Descrizione}, 
(c) \textit{Categoria}, 
(d) \textit{Periodo di apertura} (data di apertura e di chiusura),
(e) \textit{Domande} (e relative opzioni di risposta).  

Il sistema deve inoltre:
\begin{itemize}
	\item chiudere automaticamente i sondaggi alla data di scadenza;
	\item raccogliere in modo anonimo le risposte dei cittadini, rendendole disponibili all'amministrazione;
\end{itemize}

\subsection{Visualizzazione proposte}
Il sistema deve consentire al cittadino di:
\begin{itemize}
  	\item visualizzare l’elenco delle proprie proposte, distinguendo tra quelle in stato di \textit{bozza} e quelle pubblicate;
  	\item visualizzare le proposte pubblicate da altri utenti, applicando i criteri definiti in \ref{sec:sorting}.
	  \item visualizzare i sondaggi pubblicati dall'amministrazione.
\end{itemize}

\subsection{Stato e tracciabilità delle proposte} \label{sec:stato-tracciabilita}
Il sistema deve mostrare per ogni proposta lo stato corrente — \textit{bozza}, \textit{pubblicata}, \textit{in valutazione}, \textit{accettata}, \textit{rifiutata}, \textit{in attuazione}, \textit{completata} — e la cronologia pubblica delle transizioni di stato.

\subsection{Modifica collaborativa}
Il sistema deve consentire la modifica collaborativa delle proposte, come segue:

\subsubsection{Versionamento}
Ogni modifica a una proposta deve registrare autore, data/ora, descrizione e differenze rispetto alla versione precedente; tutte le versioni restano consultabili.
\subsubsection{Proposte di modifica da terzi}
Un utente diverso dal creatore può proporre una modifica; la proposta entra in uno stato di revisione non pubblica fino a decisione.
\subsubsection{Approvazione delle modifiche}
Il creatore della proposta può accettare o rifiutare una proposta di modifica; l’esito è tracciato.
\subsubsection{Ripristino}
Deve essere possibile ripristinare una versione precedente come versione principale.

\subsection{Endorsement e raccolta firme}
Il sistema deve consentire ai cittadini e alle associazioni registrate di esprimere un \textbf{endorsement} digitale (def.~\ref{def:endorsement}) sulle proposte pubblicate.  
Il sistema deve inoltre registrare e verificare le firme fisiche raccolte offline, integrandole nel conteggio del sostegno alla proposta.

% ====================================================
% RF3
% ====================================================

\section{Personalizzazione e preferenze}

\subsection{Preferiti} \label{sec:preferiti}
Il sistema deve consentire al cittadino di:
\begin{itemize}
	\item contrassegnare e rimuovere dai \textit{preferiti} proposte e sondaggi;
	\item visualizzare l’elenco dei propri elementi preferiti;
\end{itemize}

\subsection{Notifiche e avvisi}
Il sistema deve inviare notifiche automatiche per variazioni di stato o richieste di modifica (aggiornamenti relativi a elementi tra i preferiti).
Le notifiche vengono inviate solo se l’utente ha acconsentito alla ricezione di notifiche (\ref{sec:registrazione}).

\subsection{Cambio di lingua}
Il sistema deve consentire all’utente di modificare la lingua dell’interfaccia tramite un menu dedicato accessibile dall’area utente.  
La preferenza linguistica deve essere registrata nel profilo utente e mantenuta tra le sessioni di accesso.

% ====================================================
% RF4
% ====================================================

\section{Votazioni e ordinamento}

\subsection{Votazione}
Il sistema deve consentire agli utenti eleggibili al voto (cfr.~Definizioni) di esprimere un voto positivo o negativo su:
\begin{itemize}
	\item \textbf{Proposte}: voto modificabile fino allo stato di \textit{accettata} o \textit{rifiutata};
	\item \textbf{Sondaggi}: voto modificabile fino a chiusura dello stesso;
\end{itemize}
Ogni elettore può esprimere un solo voto.

\subsection{Ordinamento delle proposte}  \label{sec:sorting}
Il sistema deve consentire all’utente di ordinare l’elenco delle proposte pubblicate attraverso almeno le seguenti chiavi:
\begin{itemize}
  \item punteggio di rilevanza (valore numerico esposto);
  \item numero totale di voti;
  \item data di pubblicazione (più recente / meno recente);
  \item categoria.
\end{itemize}
L’utente deve poter invertire l’ordine (asc/desc) quando applicabile.

% ====================================================
% RF5
% ====================================================

\section{Moderazione e qualità dei contenuti}

\subsection{Moderazione automatica}
Il sistema deve analizzare automaticamente i contenuti generati dagli utenti (testi e materiali multimediali) per rilevare linguaggio inappropriato, spam o duplicati.  
In caso di rilevamento, il sistema deve segnalare l’elemento ai moderatori o sospenderne la pubblicazione in attesa di verifica.

\subsection{Intervento dei moderatori}
Il sistema deve includere un modulo di moderazione che consenta di gestire i contenuti segnalati o non conformi.  
I cittadini devono poter segnalare un contenuto.  
I moderatori devono poter sospendere o eliminare i contenuti segnalati.  
Ogni azione di moderazione deve essere registrata con identificativo utente, data e motivazione.

% ====================================================
% RF6
% ====================================================

\section{Policy Simulator e modelli statistici}

\subsection{Simulazione di scenari di policy}
Il sistema deve includere un modulo che consenta di calcolare indicatori di impatto economico, ambientale e sociale delle proposte utilizzando dataset configurabili.

\subsection{Analisi predittiva e statistica}
Il sistema deve consentire la generazione di previsioni sull’effetto delle proposte tramite modelli configurabili; gli indicatori prodotti devono essere quantitativi e misurabili (es. variazioni stimate di traffico, emissioni o costi).

\subsection{Visualizzazione interattiva delle simulazioni}
Il sistema deve visualizzare i risultati delle simulazioni tramite mappe tematiche e grafici comparativi.


% ====================================================
% RF7
% ====================================================

\section{Trasparenza e reportistica}

\subsection{Report votazioni} \label{sec:report-votazioni}
Il sistema deve consentire all’amministrazione di generare in formato CSV report riepilogativi sulle votazioni concluse.  Ogni report deve includere i seguenti indicatori: (a) numero totale di votanti; (b) distribuzione territoriale; (c) distribuzione per fasce d’età.  

\subsection{Report sui dati generali del sistema}  \label{sec:report-generali}
Il sistema deve fornire all’amministrazione una dashboard che mostri il numero di utenti attivi, le proposte per categoria, la distribuzione territoriale dei voti e i tassi di approvazione.  
I dati devono poter essere esportati in formato CSV.

\subsection{Open Data}
Il sistema deve consentire il download in locale dei dataset relativi ai report descritti in \ref{sec:report-generali} e \ref{sec:report-votazioni}.  
I dati devono essere resi disponibili in formato aperto (CSV) e comprendere esclusivamente informazioni anonimizzate.

\subsection{Trasparenza e accountability} \label{sec:trasparenza}
Il sistema deve consentire all’amministrazione comunale, al termine del processo di valutazione, di pubblicare per ogni proposta conclusa un riscontro ufficiale contenente l’esito (accettazione o rifiuto) e la relativa motivazione.  
Il riscontro deve essere visibile nella pagina pubblica della proposta, associato allo stato finale.



\setcounter{secnumdepth}{3}

\chapter{Requisiti Non Funzionali}

\renewcommand{\thesubsection}{RNF\arabic{subsection}}
\setlength{\parskip}{0.4em}
\setlength{\parindent}{0pt}

\section{Introduzione}
I requisiti non funzionali definiscono le proprietà qualitative, operative e normative che il sistema \textit{Trento Decide} deve rispettare per garantire sicurezza, efficienza, accessibilità e affidabilità.  
Ogni requisito è formulato in modo chiaro, verificabile e, ove possibile, riferito a metriche misurabili e standard riconosciuti.

\subsection{Affidabilità}
Il sistema deve garantire una disponibilità minima del 99.72\% (downtime massimo 24 ore/anno).  
Devono essere definiti obiettivi RTO $\le$ 2 ore e RPO $\le$ 12 ore.  
Deve essere previsto un piano di \textit{incident response} e test di resilienza semestrali.

\subsection{Backup e ripristino}
Il sistema deve eseguire backup automatici ogni 12 ore di dati, configurazioni e log applicativi.  
I backup devono essere conservati per almeno 90 giorni e replicati in un data center situato nell’Unione Europea qualificato AgID. 
Devono essere previsti controlli di integrità e procedure di ripristino periodicamente testate.

\subsection{Compatibilità}
Il sistema deve essere conforme agli standard HTML5, CSS3 ed ECMAScript 6, garantendo la piena compatibilità con i principali browser utilizzati nei contesti della Pubblica Amministrazione e dagli utenti finali.  
Il supporto deve coprire almeno le ultime due versioni stabili dei browser Mozilla Firefox (inclusa ESR), Google Chrome/Chromium, Microsoft Edge, Apple Safari e Opera.    
L’interfaccia deve risultare completamente responsive e accessibile da desktop, tablet e smartphone.

\subsection{Etica e imparzialità del sistema} \label{ref:etica}
Il sistema deve promuovere trasparenza, equità e neutralità nei processi di partecipazione, evitando ogni forma di bias politico, ideologico o sociale.  
Tale principio si applica sia ai comportamenti umani sia ai processi automatizzati.

\subsubsection{Neutralità e trasparenza algoritmica} \label{ref:neutralita}
Gli algoritmi di ranking e simulazione devono essere progettati per evitare distorsioni sistematiche o favoritismi impliciti.  
Le formule e i parametri utilizzati devono essere documentati, versionati e sottoposti a revisione almeno annuale.  
Ogni modifica deve essere tracciata in un registro pubblico consultabile dai cittadini.

\subsection{Internazionalizzazione e lingua} \label{ref:i18n}
Il sistema deve supportare almeno cinque lingue: Italiano, Inglese, Tedesco, Arabo (RTL) e Rumeno.  
La lingua dell’interfaccia deve essere selezionabile e persistente a livello di profilo utente.  
I contenuti generati dagli utenti rimangono nella lingua di origine, con possibilità di traduzione automatica informativa chiaramente segnalata.

\subsection{Moderazione e correttezza d’uso} \label{ref:moderazione}
Il sistema deve garantire un ambiente rispettoso mediante meccanismi di moderazione dei contenuti e delle interazioni tra utenti.  
Le segnalazioni devono essere prese in carico entro 24 ore dalla ricezione.  
Tutte le azioni di moderazione devono essere registrate (utente, data/ora, motivazione) con log non alterabile; deve esistere un canale di appello per l’utente.

\subsection{Performance} \label{ref:performance}
Il sistema deve garantire che il 95\% delle operazioni principali (login, consultazione, voto) sia completato entro 1 secondo.  
Tale requisito deve essere mantenuto con un carico simultaneo fino a 20.000 utenti.  
L’interfaccia deve rispettare i target \textit{Core Web Vitals}: LCP $\le$ 2,5 s (p75), INP $\le$ 200 ms (p75), CLS $\le$ 0,1 (p75).

\subsection{Portabilità}
Il sistema deve poter essere distribuito su infrastrutture on-premise o cloud qualificati nell’Unione Europea.  
La distribuzione deve essere containerizzata e replicabile tramite strumenti di \textit{Infrastructure as Code}.  
Compatibilità garantita con database relazionali standard (es. PostgreSQL) e storage a oggetti.

\subsection{Scalabilità}
Il sistema deve supportare la scalabilità orizzontale e verticale senza downtime.  
L’architettura deve permettere l’aggiunta di nuove istanze di servizio e la gestione di carichi concorrenti elevati.

\subsection{Sicurezza}
Il sistema deve rispettare il livello 2 dello standard OWASP ASVS e adottare il principio di \textit{security by design}.  
Le comunicazioni devono essere cifrate, i dati sensibili protetti e devono essere attuate misure contro attacchi comuni (CSRF, XSS, SQLi, brute force).  
È richiesto un audit di sicurezza annuale e la disponibilità di un piano DPIA approvato dal DPO.

\subsection{Usabilità e accessibilità}
Il sistema deve essere utilizzabile senza formazione per gli utenti esterni; la formazione per operatori non deve superare 45 minuti.  
L’interfaccia deve rispettare gli standard WCAG 2.1 livello AA e raggiungere un punteggio minimo SUS $\ge$ 80.
\chapter{Use Case Diagram}

\renewcommand{\thesubsection}{UC\arabic{subsection}}
\setlength{\parskip}{0.4em}
\setlength{\parindent}{0pt}

\section{Introduzione}
In questa sezione vengono presentati gli Use Case Diagram del sistema Trento Decide.
Essi rappresentano, in forma grafica, le interazioni tra gli attori e la piattaforma, illustrando le principali funzionalità e relazioni di inclusione o estensione tra i casi d’uso.
Questi diagrammi consentono di visualizzare in modo sintetico come i requisiti funzionali (RF) si traducono in comportamenti concreti del sistema, offrendo una visione d’insieme delle dinamiche operative e dei ruoli coinvolti.

% ====================================================
% UC1
% ====================================================

\section{Utente Anonimo}
\begin{enumerate}
    \item Fare login sulla piattaforma (\ref{sec:login})
    \item Registrare un nuovo cittadino nella piattaforma (\ref{sec:registrazione})
\end{enumerate}

\begin{center}
	\includegraphics[scale=1.1]{img/usecase/1.png}
\end{center}

% ====================================================

\subsection{Login \ref{sec:login}}

\textbf{Riassunto:}  
Questo use case descrive come un utente anonimo accede alla piattaforma tramite credenziali locali oppure tramite SPID o CIE.

\textbf{Precondizioni:}  
L’utente non è autenticato.

\textbf{Postcondizioni:}  
L’utente risulta autenticato e accede alla piattaforma.

\textbf{Descrizione:}
\begin{enumerate}
    \item L’utente visualizza la schermata di login.
    \item L’utente può scegliere una delle seguenti modalità:
    \begin{enumerate}
        \item \textbf{Credenziali locali:} inserisce email e password nei campi dedicati e conferma l’accesso.
        \item \textbf{SPID/CIE:} seleziona l’opzione SPID o CIE; il sistema reindirizza al relativo servizio di autenticazione.
    \end{enumerate}
    \item Se l’autenticazione è valida, l'utente accede al sistema \textbf{[eccezione 1][eccezione 2]}.
\end{enumerate}

\textbf{Eccezioni:}
\begin{itemize}
    \item[(1)] Se email o password non sono valide, il sistema mostra un messaggio di errore e richiede di reinserire i dati.
    \item[(2)] Se l’autenticazione SPID/CIE fallisce, viene annullata o scade, il sistema mostra un messaggio di errore e ritorna alla schermata di login.
\end{itemize}

% ====================================================

\subsection{Registrare un cittadino \ref{sec:registrazione}}

\textbf{Riassunto:}  
Questo use case descrive come un utente anonimo registra un nuovo account cittadino tramite SPID o CIE e definisce le proprie credenziali locali.

\textbf{Precondizioni:}  
L’utente non possiede ancora un account sulla piattaforma.

\textbf{Postcondizioni:}  
Un nuovo account cittadino è creato e attivo; le credenziali locali risultano configurate.

\textbf{Descrizione:}
\begin{enumerate}
    \item L’utente accede alla schermata di registrazione.
    \item L’utente seleziona una modalità di identificazione:
    \begin{enumerate}
        \item \textbf{SPID} — reindirizzamento al provider SPID;
        \item \textbf{CIE} — reindirizzamento al servizio di autenticazione tramite CIE.
    \end{enumerate}
    \item Una volta validata l’identità tramite SPID/CIE \textbf{[eccezione 1]}, il sistema richiede:
    \begin{enumerate}
        \item inserimento dell’email;
        \item definizione della password;
        \item scelta del nome utente.
    \end{enumerate}
    \item L’utente conferma i dati inseriti.
    \item Il sistema crea l’account e invia un’email di conferma all’indirizzo fornito \textbf{[eccezione 2]}.
\end{enumerate}

\textbf{Eccezioni:}
\begin{itemize}
	\item[(1)] Se l’autenticazione SPID/CIE fallisce o viene annullata, la registrazione non procede.
    	\item[(2)] Se l’email non è valida o già associata a un altro account, il sistema notifica l’errore e richiede la correzione del dato. 
\end{itemize}

\textbf{Estensioni:}
\begin{itemize}
    \item[(1)] \textit{Estensione del passo 3:} l’utente può selezionare l’opzione per ricevere notifiche via email; il sistema registra la preferenza.
\end{itemize}

% ====================================================
% UC2
% ====================================================

\section{Utente autenticato}
\begin{enumerate}
	\item Gestire le proprie credenziali (\ref{sec:credenzialihandling})
	\item Visualizzazione proposte (\ref{sec:visualizzazioneproposte}) MHHH NON SONO SICURO QUI!
	\item Ordinare e filtrare le proposte (\ref{sec:sorting}) MHHH NON SONO SICURO QUI!
	\item Cambiare lingua (\ref{sec:cambiolingua})
	\item Visualizzare policy simulator (\ref{sec:vederepolicies}) MHHH NON SONO SICURO QUI!
\end{enumerate}

\begin{center}
	% \includegraphics[scale=1.1]{img/usecase/1.png}
\end{center}

% ====================================================

% ====================================================
% UC3
% ====================================================

\section{Amministratore}
\begin{enumerate}
	\item Creare account moderatore o associazione/comitato (\ref{sec:specialregistrazione})
	\item Pubblicare sondaggi (\ref{sec:sondaggi})
	\item Interazione dashboard amministrativa (\ref{sec:dashboard})
\end{enumerate}

\begin{center}
	% \includegraphics[scale=1.1]{img/usecase/1.png}
\end{center}

% ====================================================

% ====================================================
% UC4
% ====================================================

\section{Cittadino}
\begin{enumerate}
	\item Creare una proposta (\ref{sec:creazioneproposta})
	\item Pubblicare una proposta in stato di bozza (\ref{sec:pubblicazioneproposta})
	\item Proporre una modifica ad una proposta (\ref{sec:modificaproposta})
	\item Modificare i propri preferiti (\ref{sec:preferiti})
	\item Votare una proposta (\ref{sec:voto})
\end{enumerate}

\begin{center}
	% \includegraphics[scale=1.1]{img/usecase/1.png}
\end{center}

% ====================================================

% ====================================================
% UC5
% ====================================================

\section{Associazione/Comitato}
\begin{enumerate}
	\item Creare proposte collettive (\ref{sec:creazioneproposta})
\end{enumerate}

\begin{center}
	% \includegraphics[scale=1.1]{img/usecase/1.png}
	Da capire se possiamo mergiare questo UC5 con Cittadino ma specificandone un' estensione ad hoc per questo
	tipo di accountss
\end{center}

% ====================================================

% ====================================================
% UC6
% ====================================================

\section{Moderatore}
\begin{enumerate}
	\item Fare meglio i RF di moderatore, incompleti
\end{enumerate}

\begin{center}
	% \includegraphics[scale=1.1]{img/usecase/1.png}
\end{center}

% ====================================================



% !TeX spellcheck = it_IT
\chapter{User Stories}

\newcommand{\userstory}[3]{
  #1

  \begin{description}
    \item[Criteri di accettazione:]~
    \begin{itemize}
      #2
    \end{itemize}

    \item[Tasks:]~
    \begin{itemize}
      #3
    \end{itemize}
  \end{description}
}


\section{Introduzione}
Le User Stories descrivono il comportamento del sistema dal punto di vista degli utenti finali, evidenziando bisogni, obiettivi e valore atteso.
Ogni storia segue una struttura uniforme e include criteri di accettazione verificabili e i task principali necessari alla sua implementazione.

\begin{usscope}

\section{Registrazione del cittadino~\ref{rf:registrazionecittadini}}
\userstory
{
  Come utente anonimo, voglio registrarmi alla piattaforma verificando la mia identità tramite SPID o CIE e configurando le mie credenziali locali.
}
{
  \item Il sistema deve richiedere l'autenticazione iniziale tramite SPID o CIE.
  \item Il sistema deve verificare automaticamente la residenza nel Comune di Trento tramite ANPR; se l'utente non è residente, la registrazione deve essere bloccata.
  \item L'utente deve poter inserire un indirizzo email, una password e un nome utente per i futuri accessi.
  \item L'utente deve poter selezionare la preferenza per la ricezione delle notifiche email (abilita/disabilita).
  \item Il sistema deve inviare una email di verifica contenente un link per l'attivazione dell'account.
  \item L'attivazione dell'account deve essere completata entro 10 minuti dalla ricezione dell'email, pena l'invalidazione della richiesta.
}
{
  \item Integrazione dei servizi di autenticazione SPID e CIE.
  \item Implementazione del connettore ANPR per la verifica della residenza.
  \item Sviluppo del modulo frontend per l'inserimento di email, password, username e preferenze notifiche.
  \item Implementazione controlli di unicità per email e username nel database.
  \item Configurazione del servizio SMTP per l'invio delle email transazionali.
  \item Implementazione della logica di scadenza del token di verifica (10 minuti).
}

\section{Espressione del voto~\ref{rf:voto}} \label{us:voto}
\userstory
{
  Come utente, voglio esprimere un voto (favorevole o contrario) sulle proposte e partecipare ai sondaggi attivi, per contribuire alla definizione delle priorità comunali.
}
{
  \item Il sistema deve consentire di esprimere un voto positivo o negativo sulle proposte in stato "pubblicata" o "in valutazione".
  \item Il sistema deve consentire di rispondere ai quesiti presenti nei sondaggi attivi.
  \item Ogni utente può esprimere un solo voto per ciascuna proposta o per ciascun quesito dei sondaggi.
  \item Il voto deve poter essere modificato dall'utente fino alla chiusura del sondaggio o alla definizione dell'esito della proposta.
  \item Il sistema deve registrare il voto in forma anonima, rendendo impossibile risalire all'identità del votante tramite i report pubblici.
  \item Il sistema deve aggiornare il conteggio dei voti e visualizzare l'eventuale feedback all'utente.
}
{
  \item Implementazione componenti UI per la votazione (bottoni favorevole/contrario e opzioni sondaggio).
  \item Sviluppo API backend per la ricezione e validazione del voto (controllo stato proposta).
  \item Progettazione dello schema database per garantire l'unicità del voto per utente mantenendo l'anonimato (separazione identità/preferenza).
  \item Implementazione logica per la modifica del voto.
  \item Aggiornamento real-time dei contatori visibili sulla proposta.
}

\section{Gestione dei preferiti~\ref{rf:preferiti}} \label{us:preferiti}
\userstory
{
  Come utente, voglio aggiungere o rimuovere proposte e sondaggi dai miei preferiti, per poter accedere rapidamente ai contenuti di mio interesse e ricevere notifiche automatiche sui loro aggiornamenti di stato.
}
{
  \item Il sistema deve consentire di aggiungere o rimuovere una proposta o un sondaggio dai preferiti agendo su un apposito indicatore visivo.
  \item L'azione di aggiunta/rimozione deve essere disponibile sia dalla lista generale delle proposte/sondaggi che dalla pagina di dettaglio del singolo elemento.
  \item Il sistema deve fornire una sezione dedicata "Preferiti" dove l'utente può consultare l'elenco aggiornato di tutti gli elementi salvati.
  \item Se un elemento nei preferiti viene rimosso o archiviato dal sistema, esso deve essere automaticamente rimosso dalla lista dei preferiti dell'utente, eventualmente con una notifica informativa.
  \item L'aggiornamento dello stato "preferito" deve essere immediato e persistente tra le sessioni.
}
{
  \item Implementazione del componente UI (attivo/disattivo) sulle card e nelle pagine di dettaglio.
  \item Sviluppo API backend per il toggle dello stato di preferito (add/remove).
  \item Sviluppo della pagina frontend "I miei Preferiti".
  \item Implementazione della logica di pulizia dei preferiti in caso di cancellazione della proposta originale.
}

\section{Segnalazione contenuti~\ref{rf:segnalazione}} \label{us:segnalazione}
\userstory
{
  Come utente, voglio poter segnalare i contenuti pubblicati (proposte o modifiche) ritenuti inappropriati specificandone la motivazione, affinché i moderatori possano verificarli e garantire un ambiente di discussione sicuro e rispettoso.
}
{
  \item Il sistema deve mostrare un comando "Segnala contenuto" su ogni proposta o modifica pubblicata.
  \item Il sistema deve presentare un modulo che richiede obbligatoriamente la selezione di una motivazione da un elenco predefinito (es. incitamento all'odio, informazioni false, spam, violazione privacy).
  \item L'utente deve poter inserire facoltativamente un testo descrittivo aggiuntivo per chiarire il motivo della segnalazione.
  \item Il sistema deve verificare che il contenuto segnalato sia ancora disponibile prima di registrare l'operazione.
  \item Una volta confermata, la segnalazione deve essere inserita automaticamente nella coda di revisione dei moderatori.
  \item L'utente deve ricevere un feedback visivo di conferma dell'avvenuto invio della segnalazione.
}
{
  \item Implementazione del bottone "Segnala" nell'interfaccia delle proposte.
  \item Creazione della modale con form contenente la lista predefinita di motivazioni (enum) e campo testo libero.
  \item Sviluppo API backend per la creazione dell'oggetto "Report" collegato a User e Content.
  \item Implementazione della logica di inserimento nella coda di revisione.
  \item Gestione degli errori nel caso in cui il contenuto sia stato rimosso concorrentemente.
}

\section{Espressione Endorsement~\ref{rf:endorsement}}
\userstory
{
  Come rappresentante di un'associazione, voglio poter esprimere o rimuovere un endorsement ufficiale sulle proposte pubblicate, per manifestare pubblicamente il sostegno della mia organizzazione accrescendone la rilevanza agli occhi della cittadinanza.
}
{
  \item Il sistema deve consentire alle sole utenze di tipo "Associazione" di esprimere un endorsement sulle proposte in stato "pubblicata".
  \item Ogni associazione può esprimere un solo endorsement per ciascuna proposta.
  \item L'azione di endorsement deve generare visibilmente un'etichetta sulla proposta riportante il nome dell'associazione.
  \item L'associazione deve poter rimuovere un endorsement precedentemente assegnato; la rimozione deve comportare la cancellazione immediata dell'etichetta visuale.
}
{
  \item Implementazione verifica permessi (solo ruolo Associazione).
  \item Sviluppo API backend per gestione endorsement.
  \item Aggiornamento modello dati Proposta per collegare la lista degli endorser.
  \item Sviluppo componente UI "Etichetta Endorsement" nella visualizzazione proposta.
  \item Implementazione pulsante toggle "Esprimi/Rimuovi Endorsement".
}

\section{Ordinamento proposte~\ref{rf:ordinamentoproposte}}
\userstory{
  Come utente, voglio poter ordinare le proposte che visualizzo in base a data o numero di voti, così da poter analizzare più facilmente quelle rilevanti per me.
}
{
  \item Il sistema deve consentire l’ordinamento per data di pubblicazione.
  \item Il sistema deve consentire l’ordinamento per numero totale di voti.
  \item L’utente deve poter invertire l’ordine (crescente/decrescente).
  \item L’ordinamento deve essere applicato immediatamente all’elenco visibile.
}
{
  \item Implementazione UI dei controlli di ordinamento (dropdown o pulsanti).
  \item API backend per applicare criteri di ordinamento.
  \item Aggiornamento del componente elenco proposte.
}

\section{Rimozione contenuti~\ref{rf:rimozionecontenuti}}
\userstory{
  Come moderatore, voglio poter eliminare i contenuti che non rispettano le regole della piattaforma, così da preservare un ambiente sicuro e rispettoso.
}
{
  \item Il sistema deve permettere ai moderatori di rimuovere contenuti non conformi.
  \item La rimozione deve richiedere la selezione di una motivazione.
  \item Il contenuto eliminato non deve più essere visibile al pubblico.
  \item La rimozione deve essere registrata con motivazione, data e moderatore responsabile.
}
{
  \item Interfaccia “Rimuovi contenuto” visibile esclusivamente ai moderatori.
  \item API backend per l’eliminazione del contenuto.
  \item Aggiornamento del database con registrazione dell’azione di moderazione.
  \item Aggiornamento dell’elenco dei contenuti pubblici.
}

\section{Applicazione limitazioni utenti~\ref{rf:moderazioneutenti}}
\userstory
{
  Come moderatore, voglio applicare limitazioni temporanee agli account che violano le linee guida, così da prevenire comportamenti scorretti e tutelare la comunità.
}
{
  \item Il sistema deve consentire ai moderatori di applicare limitazioni temporanee ai cittadini.
  \item Per le associazioni, il sistema deve inoltrare la richiesta all’amministrazione.
  \item Ogni limitazione deve generare l'invio di un’email all’utente con motivazione e durata.
  \item Un account limitato deve essere bloccato nelle azioni previste dalla limitazione.
}
{
  \item Interfaccia per la gestione delle limitazioni accessibile ai moderatori.
  \item API per applicare la limitazione o inviare la richiesta all’amministrazione.
  \item Sistema di notifica email per informare gli utenti coinvolti.
  \item Applicazione di limitazioni dei permessi (azioni bloccate).
}

\section{Pubblicazione sondaggi~\ref{rf:sondaggi}}
\userstory
{
  Come amministratore, voglio poter pubblicare sondaggi, così da raccogliere in modo strutturato l’opinione dei cittadini su temi di interesse pubblico.
}
{
  \item Il sistema deve permettere la creazione di un sondaggio con titolo, descrizione, categoria, periodo di validità e quesiti.
  \item La pubblicazione deve avvenire immediatamente dopo la conferma dell’amministratore.
  \item Gli utenti che hanno attivato le notifiche devono essere avvisati della pubblicazione.
  \item Il sondaggio deve chiudersi automaticamente alla data prevista.
}
{
  \item Modulo UI dedicato alla creazione dei sondaggi.
  \item API backend per creazione e pubblicazione del sondaggio.
  \item Configurazione CRON/worker per chiusura automatica.
  \item Invio delle notifiche email agli utenti interessati.
}

\section{Scaricare report~\ref{rf:report}}
\userstory
{
  Come amministratore, voglio poter scaricare report sull’attività della piattaforma, così da analizzare l’andamento del sistema e supportare le decisioni strategiche.
}
{
  \item Il sistema deve consentire il download dei report nei formati previsti, ad esempio CSV.
  \item Il sistema deve generare report aggregati e anonimizzati.
  \item Il sistema deve offrire almeno due tipi di report:
  \begin{itemize}
    \item attività per proposta;
    \item attività complessiva.
  \end{itemize}
  \item Il download deve avvenire in locale senza esposizione di dati sensibili.
}
{
\item Interfaccia per la selezione del tipo di report.
\item API backend per la generazione del report.
\item Modulo di esportazione nel formato richiesto.
}

\end{usscope}
\chapter{Design Front End}

\section*{Introduzione}
In questo capitolo vengono presentati gli elementi visuali e progettuali che definiscono l’esperienza utente della piattaforma. La sezione include i mockup delle principali interfacce, corredati dai riferimenti ai requisiti pertinenti, oltre a brevi descrizioni testuali pensate per chiarire eventuali aspetti non immediatamente evidenti dalle sole immagini.
\chapter{User Flow}

\section*{Introduzione}
In questo capitolo presentiamo gli user flow derivati dalle User Stories individuate precedentemente. Gli schemi illustrano in modo diretto le azioni concrete che permettono ad ogni utente di soddisfare i propri obiettivi all’interno dell’applicazione.

\vspace{1cm}

\begin{figure}[h!]
	\centering
	\includegraphics[scale=0.7]{img/flow/Votazione.png}
	\caption{User Flow collegato alle User Story \ref{us:registrazione}, \ref{us:voto}}
\end{figure}

\begin{figure}[h!]
	\centering
	\includegraphics[scale=0.9]{img/flow/Preferiti.png}
	\caption{User Flow collegato alla User Story \ref{us:preferiti}}
\end{figure}

\begin{figure}[h!]
	\centering
	\includegraphics[scale=0.85]{img/flow/Segnalazione.png}
	\caption{User Flow collegato alla User Story \ref{us:segnalazione}}
\end{figure}

\begin{figure}[h!]
	\centering
	\includegraphics[scale=0.85]{img/flow/Endorsement.png}
	\caption{User Flow collegato alle User Story \ref{us:endorsement}, \ref{us:ordinamento}}
\end{figure}

\begin{figure}[h!]
	\centering
	\includegraphics[scale=0.85]{img/flow/Moderazione.png}
	\caption{User Flow collegato alle User Story \ref{us:rimozione}, \ref{us:limitazione}}
\end{figure}

\begin{figure}[h!]
	\centering
	\includegraphics[scale=0.85]{img/flow/Dashboard.png}
	\caption{User Flow collegato alle User Story \ref{us:sondaggi}, \ref{us:report}}
\end{figure}

\end{document}
