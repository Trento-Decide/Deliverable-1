\chapter{Stato dell'Arte nella Civic Technology e Contesto di Riferimento}

\section{Introduzione: Dal Contesto Globale Alla Soluzione Locale}

Il progetto \textit{Trento Decide} non nasce in un vuoto, ma si inserisce in un movimento globale di innovazione democratica noto come civic technology. L'obiettivo di questo capitolo è duplice: in primo luogo, analizzare il vasto panorama delle piattaforme partecipative esistenti per comprendere modelli e approcci differenti. In secondo luogo, approfondire uno dei casi di studio più emblematici --- la piattaforma Decidim --- per derivarne lezioni apprese e sfide ricorrenti.

Questa analisi comparata non serve solo a validare le scelte progettuali di \textit{Trento Decide}, ma anche ad anticiparne criticamente i rischi, fondando i requisiti su evidenze empiriche anziché su mere assunzioni teoriche.

\section{Un Panorama Globale Di Piattaforme Partecipative}

L'adozione di strumenti digitali per coinvolgere i cittadini è una pratica consolidata a livello mondiale, con un'ampia varietà di modelli che spaziano dalla scala comunale a quella transnazionale.

\begin{itemize}
    \item \textbf{Scala Transnazionale e Europea:} L'Unione Europea stessa ha sperimentato ampiamente con la partecipazione digitale. La Conferenza sul Futuro dell'Europa (2021-2022) ha utilizzato una piattaforma multilingue per raccogliere idee da cittadini di tutti gli stati membri, dimostrando la fattibilità di processi deliberativi su larga scala. Analogamente, iniziative come la piattaforma per le Missioni UE invitano i cittadini a contribuire su temi specifici come la lotta al cancro o l'adattamento climatico.
    \item \textbf{Scala Nazionale:} Molti governi nazionali hanno istituzionalizzato propri canali di partecipazione. Il Brasile, pioniere dei bilanci partecipativi, ha lanciato la piattaforma Brasil Participativo per definire programmi governativi pluriennali. In Europa, la Francia offre canali dedicati per le petizioni all'Assemblea Nazionale e al Senato. Il Belgio utilizza MonOpinion per le consultazioni pubbliche federali e l'Italia stessa ha la piattaforma ParteciPa, gestita dal Dipartimento per le riforme istituzionali.
    \item \textbf{Scala Locale:} È a livello di città che si trovano gli esperimenti più dinamici e direttamente comparabili a \textit{Trento Decide}. Oltre al caso di Barcellona, Helsinki con la sua piattaforma OmaStadi gestisce un bilancio partecipativo di quasi 9 milioni di euro. In Cile, Plaza Pública è un altro esempio di strumento utilizzato per la partecipazione a livello locale.
\end{itemize}

Questo ecosistema eterogeneo dimostra che non esiste un modello unico, ma un insieme di strumenti adattati a contesti e obiettivi diversi: dalla raccolta di idee (consultazioni) alla richiesta di azione (petizioni), fino alla co-decisione sull'allocazione di risorse (bilanci partecipativi). Tra questi, la piattaforma Decidim emerge come un caso di studio particolarmente significativo per la sua complessità, la sua adozione internazionale e la profondità dell'analisi accademica che ha generato.

\section{Il Caso di Studio di Eccellenza: Decidim in Catalogna}
La piattaforma Decidim ("Decidiamo" in catalano), sviluppata a partire dal 2016 dal Comune di Barcellona, è oggi considerata un benchmark nel settore. Una valutazione del 2022 dell'organizzazione internazionale People Powered l'ha identificata come la piattaforma più performante tra le 30 soluzioni complesse di partecipazione digitale analizzate a livello mondiale. Il suo impatto è supportato da dati quantitativi impressionanti.

\subsection{Un Successo Misurabile: l'Impatto sul Piano Strategico di Barcellona}
L'adozione di Decidim per il Piano Strategico Municipale (PAM) 2016-2019 di Barcellona ha prodotto risultati straordinari:
\begin{itemize}
    \item \textbf{Accoglimento delle Proposte:} Delle oltre 10.800 proposte presentate dai cittadini, ben 8.160 (il 75\%) sono state accolte e integrate nel piano d'azione finale.
    \item \textbf{Impatto sul Budget:} L'esecuzione del piano strategico, definito tramite questo processo partecipativo, ha allocato quasi il 90\% del budget totale del Comune di Barcellona nel quadriennio.
    \item \textbf{Esecuzione e Accountability:} A maggio 2019, l'89,1\% del piano era stato eseguito, con uno stato di avanzamento costantemente monitorato e reso pubblico tramite il modulo di accountability della piattaforma.

    Questi numeri dimostrano che Decidim è uno strumento in grado di influenzare concretamente le politiche pubbliche e l'allocazione di immense risorse finanziarie.

\subsubsection{Le Tre Dimensioni della Piattaforma: Trasparenza, Partecipazione e Deliberazione}
L'analisi del suo utilizzo rivela una gerarchia di funzioni, con punti di forza e debolezze che costituiscono una lezione fondamentale per \textit{Trento Decide}.
\begin{enumerate}
    \item \textbf{Trasparenza (Obiettivo Primario e Raggiunto):}
    \begin{itemize}
        \item Il 97\% dei funzionari pubblici concorda che la piattaforma rende i processi più trasparenti.
        \item È utilizzata principalmente come strumento per pubblicare ogni fase del processo (calendari, regolamenti, stato di avanzamento), garantendo tracciabilità e accountability.
        \item \textbf{Criticità:} Mantenere alti livelli di trasparenza (es. fornire feedback motivati a ogni proposta) rappresenta un costo significativo in termini di tempo e risorse per l'amministrazione.
    \end{itemize}
    \item \textbf{Partecipazione (Obiettivo Raggiunto con Qualifiche):}
    \begin{itemize}
        \item L'83\% dei funzionari la ritiene utile per raccogliere proposte dai cittadini.
        \item Tuttavia, solo il 50\% ritiene che trasferisca un reale potere decisionale ai cittadini, mostrando come le istituzioni tradizionali tendano a mantenere il controllo.
        \item Viene vista come un complemento e non un sostituto della partecipazione in presenza (il 63\% concorda), aiutando anzi a organizzarla meglio.
        \item I tassi di registrazione variano notevolmente, dall'1\% a oltre il 15\% della popolazione, con i comuni di medie dimensioni (50.000-220.000 abitanti) che ottengono i risultati migliori.
    \end{itemize}
    \item \textbf{Deliberazione (Obiettivo Largamente Mancato):}
        \begin{itemize}
            \item Solo il 20\% dei funzionari ritiene che la piattaforma favorisca un reale dibattito online tra cittadini.
            \item I moduli di dibattito e commento sono scarsamente utilizzati; spesso i comuni preferiscono non abilitarli per incapacità di gestire il flusso o perché non ne vedono l'utilità, privilegiando i workshop in presenza per la fase deliberativa.
            \item Anche in casi di grande attività, come a Barcellona, oltre la metà delle proposte non riceve alcun commento, e la maggior parte dei commenti sono reazioni dirette, non conversazioni.
        \end{itemize}
\end{enumerate}
\end{itemize}

\section{Le Sfide del Mondo Reale: Lezioni per "Trento Decide"}
Lo studio evidenzia tre ostacoli cruciali che qualsiasi progetto di civic tech deve affrontare, e che hanno direttamente ispirato i requisiti di "Trento Decide".

\begin{enumerate}
    \item \textbf{Volontà Politica e Rischio di Abbandono:} Il successo della piattaforma è strettamente legato al supporto del governo in carica. A Badalona e Madrid, un cambio di maggioranza politica ha portato all'immediato abbandono delle piattaforme, identificate con l'amministrazione precedente.
    \begin{itemize}
        \item \textbf{Lezione per Trento Decide:}  Il progetto deve essere percepito come uno strumento istituzionale e non politico, garantendo neutralità e costruendo un consenso trasversale. La sua adozione deve essere radicata nei processi amministrativi per sopravvivere ai cicli elettorali.
    \end{itemize}
    \item \textbf{Digital Divide e Inclusività:} Sebbene le piattaforme digitali possano attrarre nuovi profili di partecipanti (es. professionisti, donne con figli), il divario digitale rimane una preoccupazione centrale. Le amministrazioni catalane hanno messo in campo contromisure come punti di facilitazione fisica e unità mobili.
    \begin{itemize}
        \item \textbf{Lezione per Trento Decide:}   È fondamentale non solo riconoscere il problema, ma integrarne la mitigazione nel progetto.
    \end{itemize}
    \item \textbf{Resistenza degli Attori Intermedi:} Le associazioni locali, tradizionalmente intermediarie tra cittadini e potere, possono percepire la partecipazione diretta come una minaccia al loro ruolo e alla loro influenza. Questo ha generato resistenze, portando alcuni comuni a introdurre meccanismi per "privilegiare" le proposte collettive (es. richiedendo meno endorsement).
    \begin{itemize}
        \item \textbf{Lezione per Trento Decide:}   Il sistema deve bilanciare la disintermediazione con il riconoscimento del valore del tessuto associativo.
    \end{itemize}
\end{enumerate}


\section{Applicabilità al Contesto Italiano e Trentino}
Perché un progetto come \textit{Trento Decide} può avere successo proprio in questo contesto? La risposta risiede nell'intersezione tra un bisogno percepito e l'esistenza delle condizioni abilitanti.

\subsection{Il Bisogno di Trasparenza e Partecipazione in Italia}
Dati recenti mostrano un chiaro scollamento tra cittadini e istituzioni, terreno fertile per strumenti che mirano a ricostruire la fiducia.

\begin{itemize}
    \item \textbf{Bassa Fiducia nelle Istituzioni:} Secondo il rapporto ISTAT BES 2022 (Benessere Equo e Sostenibile), in Italia solo il 9,2\% dei cittadini dichiara di avere molta fiducia nei partiti politici e il 27,7\% nel parlamento. Anche le istituzioni locali, pur godendo di maggior fiducia, necessitano di strumenti per rafforzare il legame con la cittadinanza.
    \item \textbf{Domanda di Trasparenza:} Iniziative come "Trento Decide" rispondono direttamente a questa sfiducia, offrendo un canale verificabile e trasparente, che è emerso come l'uso primario e di maggior successo della piattaforma Decidim.
\end{itemize}

\subsection{Le Condizioni Abilitanti nel Contesto Trentino}
Il successo non dipende solo dal bisogno, ma anche dalla maturità del contesto.

\begin{itemize}
    \item \textbf{Elevata Digitalizzazione:} Il Trentino-Alto Adige si classifica costantemente ai vertici in Italia per competenze digitali e utilizzo di Internet. Nel 2023, la provincia di Trento era la prima in Italia per indice di digitalizzazione dell'economia e della società (DESI). Questo riduce la barriera d'ingresso per un'ampia fascia di popolazione.
    \item \textbf{Diffusione dell'Identità Digitale:} L'Italia ha visto un'adozione massiccia di SPID e CIE, con oltre 36 milioni di italiani in possesso di SPID a fine 2023. Questo ecosistema di autenticazione forte e già diffuso è il pilastro su cui \textit{Trento Decide} fonda la sicurezza e la verificabilità della partecipazione, un vantaggio enorme rispetto ai primi esperimenti di civic tech.
    \item \textbf{Tradizione Civica e Dimensioni Ottimali:} Trento, come comune di medie-grandi dimensioni con un forte tessuto associativo e un alto livello di istruzione, rientra perfettamente nella tipologia di città in cui, secondo lo studio su Decidim, queste piattaforme hanno il maggior successo.
\end{itemize}

In conclusione, l'analisi dello stato dell'arte non solo conferma la validità dell'approccio di \textit{Trento Decide}, ma dimostra come i suoi requisiti siano stati attentamente calibrati sulle lezioni apprese dai pionieri del settore. Il contesto italiano e trentino, lungi dall'essere un ostacolo, presenta le condizioni ideali per superare alcune delle sfide storiche della partecipazione digitale e per realizzare un modello di governance realmente più aperto, trasparente ed efficace.