\chapter{Design Front End}

\section{Introduzione}
In questo capitolo vengono presentati gli elementi visuali e progettuali che definiscono l’esperienza utente della piattaforma. La sezione include i mockup delle principali interfacce, corredati dai riferimenti ai requisiti pertinenti, oltre a brevi descrizioni testuali pensate per chiarire eventuali aspetti non immediatamente evidenti dalle sole immagini.

% -------------------------------------------------------------------
\section{Login}
\begin{figure}[h!]
    \centering
    \includegraphics[width=0.45\textwidth]{img/ui/login.png}
    \caption{Schermata di accesso: credenziali locali e opzioni SPID / CIE}
\end{figure}

La schermata di login offre due modalità principali di accesso: (i) credenziali locali (email + password) e (ii) autenticazione tramite SPID o CIE\@. Sono anche presenti i collegamento per il recupero password e la registrazione.

\paragraph{Requisiti soddisfatti.}
\begin{itemize}
    \item \textbf{\ref{rf:login}: Login}\\  
    Supporto per login con credenziali locali e per autenticazione tramite SPID/CIE.

    \item \textbf{\ref{rnf:sicurezza}: Sicurezza}\\  
    Tutte le comunicazioni di autenticazione sono cifrate (TLS). Non vengono esposti dati sensibili e sono incluse protezioni anti-bruteforce lato backend.
    
    \item \textbf{\ref{rnf:lingua}: Lingua e Internazionalizzazione}\\
    Il testo e messaggi di errore sono traducibili in tutte le lingue supportate.
\end{itemize}

\paragraph{Nota.}  
Per l’accesso tramite SPID/CIE il pulsante reindirizza al provider di identità.

% -------------------------------------------------------------------
\section{Registrazione}
\begin{figure}[h!]
    \centering
    \includegraphics[width=0.45\textwidth]{img/ui/registrati-1.png}
    \includegraphics[width=0.45\textwidth]{img/ui/registrati-2.png}
    \caption{Schermate di registrazione: SPID/CIE e impostazione credenziali locali}
\end{figure}

La procedura di registrazione prevede l’autenticazione tramite SPID o CIE come passaggio iniziale. Al termine della verifica dell'identità il sistema reindirizza l'utente verso l'impostazione di credenziali locali (email, password e username) e la scelta della preferenza sulle notifiche.

\paragraph{Requisiti soddisfatti.}
\begin{itemize}
    \item \textbf{\ref{rf:registrazionecittadini}: Registrazione del cittadino}\\  
    Registrazione basato su SPID/CIE, successiva impostazione delle credenziali locali e invio dell'email di conferma con scadenza token (10 minuti).

    \item \textbf{\ref{rf:prefnotifiche}: Ricezione notifiche}\\
    Scelta del consenso alla ricezione di notifiche durante la registrazione.

    \item \textbf{\ref{rnf:sicurezza}: Sicurezza}\\  
    Validazione lato client e server dei campi (unicità email/username, complessità password), crittografia dei dati sensibili.

    \item \textbf{\ref{rnf:compatibilita}: Compatibilità}\\
    Il form è responsive.

    \item \textbf{\ref{rnf:lingua}: Internazionalizzazione e lingua}\\
    Il form è completamente traducibile, come i messaggi di errore e le istruzioni.
\end{itemize}

% -------------------------------------------------------------------

\section{Homepage}
\begin{figure}[h!]
    \centering
    \includegraphics[width=0.45\textwidth]{img/ui/homepage.png}
    \caption{Homepage personalizzata dell'utente autenticato}
\end{figure}

La homepage accoglie l'utente con un messaggio 
personalizzato \emph{``Bentornato \texttt{@username}!''}
e presenta una serie di sezioni utili come: aggiornamenti recenti, proposte viste di recente, sondaggi attivi ed esplorazione per categoria.
La struttura a blocchi rende la schermata immediatamente leggibile e orientata alle attività più frequenti.

\paragraph{Requisiti soddisfatti.}
\begin{itemize}
    \item \textbf{\ref{rf:preferiti}: Gestione dei preferiti}\\
    La sezione ``Proposte viste di recente'' utilizza le informazioni derivate dalle interazioni dell’utente.
    
    \item \textbf{\ref{rf:filtroproposte}: Filtraggio delle proposte}\\
    La sezione ``Esplora per categoria'' e la barra di ricerca facilita la navigazione all’interno della piattaforma.

    \item \textbf{\ref{rnf:compatibilita}: Compatibilità}\\
    Il layout è pienamente responsive e conforme agli standard web elencati nel relativo requisito.

    \item \textbf{\ref{rnf:lingua}: Internazionalizzazione e lingua}\\
    Tutti i testi dell’interfaccia sono traducibili automaticamente e non, come previsto dal relativo requisito non funzionale.
\end{itemize}

% -------------------------------------------------------------------
\section{Profilo Utente}
\begin{figure}[h!]
    \centering
    \includegraphics[width=0.45\textwidth]{img/ui/profilo.png}
    \includegraphics[width=0.45\textwidth]{img/ui/preferiti.png}
    \caption{Schermata del profilo utente: dati personali e preferiti}
\end{figure}

La schermata del profilo è suddivisa in due blocchi:  
(i) gestione dei dati personali e delle credenziali,  
(ii) elenco dei propri contenuti preferiti.  

L'interfaccia distingue chiaramente i dati modificabili (es. email, password, username) da quelli provenienti da SPID/CIE, mostrati ma non modificabili.

\paragraph{Requisiti soddisfatti.}
\begin{itemize}
    \item \textbf{\ref{rf:gestioneprofiloedati}: Gestione del profilo}\\
    L’utente può aggiornare email, password e nome utente tramite i moduli dedicati.

    \item \textbf{\ref{rf:preferiti}: Gestione dei preferiti}\\
    Il secondo blocco presenta la lista costantemente aggiornata dei contenuti salvati dall'utente.

    \item \textbf{\ref{rnf:lingua}: Internazionalizzazione e lingua}\\
    La schermata supporta la traduzione dell’interfaccia come previsto dal requsito non funzionale.
\end{itemize}

% -------------------------------------------------------------------
\section{Visualizzazione di una proposta}
\begin{figure}[h!]
    \centering
    \includegraphics[width=0.45\textwidth]{img/ui/proposta.png}
    \caption{Pagina di dettaglio di una proposta}
\end{figure}

La schermata mostra il contenuto completo di una proposta: titolo, descrizione, categoria, autore, data di pubblicazione, voti, mappa, allegati e altri strumenti di interazione nascosti dal pulsante a 3 punti come segnalazione del contenuto, aggiunta ai preferiti, traduzione.  
Sono inoltre presenti le sezioni dedicate alle modifiche collaborative e alla cronologia delle versioni.

\paragraph{Requisiti soddisfatti.}
\begin{itemize}
    \item \textbf{\ref{rf:modificacollaborativa}: Modifica collaborativa}\\
    L’interfaccia include i pulsanti per avviare una modifica condivisa o visualizzare la cronologia.

    \item \textbf{\ref{rf:versionamento}: Versionamento}\\
    L’indicazione ``Ultima modifica'' introduce informazioni sulla versione corrente della proposta.

    \item \textbf{\ref{rnf:lingua}: Internazionalizzazione e lingua}\\
    L’interfaccia è completamente traducibile, mentre i contenuti generati dagli utenti rimangono nella lingua originale e sono opzionalmente tradotti automaticamente.
\end{itemize}

% -------------------------------------------------------------------
\section{Modifica Proposta (autore)}
\begin{figure}[h!]
    \centering
    \includegraphics[width=0.45\textwidth]{img/ui/modifica-proposta.png}
    \caption{Modulo di modifica di una proposta esistente}
\end{figure}

In questa schermata l’autore può aggiornare una proposta già pubblicata.  
Tutti i campi sono precompilati con i valori correnti, facilitando la modifica mirata su singoli elementi senza dover riscrivere l’intera proposta.

\paragraph{Requisiti soddisfatti.}
\begin{itemize}
    \item \textbf{\ref{rf:modificaproposta}: Modifica proposta}\\
    Sono modificabili titolo, descrizione, categoria, luogo (quando previsto) e allegati.

    \item \textbf{\ref{rnf:lingua}: Internazionalizzazione e lingua}\\
    L’interfaccia rispetta i requisiti di internazionalizzazione.
\end{itemize}

% -------------------------------------------------------------------
\section{Creazione di una proposta}
\begin{figure}[h!]
    \centering
    \includegraphics[width=0.45\textwidth]{img/ui/crea-proposta.png}
    \caption{Modulo di creazione di una nuova proposta}
\end{figure}

Questa schermata mostra la struttura del modulo di modifica, ma con campi inizialmente vuoti.  
L’utente può compilare tutti i campi richiesti e scegliere se salvare una bozza o procedere alla pubblicazione con richiesta di una ulteriore conferma.

\paragraph{Requisiti soddisfatti.}
\begin{itemize}
    \item \textbf{\ref{rf:creazioneproposta}: Creazione di una proposta}\\
    Sono presenti tutti i campi obbligatori (titolo, descrizione, luogo quando richiesto, categoria) e i campi relativi alla categoria selezionata.

    \item \textbf{\ref{rf:salvataggiobozza}: Salvataggio bozza}\\
    Il pulsante ``Salva bozza'' consente di archiviare la proposta in uno stato non pubblico.

    \item \textbf{\ref{rf:scartarebozza}: Scartare proposta}\\
    Il pulsante ``Elimina bozza'' permette di annullare completamente la compilazione.

    \item \textbf{\ref{rf:pubblicazioneproposta}: Pubblicare proposta}\\
    ``Pubblica proposta'' è disponibile solo quando tutti i campi obbligatori sono completi.

    \item \textbf{\ref{rnf:lingua}: Internazionalizzazione e lingua}\\
    L’intera schermata è traducibile secondo le lingue supportate.
\end{itemize}