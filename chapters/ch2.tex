\chapter{Requisiti Funzionali}

\renewcommand{\thesubsection}{\textbf{RF\arabic{section}.\arabic{subsection}}}

\setlength{\parskip}{0.4em}
\setlength{\parindent}{0pt}

\section{Introduzione}
In questa sezione vengono descritti i requisiti funzionali del sistema \textit{Trento Decide}.  
Essi definiscono le funzioni che la piattaforma deve offrire per supportare la partecipazione civica e la collaborazione tra cittadini e amministrazione comunale.  
Ogni requisito è identificato con il prefisso \textbf{RF}, seguito da un numero progressivo e da eventuali sottosezioni tematiche.

% ====================================================
% RF1
% ====================================================
\section{Gestione utenti e autenticazione}

\subsection{Registrazione e login}
Il sistema deve consentire ai cittadini, ai moderatori e agli amministratori comunali di registrarsi e accedere tramite autenticazione tradizionale (email e password) o tramite SPID/CIE.  
Durante il login, l’utente deve poter scegliere la modalità di accesso preferita.

\subsection{Gestione profilo utente}
Il sistema deve permettere a ciascun utente di visualizzare e modificare i propri dati personali.  
L’utente deve poter aggiornare la password.  
Ogni modifica deve essere tracciata nel registro delle azioni.

\subsection{Gestione ruoli e permessi}
Il sistema deve gestire i seguenti ruoli predefiniti:
\begin{itemize}
    \item \textbf{Cittadino}: crea proposte, partecipa a consultazioni e votazioni;
    \item \textbf{Moderatore}: supervisiona e convalida i contenuti;
    \item \textbf{Amministratore comunale}: gestisce processi e fornisce risposte ufficiali;
    \item \textbf{Associazione o comitato}: presenta proposte collettive e fornisce endorsement.
\end{itemize}
Ogni ruolo deve avere un insieme di permessi configurabile dall’amministrazione comunale.

\subsection{Gestione account e sicurezza}
Il sistema deve consentire il recupero della password, la disattivazione dell’account e la visualizzazione della cronologia di accessi.  
Deve inoltre permettere la configurazione delle preferenze di notifica per canale (email, push, App IO).

% ====================================================
% RF2
% ====================================================
\section{Gestione proposte cittadine}

\subsection{Creazione proposta}
Il sistema deve consentire ai cittadini autenticati di creare una proposta compilando un modulo composto da campi specifici per la categoria selezionata.  
I campi obbligatori e facoltativi per ciascuna categoria devono essere configurabili dall’amministrazione comunale.  
Il sistema deve salvare automaticamente una prima versione con stato “Bozza” e rendere pubblica la proposta solo dopo la pubblicazione da parte dell’utente.

\subsection{Modifica collaborativa}
Il sistema deve consentire la modifica collaborativa delle proposte, mantenendo uno storico delle versioni contenente autore, data e descrizione del cambiamento.  
Deve essere possibile confrontare due versioni successive e ripristinare una versione precedente.

\subsection{Stato e tracciabilità delle proposte}
Il sistema deve gestire lo stato di ogni proposta aggiornandolo automaticamente in base alle azioni dell’utente o dell’amministrazione comunale (creazione, pubblicazione, valutazione, approvazione, implementazione).  
Deve inoltre mostrare lo stato corrente e la cronologia completa dei cambiamenti associati a ciascuna proposta.

\subsection{Endorsement e raccolta firme}
Il sistema deve consentire l’espressione di endorsement digitali a favore delle proposte da parte di cittadini o associazioni registrate.  
Deve inoltre permettere di registrare le firme fisiche raccolte offline e comunicate all’amministrazione comunale.

\subsection{Proposte collettive}
Le associazioni registrate devono poter presentare proposte collettive tramite un canale dedicato.  
Tali proposte devono essere etichettate come collettive e soggette alle stesse regole di validazione delle proposte individuali.  
Le proposte collettive devono avere un peso maggiore nel calcolo del ranking o nei processi di valutazione, con coefficiente configurabile dall’amministrazione comunale.

% ====================================================
% RF3
% ====================================================
\section{Votazioni e consultazioni}

\subsection{Votazione delle proposte}
Il sistema deve consentire agli utenti di esprimere un voto positivo o negativo su una proposta attiva.  
Il conteggio dei voti deve essere aggiornato in tempo reale.

\subsection{Consultazioni pubbliche e sondaggi}
Il sistema deve consentire all’amministrazione comunale di creare consultazioni e sondaggi tematici, raccogliendo risposte dai cittadini.  
I risultati devono essere esportabili in formato CSV e visualizzabili pubblicamente tramite grafici o tabelle aggregate.

\subsection{Algoritmo di ranking}
Il sistema deve calcolare automaticamente un punteggio di rilevanza per ciascuna proposta in base a parametri configurabili dall'amministrazione comunale.  
Le proposte devono poter essere ordinate per punteggio, data o categoria.

\subsection{Report votazioni}
Il sistema deve generare report delle votazioni contenenti almeno i seguenti indicatori: numero totale di votanti, distribuzione territoriale e fasce d’età.  
Ulteriori indicatori devono poter essere aggiunti tramite configurazione amministrativa.

% ====================================================
% RF4
% ====================================================
\section{Moderazione e qualità dei contenuti}

\subsection{Moderazione automatica}
Il sistema deve analizzare automaticamente i contenuti generati dagli utenti (testi e materiali multimediali) per rilevare linguaggio inappropriato, spam o duplicati.  
In caso di rilevamento, il sistema deve
segnalare l’elemento ai moderatori o sospenderne la pubblicazione.

\subsection{Intervento dei moderatori}
Il sistema deve includere un modulo di moderazione che consenta di gestire i contenuti segnalati o non conformi.  
Gli utenti devono poter segnalare un contenuto.  
I moderatori devono poter sospendere o eliminare i contenuti segnalati.  
Ogni azione di moderazione deve essere registrata con identificativo utente, data e motivazione.

% ====================================================
% RF5
% ====================================================
\section{Policy Simulator e modelli statistici}

\subsection{Simulazione di scenari di policy}
Il sistema deve includere un modulo di simulazione che permetta di calcolare indicatori di impatto economico, ambientale e sociale delle proposte utilizzando dataset comunali o dati di riferimento configurabili.  

\subsection{Analisi predittiva e statistica}
Il sistema deve consentire la generazione di previsioni sull’effetto delle proposte tramite modelli statistici configurabili.  
Gli indicatori prodotti devono essere quantitativi e misurabili, come variazioni stimate di traffico, emissioni o costi.

\subsection{Visualizzazione interattiva}
Il sistema deve visualizzare i risultati delle simulazioni tramite mappe tematiche, grafici comparativi e dashboard interattive accessibili via interfaccia web.

% ====================================================
% RF6
% ====================================================
\section{Comunicazione, eventi e integrazione con strumenti esistenti}

\subsection{Notifiche e avvisi}
Il sistema deve inviare notifiche automatiche agli utenti in caso di nuovi processi, cambi di stato, aggiornamenti o inviti a eventi.  
Le notifiche devono essere configurabili per canale (email, push, App IO).

\subsection{Integrazione con strumenti comunali}
Il sistema deve integrarsi con strumenti già in uso presso l’amministrazione comunale, come il portale Open Data, l’App IO e i servizi di segnalazione, tramite API dedicate.

\subsection{Supporto a eventi dal vivo}
Il sistema deve consentire l’inserimento manuale o automatico, tramite API, dei contributi raccolti durante eventi fisici (assemblee, workshop, laboratori) nei processi digitali associati.

\subsection{Feed informativo}
Il sistema deve fornire un feed aggiornato che mostri aggiornamenti riguardo le attività in corso, le consultazioni attive e i risultati delle iniziative concluse.

% ====================================================
% RF7
% ====================================================
\section{Analisi, trasparenza e reportistica}

\subsection{Dashboard amministrativa}
Il sistema deve fornire all’amministrazione comunale una dashboard che mostri in tempo reale il numero di utenti attivi, le proposte per categoria, la distribuzione territoriale dei voti e i tassi di approvazione.  
I dati devono poter essere esportati in formato CSV o PDF.

\subsection{Esportazione e open data}
Il sistema deve fornire un’interfaccia API pubblica per la consultazione anonima dei dati aggregati relativi a proposte, voti e stati.  
L’amministrazione comunale deve poter esportare i dataset per la pubblicazione sul portale Open Data.

\subsection{Trasparenza e accountability}
Il sistema deve consentire all’amministrazione comunale, per ogni proposta conclusa, di pubblicare un riscontro ufficiale con la motivazione di accettazione o rifiuto.  
Il riscontro deve essere visibile nella pagina della proposta e registrato nel registro delle azioni.
