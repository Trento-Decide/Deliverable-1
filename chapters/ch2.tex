\chapter{Requisiti Funzionali}

\renewcommand{\thesubsection}{\textbf{RF\arabic{section}.\arabic{subsection}}}

\setlength{\parskip}{0.4em}
\setlength{\parindent}{0pt}

\section{Introduzione}
I requisiti funzionali definiscono le funzioni che il sistema \textit{Trento Decide} deve offrire per supportare la partecipazione civica, la collaborazione tra cittadini e amministrazione comunale e la trasparenza del processo decisionale.  
Ogni requisito è formulato in modo chiaro e verificabile ed è identificato dal prefisso \textbf{RF}, seguito da un numero progressivo e, ove necessario, da sottosezioni tematiche.  

% ====================================================
% RF1
% ====================================================

\section{Autenticazione e gestione credenziali}

\subsection{Registrazione dei cittadini}
L'utente che intende registrarsi come cittadino deve accedere inizialmente con SPID o CIE, il sistema richiede l'inserimento di email, password e nome utente; generando così le credenziali locali. Al termine, il sistema invia all'utente una email di conferma.

\subsection{Registrazione di moderatori e associazioni}
Gli account dei moderatori e delle associazioni/comitati vengono creati dall’amministratore tramite l’interfaccia di gestione.
Il sistema invia ai destinatari le credenziali di accesso tramite email.

\subsection{Login}
Il sistema deve consentire il login tramite credenziali locali (email e password) oppure tramite SPID o CIE.

\subsection{Gestione credenziali utente}
L'utente deve poter:
\begin{itemize}
  \item visualizzare e aggiornare email, username e password;
   \item consultare lo storico di modifiche alle proprie credenziali;
  \item richiedere la cancellazione del proprio profilo;
\end{itemize}

% ====================================================
% RF2
% ====================================================

\section{Gestione proposte cittadine}

\subsection{Creazione proposta}
Il sistema deve consentire ai cittadini di creare nuove proposte in stato di \textit{bozza}, compilando un modulo composto dai seguenti campi obbligatori:
\begin{itemize}
    \item \textbf{Titolo} — testo breve che identifica la proposta;
    \item \textbf{Descrizione} — testo esteso che illustra il contenuto della proposta;
    \item \textbf{Luogo} — indicazione dell’area geografica interessata, selezionabile tramite indirizzo o mappa;
    \item \textbf{Categoria} — definizione \ref{sec:cat};
\end{itemize}

Inoltre il sistema deve salvare automaticamente, per ogni proposta, le seguenti informazioni:
\begin{itemize}
    \item \textbf{Autore} — nome utente del cittadino che ha creato la proposta;
    \item \textbf{Data} — timestamp di creazione generato dal sistema;
\end{itemize}

I valori e la presenza di ciascun campo possono variare in base alla categoria selezionata.  
Il sistema deve utilizzare per ogni categoria l’elenco dei campi obbligatori e facoltativi configurato dall’amministrazione comunale.

\subsection{Pubblicazione proposta}
Il sistema deve consentire al cittadino autore di pubblicare una proposta precedentemente salvata in stato di \textit{bozza}.  
La pubblicazione è possibile solo se tutti i campi obbligatori previsti per la categoria selezionata risultano compilati.  
Al momento della pubblicazione il sistema aggiorna lo stato della proposta a \textit{pubblicata}.

\subsection{Proposte collettive}
Il sistema deve consentire ad associazioni e comitati di presentare \textbf{proposte collettive} (def.~\ref{def:proposta-collettiva}).  
Il sistema deve etichettare tali proposte come “collettive” e applicare ad esse gli stessi processi di validazione e pubblicazione previsti per le proposte individuali.

\subsection{Visualizzazione proposte}
Il sistema deve consentire al cittadino di:
\begin{itemize}
  \item visualizzare l’elenco delle proprie proposte, distinguendo tra quelle in stato di \textit{bozza} e quelle pubblicate;
  \item visualizzare le proposte pubblicate da altri utenti, applicando i criteri definiti in \ref{sec:sorting}.
\end{itemize}

\subsection{Stato e tracciabilità delle proposte} \label{sec:stato-tracciabilita}
Il sistema deve mostrare per ogni proposta lo stato corrente — \textit{bozza}, \textit{pubblicata}, \textit{in valutazione}, \textit{accettata}, \textit{rifiutata}, \textit{in attuazione}, \textit{completata} — e la cronologia pubblica delle transizioni di stato.

\subsection{Modifica collaborativa}
Il sistema deve consentire la modifica collaborativa delle proposte, come segue:

\subsubsection{Versionamento}
Ogni modifica a una proposta deve registrare autore, data/ora, descrizione e differenze rispetto alla versione precedente; tutte le versioni restano consultabili.
\subsubsection{Proposte di modifica da terzi}
Un utente diverso dal creatore può proporre una modifica; la proposta entra in uno stato di revisione non pubblica fino a decisione.
\subsubsection{Approvazione delle modifiche}
Il creatore della proposta può accettare o rifiutare una proposta di modifica; l’esito è tracciato.
\subsubsection{Ripristino}
Deve essere possibile ripristinare una versione precedente come versione principale.

\subsection{Endorsement e raccolta firme}
Il sistema deve consentire ai cittadini e alle associazioni registrate di esprimere un \textbf{endorsement} digitale (def.~\ref{def:endorsement}) sulle proposte pubblicate.  
Il sistema deve inoltre registrare e verificare le firme fisiche raccolte offline, integrandole nel conteggio del sostegno alla proposta.

% ====================================================
% RF3
% ====================================================

\section{Personalizzazione e preferenze}

\subsection{Preferiti}
Il sistema deve consentire ai cittadini di contrassegnare come “preferiti” proposte o sondaggi, ricevendo aggiornamenti automatici in caso di variazioni di stato o nuovi commenti.  

\subsection{Cambio di lingua}
Il sistema deve consentire all’utente di modificare in qualsiasi momento la lingua dell’interfaccia, selezionandola da un menu dedicato.  
La preferenza linguistica deve essere salvata nel profilo utente, in modo da persistere tra diverse sessioni di accesso e dispositivi.  

% ====================================================
% RF4
% ====================================================

\section{Votazioni e ranking}

\subsection{Votazione delle proposte}
Il sistema deve consentire agli utenti eleggibili al voto (cfr. Definizioni) di esprimere un voto positivo o negativo su ciascuna proposta attiva.  
Ogni elettore può esprimere al massimo un voto per proposta e può modificare o revocare il proprio voto entro la finestra temporale della consultazione.  

\subsection{Consultazioni e sondaggi}
Il sistema deve consentire all’amministrazione di creare consultazioni/sondaggi tematici, definirne il periodo di apertura e raccogliere le risposte dai cittadini.

\subsection{Ordinamento delle proposte}  \label{sec:sorting}
Il sistema deve consentire all’utente di ordinare l’elenco delle proposte pubblicate attraverso almeno le seguenti chiavi:
\begin{itemize}
  \item punteggio di rilevanza (valore numerico esposto);
  \item numero totale di voti;
  \item data di pubblicazione (più recente / meno recente);
  \item categoria.
\end{itemize}
L’utente deve poter invertire l’ordine (asc/desc) quando applicabile.

\subsection{Report votazioni}
Il sistema deve generare report contenenti almeno i seguenti indicatori: numero totale di votanti, distribuzione territoriale e fasce d’età.  
Ulteriori indicatori devono poter essere aggiunti tramite configurazione amministrativa.

% ====================================================
% RF5
% ====================================================

\section{Moderazione e qualità dei contenuti}

\subsection{Moderazione automatica}
Il sistema deve analizzare automaticamente i contenuti generati dagli utenti (testi e materiali multimediali) per rilevare linguaggio inappropriato, spam o duplicati.  
In caso di rilevamento, il sistema deve segnalare l’elemento ai moderatori o sospenderne la pubblicazione in attesa di verifica.

\subsection{Intervento dei moderatori}
Il sistema deve includere un modulo di moderazione che consenta di gestire i contenuti segnalati o non conformi.  
I cittadini devono poter segnalare un contenuto.  
I moderatori devono poter sospendere o eliminare i contenuti segnalati.  
Ogni azione di moderazione deve essere registrata con identificativo utente, data e motivazione.

% ====================================================
% RF6
% ====================================================

\section{Policy Simulator e modelli statistici}

\subsection{Simulazione di scenari di policy}
Il sistema deve includere un modulo che consenta di calcolare indicatori di impatto economico, ambientale e sociale delle proposte utilizzando dataset configurabili.

\subsection{Analisi predittiva e statistica}
Il sistema deve consentire la generazione di previsioni sull’effetto delle proposte tramite modelli configurabili; gli indicatori prodotti devono essere quantitativi e misurabili (es. variazioni stimate di traffico, emissioni o costi).

\subsection{Visualizzazione interattiva delle simulazioni}
Il sistema deve visualizzare i risultati delle simulazioni tramite mappe tematiche e grafici comparativi.

% ====================================================
% RF7
% ====================================================

\section{Comunicazione e integrazione}

\subsection{Notifiche e avvisi}
Il sistema deve inviare notifiche automatiche per variazioni di stato, voti, risposte o aggiornamenti relativi a elementi tra i preferiti.  
Le notifiche devono essere configurabili per evento e canale (es. email, push, App IO).

\subsection{Feed informativo}
Il sistema deve fornire un feed aggiornato che mostri attività in corso, consultazioni attive, proposte popolari e risultati di iniziative concluse.

% ====================================================
% RF8
% ====================================================

\section{Trasparenza e reportistica}

\subsection{Dashboard amministrativa}
Il sistema deve fornire all’amministrazione una dashboard che mostri il numero di utenti attivi, le proposte per categoria, la distribuzione territoriale dei voti e i tassi di approvazione.  
I dati devono poter essere esportati almeno in formato CSV o PDF.

\subsection{API e Open Data}
Il sistema deve fornire un’interfaccia API pubblica per la consultazione dei dati aggregati relativi a proposte, voti e stati.  
L’amministrazione comunale deve poter esportare i dataset per la pubblicazione sul portale Open Data.

\subsection{Trasparenza e accountability}
Il sistema deve consentire all’amministrazione comunale, per ogni proposta conclusa, di pubblicare un riscontro ufficiale con la motivazione di accettazione o rifiuto.  
Il riscontro deve essere visibile nella pagina della proposta e registrato nel registro delle azioni.