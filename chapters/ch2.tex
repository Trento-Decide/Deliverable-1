\chapter{Requisiti Funzionali}

\renewcommand{\thesubsection}{\textbf{RF\arabic{section}.\arabic{subsection}}}
\renewcommand{\thesubsubsection}{\textbf{RF\arabic{section}.\arabic{subsection}.\arabic{subsubsection}}}

\setlength{\parskip}{0.4em}
\setlength{\parindent}{0pt}

\section*{Introduzione}
I requisiti funzionali definiscono le funzioni che il sistema \textit{Trento Decide} deve offrire per supportare la partecipazione civica, la collaborazione tra cittadini e amministrazione comunale e la trasparenza del processo decisionale.
Ogni requisito è formulato in modo chiaro e verificabile ed è identificato dal prefisso \textbf{RF}, seguito da un numero progressivo e, ove necessario, da sottosezioni tematiche.

\section{Autenticazione e gestione credenziali}

\subsection{Registrazione dei cittadini} \label{sec:registrazione}
L’utente che intende registrarsi come cittadino deve accedere tramite SPID o CIE.  
Dopo l’autenticazione SPID/CIE, il sistema verifica la residenza dell’utente nel Comune attraverso ANPR.  
Superata la verifica, il sistema richiede l’inserimento delle credenziali locali (email, password e nome utente).

L’utente può inoltre acconsentire facoltativamente alla ricezione di notifiche via email (abilita/disabilita).  
Al termine della procedura, il sistema invia un’email di conferma all’indirizzo fornito.

\subsection{Registrazione di moderatori e associazioni} \label{sec:specialregistrazione}
Gli account dei moderatori e delle associazioni/comitati vengono creati dall’amministratore tramite l’interfaccia di gestione.
Il sistema invia ai destinatari le credenziali di accesso tramite email.

\subsection{Login} \label{sec:login}
Il sistema deve consentire il login tramite credenziali locali (email e password) oppure tramite SPID o CIE.

\subsection{Logout} \label{sec:logout}
Il sistema deve consentire ad ogni utente di terminare la propria sessione tramite una funzione di logout.
Il sistema dunque invalida token di autenticazione e reindirizza l’utente alla schermata di login.

\subsection{Gestione profilo e dati personali} \label{sec:credenzialihandling}
L’utente deve poter visualizzare e gestire i propri dati personali, come segue:

\subsubsection{Credenziali}
Il sistema deve consentire all’utente di aggiornare email, password e username.
Il sistema deve inoltre mostrare all’utente i valori correnti di email e username.

\subsubsection{Ricezione notifiche}
Il sistema deve consentire all’utente di modificare la propria preferenza riguardo alla ricezione delle notifiche via email, scegliendo tra le opzioni abilita/disabilita.

\subsubsection{Storico modifiche}
Il sistema deve mostrare all’utente lo storico delle modifiche effettuate alle proprie credenziali, includendo data e tipo di modifica.

\subsubsection{Eliminazione profilo}
Il sistema deve consentire all’utente di richiedere la cancellazione del proprio profilo.
Una volta richiesta l’eliminazione, l’utente deve poter accedere al proprio profilo per le successive 48 ore, durante le quali può annullare la richiesta.

\section{Gestione delle proposte}

\subsection{Creazione proposta} \label{sec:creazioneproposta}
Il sistema deve consentire ai cittadini di creare nuove proposte, compilando un modulo composto dai seguenti campi obbligatori:
\begin{itemize}
    \item \textbf{Titolo} — testo breve che identifica la proposta;
    \item \textbf{Descrizione} — testo esteso che illustra il contenuto della proposta;
    \item \textbf{Luogo} — indicazione dell’area geografica interessata, selezionabile tramite indirizzo;
    \item \textbf{Categoria} — \defref{def:categoria};
\end{itemize}

Inoltre il sistema deve salvare automaticamente, per ogni proposta, le seguenti informazioni:
\begin{itemize}
    \item \textbf{Autore} — nome utente del cittadino che ha creato la proposta;
    \item \textbf{Data} — timestamp di creazione generato dal sistema;
\end{itemize}

I valori e la presenza di ciascun campo possono variare in base alla categoria selezionata.
Il sistema deve utilizzare per ogni categoria il proprio elenco di campi specifici e obbligatori \defref{def:categoria}.

\subsection{Salvare proposta come bozza} \label{sec:salvacomebozza}
Il sistema deve permettere al cittadino di salvare una proposta nello stato di \textit{bozza}. La proposta verrà contrassegnata con tale stato, che sarà visibile anche nella lista delle proposte. Inoltre, selezionando una tale proposta, il sistema riaprirà il modulo di compilazione per consentirne la modifica e/o la pubblicazione.

\subsection{Pubblicazione proposte collettive} \label{sec:propostacollettiva}
Il sistema deve consentire ad associazioni e comitati di presentare \textit{proposte collettive} (def.~\ref{def:proposta-collettiva}).
Il sistema rende riconoscibili tali proposte etichettandole con il nome dell'associazione autrice.

\subsection{Visualizzazione proposte e sondaggi} \label{sec:visualizzapropsond}
Il sistema deve consentire ad un utente di:
\begin{itemize}
  	\item (solo cittadini e associazioni) visualizzare un elenco personale contenente le proprie proposte, sia pubblicate che in stato di bozza che eliminate;
  	\item visualizzare le proposte pubblicate da altri utenti, applicando i criteri definiti in \ref{sec:sorting}.
	  \item visualizzare i sondaggi pubblicati dall'amministrazione \ref{sec:sondaggi}.
\end{itemize}

\subsection{Modifica collaborativa} \label{sec:modificaproposta}
Il sistema deve consentire la modifica collaborativa delle proposte, come segue:

\subsubsection{Versionamento}
Ogni modifica a una proposta deve registrare autore, data/ora, descrizione e differenze rispetto alla versione precedente; tutte le versioni restano consultabili.
\subsubsection{Proposte di modifica da terzi}
Un utente diverso dal creatore può proporre una modifica; la proposta entra in uno stato di revisione non pubblica fino a decisione.
\subsubsection{Approvazione delle modifiche}
Il creatore della proposta può accettare o rifiutare una proposta di modifica; l’esito è tracciato.
\subsubsection{Ripristino}
Deve essere possibile ripristinare una versione precedente come versione principale.

\subsection{Endorsement} \label{sec:endorsement}
Il sistema deve consentire alle associazioni di esprimere un \textit{endorsement} digitale (def.~\ref{def:endorsement}) sulle proposte pubblicate.

\section{Personalizzazione e preferenze}

\subsection{Preferiti} \label{sec:preferiti}
Il sistema deve consentire al cittadino di:
\begin{itemize}
	\item contrassegnare e rimuovere dai \textit{preferiti} proposte e sondaggi;
	\item visualizzare l’elenco dei propri elementi preferiti;
\end{itemize}

\subsection{Notifiche e avvisi}
Il sistema deve inviare notifiche automatiche per variazioni di stato riguardo elementi posti nei propri preferiti.
Inoltre il sistema invia notifiche per qualsiasi aggiornamento in proposte proprie (variazioni di stato e richieste di modifica).
Le notifiche vengono inviate solo se il cittadino ha acconsentito alla ricezione di notifiche (\ref{sec:registrazione}).

\subsection{Cambio di lingua} \label{sec:cambiolingua}
Il sistema deve consentire all’utente di modificare la lingua (definite in \ref{ref:i18n}) dell’interfaccia tramite un menu dedicato accessibile dall’area utente.
La preferenza linguistica deve essere registrata nel profilo utente e mantenuta tra le sessioni di accesso.

\section{Votazioni e ordinamento}

\subsection{Votazione} \label{sec:voto}
Il sistema deve consentire agli utenti eleggibili al voto (cfr.~Definizioni) di esprimere un voto positivo o negativo su:
\begin{itemize}
	\item \textbf{Proposte}: voto modificabile fino allo stato di \textit{accettata} o \textit{rifiutata};
	\item \textbf{Sondaggi}: voto modificabile fino a chiusura dello stesso;
\end{itemize}
Ogni elettore può esprimere un solo voto per proposta o quesito di ogni sondaggio.

\subsection{Ordinamento delle proposte}  \label{sec:sorting}
Il sistema deve consentire all’utente di ordinare l’elenco delle proposte pubbliche attraverso le seguenti chiavi:
\begin{itemize}
  \item numero totale di voti;
  \item data di pubblicazione;
  \item categoria;
\end{itemize}
L’utente deve poter invertire l’ordine (asc/desc) quando applicabile, sia numerico che alfabetico.

\section{Moderazione e qualità dei contenuti}

\subsection{Segnalazione dei contenuti da parte dei cittadini} \label{sec:segnalazione}
Il sistema deve consentire ai cittadini di segnalare contenuti pubblicati (proposte, modifiche) selezionando una motivazione da un elenco predefinito
(incitamento all’odio, contenuti offensivi o volgari, informazioni false, spam, duplicato, violazione della privacy, altro) e aggiungendo un testo descrittivo.
Ogni segnalazione deve generare un elemento nella coda di revisione dei moderatori.

\subsection{Moderazione automatica}
Il sistema deve analizzare automaticamente i testi generati dagli utenti per rilevare linguaggio inappropriato, spam o duplicati.
In caso di rilevamento, il sistema deve inserire l’elemento nella coda di revisione o rimuoverlo.

\subsection{Visualizzazione coda di revisione} \label{sec:validazione-contenuti}
Il sistema deve permettere ai moderatori di visualizzare la coda di revisione, ovvero un elenco di proposte segnalate automaticamente o da utenti.

\subsection{Rimozione dei contenuti} \label{sec:rimozione}
Il sistema deve consentire ai moderatori di eliminare un contenuto dalla piattaforma.
Ogni rimozione deve essere registrata indicando motivazione, timestamp e moderatore responsabile.

\subsection{Moderazione sugli utenti}

\subsubsection{Limitazioni sui cittadini} \label{sec:modcittadini}
Il sistema deve consentire ai moderatori di sospendere un cittadino scegliendo una durata tra 1, 7, 30 o 90 giorni, oppure di procedere alla rimozione definitiva del suo account.
La rimozione definitiva comporta l’impossibilità di autenticarsi con le credenziali associate.

\subsubsection{Limitazioni sulle associazioni} \label{sec:modassociazioni}
Il sistema deve consentire ai moderatori di richiedere all’amministratore la sospensione o rimozione di un account associazione, bloccandone le interazioni pubbliche per un periodo temporaneo (48 ore) in attesa della decisione amministrativa.

\section{Policy Simulator e modelli statistici}

\subsection{Simulazione di scenari di policy}
Il sistema deve includere un modulo che consenta all’amministrazione di calcolare indicatori di impatto economico, sociale e ambientale associati alle proposte.
Il modulo deve permettere la configurazione di parametri e coefficienti specifici dall’amministrazione.
Gli indicatori generati devono essere visibili esclusivamente all’amministrazione \defref{def:classificazione}.

\subsection{Analisi predittiva e statistica}
Il sistema deve consentire all’amministrazione di generare stime sugli effetti futuri delle proposte sulla base di dataset caricati.
Il modulo deve permettere la selezione del modello da utilizzare, la definizione delle variabili di input e la produzione di indicatori quantitativi relativi agli scenari simulati.

\subsection{Visualizzazione dei risultati} \label{sec:vederepolicies}
Il sistema deve consentire all’amministrazione di visualizzare gli indicatori prodotti dai moduli di simulazione tramite rappresentazioni grafiche e tabelle comparative.

\section{Dashboard amministrativa}

\subsection{Dashboard amministrativa} \label{sec:dashboard}
Il sistema deve fornire all’amministrazione una dashboard che comprende:
\begin{itemize}
  \item I seguenti dati: numero di utenti attivi, proposte per categoria, distribuzione territoriale dei voti, tassi di approvazione.

  \item Strumenti dell'amministrazione: creazione sondaggi \ref{sec:sondaggi}, scaricare report \ref{sec:report-attivita}, registrare account associazioni e moderatori \ref{sec:specialregistrazione}.
\end{itemize}

\subsection{Report sull’attività del sistema} \label{sec:report-attivita}
Il sistema deve consentire all’amministrazione di generare e scaricare in locale report contenenti dati aggregati e anonimizzati (\ref{ref:anonimato}) sull’attività della piattaforma.
I report devono essere disponibili in formato aperto (CSV) e comprendere:

\begin{itemize}
  \item \textbf{Attività per proposta}: numero totale di votanti, distribuzione territoriale, distribuzione per fasce d’età.
\end{itemize}

\subsection{Sondaggi} \label{sec:sondaggi}
Il sistema deve consentire all’amministrazione (tramite dashboard) di creare, pubblicare e gestire sondaggi tematici rivolti ai cittadini.
Ogni sondaggio deve comprendere:
(a) \textit{Titolo},
(b) \textit{Descrizione},
(c) \textit{Categoria},
(d) \textit{Periodo di validità} (data di apertura e di chiusura),
(e) \textit{Domande} (e relative opzioni di risposta).

Il sistema deve inoltre:
\begin{itemize}
	\item chiudere automaticamente i sondaggi alla data di scadenza;
	\item raccogliere in modo anonimo le risposte dei cittadini, rendendole disponibili all'amministrazione;
\end{itemize}

\subsection{Modifica dello stato di una proposta} \label{sec:editproposta}
Il sistema deve consentire all’amministratore di aggiornare lo stato di una proposta \defref{def:statoproposta} secondo il flusso previsto: da \textit{pubblicata} a \textit{in valutazione}, quindi a \textit{accettata} o \textit{rifiutata}.
In caso di stato \textit{accettata}, il sistema deve permettere il passaggio agli stati \textit{in attuazione} e \textit{completata}.
In caso di stato \textit{rifiutata}.
In ogni caso il sistema deve consentire all’amministratore di pubblicare una motivazione associata alla decisione.
