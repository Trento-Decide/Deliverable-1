\chapter{Requisiti Funzionali}

\renewcommand{\thesubsection}{\textbf{RF\arabic{section}.\arabic{subsection}}}

\setlength{\parskip}{0.4em}
\setlength{\parindent}{0pt}

\section{Introduzione}
I requisiti funzionali definiscono le funzioni che il sistema \textit{Trento Decide} deve offrire per supportare la partecipazione civica, la collaborazione tra cittadini e amministrazione comunale e la trasparenza del processo decisionale.  
Ogni requisito è formulato in modo chiaro e verificabile ed è identificato dal prefisso \textbf{RF}, seguito da un numero progressivo e, ove necessario, da sottosezioni tematiche.  

% ====================================================
% RF1
% ====================================================

\section{Gestione utenti e autenticazione}

\subsection{Registrazione e login}
Il sistema deve consentire ai cittadini, ai moderatori e agli amministratori comunali di registrarsi e accedere tramite autenticazione tradizionale (email e password) o tramite SPID/CIE.  


\subsection{Gestione profilo utente}
Il sistema deve permettere a ciascun utente di visualizzare e modificare i propri dati personali nella sezione profilo, in conformità al GDPR.  
L’utente deve poter esercitare il diritto di rettifica e cancellazione dei dati.  
Il profilo mostra inoltre lo storico delle proposte create, dei voti espressi e delle attività recenti.

\subsection{Gestione ruoli e permessi} \label{sec:ruoli}
Il sistema deve implementare un modello di \textbf{controllo accessi basato su ruoli (RBAC)}, con interfaccia di amministrazione dedicata.  
Sono previsti i seguenti ruoli:
\begin{itemize}
   \item \textbf{Cittadino}: può creare e modificare proposte, suggerire revisioni, votare e partecipare a consultazioni pubbliche, secondo le condizioni di eleggibilità definite nelle \textit{Definizioni} \ref{def:cittadino}–\ref{def:residente}.
   \item \textbf{Associazione / Comitato}: può presentare proposte collettive e fornire endorsement verificati;
    \item \textbf{Moderatore}: supervisiona i contenuti generati dagli utenti, gestisce le segnalazioni, valida o rimuove contenuti non conformi in base alle policy di moderazione (\ref{ref:moderazione}).
    \item \textbf{Amministratore comunale}: coordina il flusso di valutazione delle proposte, approva o rifiuta le iniziative e pubblica gli esiti ufficiali;
\end{itemize}

Ogni ruolo deve avere un insieme di permessi configurabile dall’amministrazione comunale.

% ====================================================
% RF2
% ====================================================

\section{Personalizzazione e preferenze}

\subsection{Preferiti}
Il sistema deve consentire ai cittadini di contrassegnare come “preferiti” proposte o consultazioni di interesse, ricevendo aggiornamenti automatici in caso di variazioni di stato o nuovi commenti.  
Le notifiche devono essere configurabili per canale (email, push, App IO).

\subsection{Cambio di lingua}
Il sistema deve consentire all’utente di modificare in qualsiasi momento la lingua dell’interfaccia, selezionandola da un menu dedicato.  
La preferenza linguistica deve essere salvata nel profilo utente, in modo da persistere tra diverse sessioni di accesso e dispositivi.  
Le lingue disponibili sono definite nel requisito non funzionale \ref{ref:i18n}.
I contenuti generati dagli utenti (proposte, commenti, titoli) rimangono nella lingua di origine.  
È prevista la possibilità di fornire traduzioni automatiche a scopo informativo, chiaramente marcate come tali, per evitare la creazione di silos linguistici.

% ====================================================
% RF3
% ====================================================

\section{Gestione proposte cittadine}

\subsection{Creazione proposta}
Il sistema deve consentire ai cittadini di redigere nuove proposte in stato di \textit{bozza}, compilando un modulo composto da campi specifici per la categoria selezionata.  
I campi obbligatori e facoltativi per ciascuna categoria devono essere configurabili dall’amministrazione comunale.

\subsection{Pubblicazione proposta}
Il sistema deve permettere la pubblicazione di una proposta precedentemente salvata in stato di \textit{bozza}.

\subsection{Modifica collaborativa}
Il sistema deve consentire la modifica collaborativa delle proposte, implementando un meccanismo di controllo versioni che registri autore, data, descrizione e differenze tra le revisioni.  
Ogni modifica proposta da terzi deve essere approvata dal creatore della proposta o da un moderatore.  
Tutte le versioni precedenti devono rimanere consultabili con indicazione dell’autore e della motivazione della modifica, e deve essere possibile ripristinare una versione precedente come versione principale.

\subsection{Stato e tracciabilità delle proposte} \label{sec:stato-tracciabilita}
Il sistema deve gestire automaticamente lo stato di ogni proposta in base alle azioni del cittadino o dell’amministrazione comunale, garantendone la piena tracciabilità.  
Ogni proposta deve mostrare lo stato corrente — \textit{bozza}, \textit{pubblicata}, \textit{in valutazione}, \textit{accettata}, \textit{rifiutata}, \textit{in attuazione}, \textit{completata} — e la cronologia pubblica delle transizioni.

\subsection{Endorsement e raccolta firme}
Il sistema deve consentire ai cittadini e alle associazioni registrate di esprimere un \textbf{endorsement} digitale (def.~\ref{def:endorsement}) a favore delle proposte pubblicate.  
Ogni endorsement contribuisce al punteggio di ranking in misura proporzionale a un coefficiente configurabile dall’amministrazione comunale.  
Il sistema deve inoltre consentire la registrazione e la verifica di eventuali firme fisiche raccolte offline, per garantire l’inclusione dei cittadini privi di strumenti digitali.

\subsection{Proposte collettive}
Il sistema deve consentire ad associazioni e comitati riconosciuti di presentare \textbf{proposte collettive} (def.~\ref{def:proposta-collettiva}) tramite un canale dedicato.  
Tali proposte devono essere etichettate come “collettive” e sottoposte alle stesse regole di validazione e pubblicazione delle proposte individuali.  
Il loro peso nel calcolo del ranking deve essere maggiore rispetto a quello delle proposte individuali, secondo un coefficiente configurabile e documentato.


% ====================================================
% RF4
% ====================================================

\section{Votazioni e ranking}

\subsection{Votazione delle proposte}
Il sistema deve consentire ai cittadini di esprimere un voto positivo o negativo su ciascuna proposta attiva, purché soddisfino i criteri di eleggibilità al voto indicati nelle Definizioni~\ref{def:cittadino} e~\ref{def:residente}.  
Il voto può essere modificato o revocato entro la finestra temporale stabilita per la consultazione.  
L’identità dell’elettore deve restare separata dal contenuto del voto, garantendo l’anonimato secondo quanto descritto nella Definizione~\ref{def:voto}.

\subsection{Consultazioni e sondaggi}
Il sistema deve consentire all’amministrazione comunale di creare consultazioni e sondaggi tematici, raccogliendo risposte dai cittadini. 

\subsection{Algoritmo di ranking}
Il sistema deve assegnare a ciascuna proposta un punteggio di rilevanza calcolato su parametri configurabili dall’amministrazione comunale.  
Le proposte devono poter essere ordinate per punteggio, numero di voti, data o categoria.  
Le modalità di calcolo e i criteri di equità sono descritte nei Requisiti Non Funzionali relativi alla trasparenza algoritmica (\ref{ref:neutralita}).

\subsection{Report votazioni}
Il sistema deve generare report contenenti almeno i seguenti indicatori:numero totale di votanti, distribuzione territoriale e fasce d’età.  
Ulteriori indicatori devono poter essere aggiunti tramite configurazione amministrativa.

% ====================================================
% RF5
% ====================================================

\section{Moderazione e qualità dei contenuti}

\subsection{Moderazione automatica}
Il sistema deve analizzare automaticamente i contenuti generati dagli utenti (testi e materiali multimediali) per rilevare linguaggio inappropriato, spam o duplicati.  
In caso di rilevamento, il sistema deve segnalare l’elemento ai moderatori o sospenderne la pubblicazione.

\subsection{Intervento dei moderatori}
Il sistema deve includere un modulo di moderazione che consenta di gestire i contenuti segnalati o non conformi.  
I cittadini devono poter segnalare un contenuto.  
I moderatori devono poter sospendere o eliminare i contenuti segnalati.  
Ogni azione di moderazione deve essere registrata con identificativo utente, data e motivazione.

% ====================================================
% RF6
% ====================================================

\section{Policy Simulator e modelli statistici}

\subsection{Simulazione di scenari di policy}
Il sistema deve includere un modulo di simulazione che permetta di calcolare indicatori di impatto economico, ambientale e sociale delle proposte utilizzando dataset comunali o dati di riferimento configurabili.\footnote{L'implementazione di questo modulo rappresenta la sfida tecnica più significativa del progetto, richiedendo non solo l'accesso a dataset comunali di alta qualità (spesso eterogenei e non standardizzati), ma anche competenze avanzate in data science e modellistica. Si prevede uno sviluppo incrementale, partendo da simulazioni semplici basate su pochi indicatori affidabili.}

\subsection{Analisi predittiva e statistica}
Il sistema deve consentire la generazione di previsioni sull’effetto delle proposte tramite modelli statistici configurabili.  
Gli indicatori prodotti devono essere quantitativi e misurabili, come variazioni stimate di traffico, emissioni o costi.

\subsection{Visualizzazione interattiva delle simulazioni}
Il sistema deve visualizzare i risultati delle simulazioni tramite mappe tematiche e grafici comparativi.

% ====================================================
% RF7
% ====================================================

\section{Comunicazione e integrazione}

\subsection{Notifiche e avvisi}
Il sistema deve inviare notifiche automatiche agli utenti per variazioni di stato, voti, risposte o aggiornamenti relativi a proposte tra i propri preferiti.  
Le notifiche devono essere configurabili per canale (email, push, App IO).

\subsection{Feed informativo}
Il sistema deve fornire un feed aggiornato che mostri le attività in corso, le consultazioni attive, le proposte popolari e i risultati delle iniziative concluse.

% ====================================================
% RF8
% ====================================================

\section{Trasparenza e reportistica}

\subsection{Dashboard amministrativa}
Il sistema deve fornire all’amministrazione comunale una dashboard che mostri il numero di utenti attivi, le proposte per categoria, la distribuzione territoriale dei voti e i tassi di approvazione.  
I dati devono poter essere esportati in formato CSV o PDF.

\subsection{API e Open Data}
Il sistema deve fornire un’interfaccia API pubblica per la consultazione anonima dei dati aggregati relativi a proposte, voti e stati.  
L’amministrazione comunale deve poter esportare i dataset per la pubblicazione sul portale Open Data.

\subsection{Trasparenza e accountability}
Il sistema deve consentire all’amministrazione comunale, per ogni proposta conclusa, di pubblicare un riscontro ufficiale con la motivazione di accettazione o rifiuto.  
Il riscontro deve essere visibile nella pagina della proposta e registrato nel registro delle azioni.