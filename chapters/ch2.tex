\chapter{Requisiti Funzionali}

\renewcommand{\thesubsection}{\textbf{RF\arabic{section}.\arabic{subsection}}}

\setlength{\parskip}{0.4em}
\setlength{\parindent}{0pt}

\section{Introduzione}
I requisiti funzionali definiscono le funzioni che il sistema \textit{Trento Decide} deve offrire per supportare la partecipazione civica, la collaborazione tra cittadini e amministrazione comunale e la trasparenza del processo decisionale.  
Ogni requisito è formulato in modo chiaro e verificabile ed è identificato dal prefisso \textbf{RF}, seguito da un numero progressivo e, ove necessario, da sottosezioni tematiche.  

% ====================================================
% RF1
% ====================================================

\section{Gestione utenti e autenticazione}

\subsection{Registrazione e login}
Il sistema deve consentire ai cittadini, ai moderatori e agli amministratori comunali di registrarsi e accedere tramite autenticazione tradizionale (email e password) o tramite SPID/CIE.  

\subsection{Gestione profilo utente}
L’utente deve poter: (a) visualizzare e aggiornare i propri dati nel profilo; (b) richiedere la rettifica dei dati; (c) avviare una richiesta di cancellazione dei dati personali; (d) consultare lo storico di proposte, revisioni, endorsement e voti.

\subsection{Gestione ruoli e permessi} \label{sec:ruoli}
Il sistema deve fornire un modulo di amministrazione per:

\begin{enumerate}
    \item Creare o assegnare ruoli (Cittadino, Associazione/Comitato, Moderatore, Amministratore).
    \item Assegnare a ciascun ruolo i permessi disponibili tra: \texttt{crea\_proposta}, \texttt{modifica\_propria\_proposta}, \texttt{propone\_revisione}, 
    \texttt{modera\_contenuti}, \texttt{approva\_rifiuta\_proposta}, \texttt{configura\_parametri}.
    \item Visualizzare l’elenco di tutte le modifiche ai ruoli e permessi (audit) con: utente che ha effettuato la modifica, ruolo coinvolto, permesso aggiunto/rimosso, data/ora.
\end{enumerate}

% ====================================================
% RF2
% ====================================================

\section{Personalizzazione e preferenze}

\subsection{Preferiti}
Il sistema deve consentire ai cittadini di contrassegnare come “preferiti” proposte o consultazioni di interesse, ricevendo aggiornamenti automatici in caso di variazioni di stato o nuovi commenti.  

\subsection{Cambio di lingua}
Il sistema deve consentire all’utente di modificare in qualsiasi momento la lingua dell’interfaccia, selezionandola da un menu dedicato.  
La preferenza linguistica deve essere salvata nel profilo utente, in modo da persistere tra diverse sessioni di accesso e dispositivi.  

% ====================================================
% RF3
% ====================================================

\section{Gestione proposte cittadine}

\subsection{Creazione proposta}
Il sistema deve consentire ai cittadini di redigere nuove proposte in stato di \textit{bozza}, compilando un modulo composto da campi specifici per la categoria selezionata.  
I campi obbligatori e facoltativi per ciascuna categoria devono essere configurabili dall’amministrazione comunale.

\subsection{Pubblicazione proposta}
Il sistema deve permettere la pubblicazione di una proposta precedentemente salvata in stato di \textit{bozza}.

\subsection{Modifica collaborativa}
Il sistema deve consentire la modifica collaborativa delle proposte, come segue:

\subsubsection{Versionamento}
Ogni modifica a una proposta deve registrare autore, data/ora, descrizione e differenze rispetto alla versione precedente; tutte le versioni restano consultabili.
\subsubsection{Proposte di modifica da terzi}
Un utente diverso dal creatore può proporre una modifica; la proposta entra in uno stato di revisione non pubblica fino a decisione.
\subsubsection{Approvazione delle modifiche}
Il creatore della proposta o un moderatore può accettare o rifiutare una proposta di modifica; l’esito è notificato al proponente e tracciato.
\subsubsection{Ripristino}
Deve essere possibile ripristinare una versione precedente come versione principale, con tracciamento dell’operazione.

\subsection{Stato e tracciabilità delle proposte} \label{sec:stato-tracciabilita}
Il sistema deve gestire automaticamente lo stato di ogni proposta in base alle azioni dell’utente o dell’amministrazione, garantendone la tracciabilità.   
Ogni proposta deve mostrare lo stato corrente — \textit{bozza}, \textit{pubblicata}, \textit{in valutazione}, \textit{accettata}, \textit{rifiutata}, \textit{in attuazione}, \textit{completata} — e la cronologia pubblica delle transizioni.

\subsection{Endorsement e raccolta firme}
Il sistema deve consentire a cittadini e associazioni registrate di esprimere un \textbf{endorsement} digitale (def.~\ref{def:endorsement}) a favore delle proposte pubblicate.  
Il sistema deve inoltre consentire la registrazione e la verifica di firme fisiche raccolte offline, affinché siano conteggiate nel sostegno alla proposta.

\subsection{Proposte collettive}
Il sistema deve consentire ad associazioni e comitati riconosciuti di presentare \textbf{proposte collettive} (def.~\ref{def:proposta-collettiva}) tramite un canale dedicato.  
Tali proposte devono essere etichettate come “collettive” e sottostare alle stesse regole di validazione e pubblicazione delle proposte individuali. 

% ====================================================
% RF4
% ====================================================

\section{Votazioni e ranking}

\subsection{Votazione delle proposte}
Il sistema deve consentire agli utenti eleggibili al voto (cfr. Definizioni) di esprimere un voto positivo o negativo su ciascuna proposta attiva.  
Ogni elettore può esprimere al massimo un voto per proposta e può modificare o revocare il proprio voto entro la finestra temporale della consultazione.  

\subsection{Consultazioni e sondaggi}
Il sistema deve consentire all’amministrazione di creare consultazioni/sondaggi tematici, definirne il periodo di apertura e raccogliere le risposte dai cittadini.

\subsection{Ordinamento delle proposte}
Il sistema deve consentire all’utente di ordinare l’elenco delle proposte pubblicate attraverso almeno le seguenti chiavi:
\begin{itemize}
  \item punteggio di rilevanza (valore numerico esposto);
  \item numero totale di voti;
  \item data di pubblicazione (più recente / meno recente);
  \item categoria.
\end{itemize}
L’utente deve poter invertire l’ordine (asc/desc) quando applicabile.

\subsection{Report votazioni}
Il sistema deve generare report contenenti almeno i seguenti indicatori: numero totale di votanti, distribuzione territoriale e fasce d’età.  
Ulteriori indicatori devono poter essere aggiunti tramite configurazione amministrativa.

% ====================================================
% RF5
% ====================================================

\section{Moderazione e qualità dei contenuti}

\subsection{Moderazione automatica}
Il sistema deve analizzare automaticamente i contenuti generati dagli utenti (testi e materiali multimediali) per rilevare linguaggio inappropriato, spam o duplicati.  
In caso di rilevamento, il sistema deve segnalare l’elemento ai moderatori o sospenderne la pubblicazione in attesa di verifica.

\subsection{Intervento dei moderatori}
Il sistema deve includere un modulo di moderazione che consenta di gestire i contenuti segnalati o non conformi.  
I cittadini devono poter segnalare un contenuto.  
I moderatori devono poter sospendere o eliminare i contenuti segnalati.  
Ogni azione di moderazione deve essere registrata con identificativo utente, data e motivazione.

% ====================================================
% RF6
% ====================================================

\section{Policy Simulator e modelli statistici}

\subsection{Simulazione di scenari di policy}
Il sistema deve includere un modulo che consenta di calcolare indicatori di impatto economico, ambientale e sociale delle proposte utilizzando dataset configurabili.

\subsection{Analisi predittiva e statistica}
Il sistema deve consentire la generazione di previsioni sull’effetto delle proposte tramite modelli configurabili; gli indicatori prodotti devono essere quantitativi e misurabili (es. variazioni stimate di traffico, emissioni o costi).

\subsection{Visualizzazione interattiva delle simulazioni}
Il sistema deve visualizzare i risultati delle simulazioni tramite mappe tematiche e grafici comparativi.

% ====================================================
% RF7
% ====================================================

\section{Comunicazione e integrazione}

\subsection{Notifiche e avvisi}
Il sistema deve inviare notifiche automatiche per variazioni di stato, voti, risposte o aggiornamenti relativi a elementi tra i preferiti.  
Le notifiche devono essere configurabili per evento e canale (es. email, push, App IO).

\subsection{Feed informativo}
Il sistema deve fornire un feed aggiornato che mostri attività in corso, consultazioni attive, proposte popolari e risultati di iniziative concluse.

% ====================================================
% RF8
% ====================================================

\section{Trasparenza e reportistica}

\subsection{Dashboard amministrativa}
Il sistema deve fornire all’amministrazione una dashboard che mostri il numero di utenti attivi, le proposte per categoria, la distribuzione territoriale dei voti e i tassi di approvazione.  
I dati devono poter essere esportati almeno in formato CSV o PDF.

\subsection{API e Open Data}
Il sistema deve fornire un’interfaccia API pubblica per la consultazione dei dati aggregati relativi a proposte, voti e stati.  
L’amministrazione comunale deve poter esportare i dataset per la pubblicazione sul portale Open Data.

\subsection{Trasparenza e accountability}
Il sistema deve consentire all’amministrazione comunale, per ogni proposta conclusa, di pubblicare un riscontro ufficiale con la motivazione di accettazione o rifiuto.  
Il riscontro deve essere visibile nella pagina della proposta e registrato nel registro delle azioni.