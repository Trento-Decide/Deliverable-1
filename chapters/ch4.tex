\chapter{Use Case Diagram}

\renewcommand{\thesubsection}{UC\arabic{subsection}}
\setlength{\parskip}{0.4em}
\setlength{\parindent}{0pt}

\section*{Introduzione}
In questa sezione vengono presentati gli Use Case Diagram del sistema Trento Decide.
Essi rappresentano, in forma grafica, le interazioni tra gli attori e la piattaforma, illustrando le principali funzionalità e relazioni di inclusione o estensione tra i casi d’uso.
Questi diagrammi consentono di visualizzare in modo sintetico come i requisiti funzionali (RF) si traducono in comportamenti concreti del sistema, offrendo una visione d’insieme delle dinamiche operative e dei ruoli coinvolti.

% ====================================================
% UC1
% ====================================================

\section{Utente Anonimo}
\begin{enumerate}
    \item Fare login sulla piattaforma (\ref{sec:login})
    \item Registrarsi come cittadino (\ref{sec:registrazione})
\end{enumerate}

\begin{center}
	\includegraphics[scale=1.1]{img/usecase/utenteanonimo.png}
\end{center}

% ====================================================

\subsection{Login \ref{sec:login}}

\textbf{Riassunto:}  
Questo use case descrive come un utente anonimo accede alla piattaforma tramite credenziali locali oppure tramite SPID o CIE.

\textbf{Precondizioni:}  
L’utente non è autenticato.

\textbf{Postcondizioni:}  
L’utente risulta autenticato e accede alla piattaforma.

\textbf{Descrizione:}
\begin{enumerate}
    \item L’utente visualizza la schermata di login.
    \item L’utente può scegliere una delle seguenti modalità:
    \begin{enumerate}
        \item \textbf{Credenziali locali:} inserisce email e password nei campi dedicati e conferma l’accesso.
        \item \textbf{SPID/CIE:} seleziona l’opzione SPID o CIE; il sistema reindirizza al relativo servizio di autenticazione.
    \end{enumerate}
    \item Se l’autenticazione è valida, l'utente accede al sistema \textbf{[eccezione 1][eccezione 2]}.
\end{enumerate}

\textbf{Eccezioni:}
\begin{itemize}
    \item[(1)] Se email o password non sono valide, il sistema mostra un messaggio di errore e richiede di reinserire i dati.
    \item[(2)] Se l’autenticazione SPID/CIE fallisce, viene annullata o scade, il sistema mostra un messaggio di errore e ritorna alla schermata di login.
\end{itemize}

% ====================================================

\subsection{Registrare un cittadino \ref{sec:registrazione}}

\textbf{Riassunto:}  
Questo use case descrive come un utente anonimo registra un nuovo account cittadino tramite SPID o CIE e definisce le proprie credenziali locali.

\textbf{Precondizioni:}  
L’utente non possiede ancora un account sulla piattaforma.

\textbf{Postcondizioni:}  
Un nuovo account cittadino è creato e attivo; le credenziali locali risultano configurate.

\textbf{Descrizione:}
\begin{enumerate}
    \item L’utente accede alla schermata di registrazione.
    \item L’utente seleziona una modalità di identificazione:
    \begin{enumerate}
        \item \textbf{SPID} — reindirizzamento al provider SPID;
        \item \textbf{CIE} — reindirizzamento al servizio di autenticazione tramite CIE.
    \end{enumerate}
    \item Una volta validata l’identità tramite SPID/CIE \textbf{[eccezione 1]}, il sistema richiede:
    \begin{enumerate}
        \item inserimento dell’email;
        \item definizione della password;
        \item scelta del nome utente.
        \item \textit{facoltativamente} ricezione delle modifiche via email.
    \end{enumerate}
    \item L’utente conferma i dati inseriti.
    \item Il sistema crea l’account e invia un’email di conferma all’indirizzo fornito \textbf{[eccezione 2]}.
\end{enumerate}

\textbf{Eccezioni:}
\begin{itemize}
	\item[(1)] Se l’autenticazione SPID/CIE fallisce o viene annullata, la registrazione non procede.
    	\item[(2)] Se l’email non è valida o già associata a un altro account, il sistema notifica l’errore e richiede la correzione del dato. 
\end{itemize}

% ====================================================
% UC2
% ====================================================

\section{Utente autenticato}
\begin{enumerate}
	\item Fare logout dalla piattaforma (\ref{sec:logout})
	\item Gestione profilo e dati personali (\ref{sec:credenzialihandling})
	\item Visualizzazione proposte (con ordinamento) e sondaggi (\ref{sec:visualizzapropsond}, \ref{sec:sorting})
	\item Cambiare lingua (\ref{sec:cambiolingua})
\end{enumerate}

\begin{center}
	 \includegraphics[scale=1.25]{img/usecase/utenteautenticato.png}
\end{center}

% ====================================================

\subsection{Gestire le proprie credenziali \ref{sec:credenzialihandling}}

\textbf{Riassunto:}  
Questo use case descrive come un utente autenticato gestisce le proprie credenziali di accesso e preferenze personali tramite l’area profilo.

\textbf{Precondizioni:}  
L’utente è autenticato nella piattaforma.

\textbf{Postcondizioni:}  
Le modifiche alle credenziali o alle preferenze vengono salvate e risultano effettive per le successive sessioni di accesso.

\textbf{Descrizione:}
\begin{enumerate}
	\item L’utente accede alla sezione “Profilo”.
	\item Il sistema mostra i dati attuali: email, username, password mascherata e stato della preferenza notifiche.
	\item L’utente può:
	\begin{enumerate}
		\item modificare l’\textbf{email}, il sistema invia un messaggio di verifica al vecchio e al nuovo indirizzo email \textbf{[eccezione 1]};
		\item modificare l’\textbf{username};
		\item modificare la \textbf{password};
		\item aggiornare la preferenza di \textbf{ricezione notifiche}.
	\end{enumerate}
	\item Se le modifiche sono valide il sistema le salva. \textbf{[eccezione 2]}, \textbf{[eccezione 3]}.
\end{enumerate}

\textbf{Eccezioni:}
\begin{itemize}
		\item[(1)] Se da parte dei due indirizzi email coinvolti il sistema non riceve da entrambi conferma della modifica, il processo di cambio email viene annullato.
	\item[(2)] Se l’email inserita non è valida o già associata a un altro account, il sistema notifica l’errore e annulla la modifica.
		\item[(3)] Se il nome utente inserito non è valido o già associato a un altro account, il sistema notifica l’errore e annulla la modifica.
\end{itemize}

% ====================================================

\subsection{Visualizzare proposte e sondaggi \ref{sec:visualizzapropsond}, \ref{sec:sorting}}

\textbf{Riassunto:}  
Questo use case descrive come un utente autenticato consulta le proposte pubblicate e i sondaggi disponibili, applicando criteri di ordinamento e filtraggio.

\textbf{Precondizioni:}  
L’utente è autenticato nella piattaforma e sono presenti proposte o sondaggi pubblicati.

\textbf{Postcondizioni:}  
Le informazioni visualizzate sono aggiornate in base ai criteri di ordinamento o filtro selezionati.

\textbf{Descrizione:}
\begin{enumerate}
	\item L’utente accede alla sezione “Proposte e Sondaggi”.
	\item Il sistema mostra:
	\begin{enumerate}
		\item le \textbf{proposte} pubblicate, con titolo, autore, stato e categoria;
		\item i \textbf{sondaggi} attivi e chiusi con data di apertura e chiusura.
	\end{enumerate}
	\item L’utente può:
	\begin{enumerate}
		\item ordinare le proposte per rilevanza, numero di voti, data di pubblicazione o categoria;
		\item filtrare i risultati per categoria o stato della proposta;
		\item accedere al dettaglio di una proposta o di un sondaggio selezionato.
	\end{enumerate}
	\item Il sistema aggiorna dinamicamente l’elenco in base ai criteri scelti  \textbf{[eccezione 1]}.
\end{enumerate}

\textbf{Eccezioni:}
\begin{itemize}
	\item[(1)] Se non sono presenti proposte o sondaggi attivi, il sistema mostra un messaggio informativo.
\end{itemize}

% ====================================================

\subsection{Cambiare lingua \ref{sec:cambiolingua}}

\textbf{Riassunto:}  
Questo use case descrive come un utente autenticato modifica la lingua dell’interfaccia della piattaforma.

\textbf{Precondizioni:}  
L’utente è autenticato nella piattaforma.

\textbf{Postcondizioni:}  
La lingua selezionata diventa attiva immediatamente e viene salvata nel profilo utente.

\textbf{Descrizione:}
\begin{enumerate}
	\item L’utente accede all’area “Impostazioni lingua” dal menu del profilo.
	\item Il sistema mostra l’elenco delle lingue disponibili.
	\item L’utente seleziona la lingua desiderata.
	\item Il sistema aggiorna l’interfaccia nella lingua scelta e registra la preferenza nel profilo utente.
\end{enumerate}

% ====================================================
% UC3
% ====================================================

\section{Amministratore}

\begin{enumerate}
	\item Registrare moderatore o associazione (\ref{sec:specialregistrazione})
	\item Pubblicare sondaggi (\ref{sec:sondaggi})
	\item Visualizzare risultati policy simulation (\ref{sec:vederepolicies})
	\item Visualizzare dashboard e dati (\ref{sec:vederedashboard})
	\item Modificare lo stato di una proposta (\ref{sec:editproposta})
	\item Generazione dei report ({\ref{sec:report-attivita}})
\end{enumerate}

\begin{center}
	\includegraphics[scale=0.53]{img/usecase/Amministratore.png}
\end{center}

\subsection{Creare account moderatore o associazione/comitato \ref{sec:specialregistrazione}}
\textbf{Riassunto:}  
Questo caso d’uso descrive la procedura di creazione di un nuovo account di tipo moderatore o associazione/comitato tramite l’interfaccia amministrativa.

\textbf{Precondizioni:}  
L’amministratore è autenticato e ha i permessi di gestione utenti.

\textbf{Postcondizioni:}  
L’account risulta creato, registrato e notificato via email.

\textbf{Descrizione:}
\begin{enumerate}
  \item L’amministratore accede alla sezione di gestione utenti.
  \item Seleziona il tipo di account da creare (moderatore o associazione/comitato).
  \item Inserisce i dati richiesti e conferma l’operazione.
  \item Il sistema verifica la correttezza e completezza dei dati inseriti.
  \item In caso di esito positivo, il sistema crea il nuovo account.
  \item Il sistema invia un’email di conferma e, se previsto, le credenziali di accesso.
\end{enumerate}

\textbf{Eccezioni:}
\begin{itemize}
  \item [E1.] Dati mancanti o non validi: il sistema mostra un messaggio d’errore e richiede la correzione.
  \item [E2.] Indirizzo email già associato a un altro account: il sistema notifica il conflitto e annulla la creazione.
  \item [E3.] Errore nell’invio dell’email: il sistema registra l’errore e notifica l’amministratore.
\end{itemize}

\subsection{Pubblicare sondaggi \ref{sec:sondaggi}} 
\textbf{Riassunto:}  
Questo caso d’uso descrive la pubblicazione di un nuovo sondaggio o consultazione pubblica da parte dell’amministratore.

\textbf{Precondizioni:}  
L’amministratore è autenticato e dispone dei permessi di gestione sondaggi.

\textbf{Postcondizioni:}  
Il sondaggio è pubblicato e reso visibile ai cittadini.

\textbf{Descrizione:}
\begin{enumerate}
  \item L’amministratore accede al modulo di gestione sondaggi.
  \item Seleziona “Crea nuovo sondaggio”.
  \item Compila titolo, descrizione, periodo di apertura e categorie tematiche.
  \item Conferma la pubblicazione.
  \item Il sistema pubblica il sondaggio e invia notifiche ai cittadini iscritti.
  \item L’amministratore può successivamente modificare o annullare la pubblicazione.
\end{enumerate}

\textbf{Eccezioni:}
\begin{itemize}
  \item [E1.] Campi obbligatori mancanti: il sistema richiede di completare i dati.
  \item [E2.] Periodo di validità non coerente (data fine antecedente all’inizio): il sistema segnala l’errore.
  \item [E3.] Errore durante l’invio delle notifiche: il sistema completa la pubblicazione ma registra l’errore di comunicazione.
\end{itemize}

\subsection{Interazione con dashboard amministrativa \ref{sec:dashboard}} 
\textbf{Riassunto:}  
Questo caso d’uso descrive l’utilizzo della dashboard amministrativa per consultare, filtrare ed esportare i dati aggregati del sistema.

\textbf{Precondizioni:}  
L’amministratore è autenticato.

\textbf{Postcondizioni:}  
I dati sono visualizzati, filtrati o esportati secondo le operazioni eseguite.

\textbf{Descrizione:}
\begin{enumerate}
  \item L’amministratore accede alla dashboard amministrativa.
  \item Visualizza gli indicatori principali (attività utenti, voti, proposte per categoria).
  \item Applica filtri e ordinamenti sui dati visualizzati.
  \item Esporta i dati in formato CSV o PDF.
  \item Può configurare la dashboard per personalizzare i widget o i parametri mostrati.
\end{enumerate}

\textbf{Eccezioni:}
\begin{itemize}
  \item [E1.] Nessun dato disponibile per i criteri selezionati: il sistema mostra un messaggio “Nessun risultato trovato”.
  \item [E2.] Errore di esportazione (ad es. permessi insufficienti o formato non supportato): il sistema notifica l’errore e annulla l’operazione.
  \item [E3.] Errore nel caricamento dei dati (timeout o perdita connessione): il sistema mostra un messaggio di errore e consente di riprovare.
\end{itemize}

\subsection{Approvare/rifiutare una proposta}

\textbf{Riassunto:}  
Questo caso d’uso descrive come l’amministratore esamina una proposta in stato \textit{in valutazione} e decide se approvarla oppure rifiutarla, aggiornandone lo stato.

\textbf{Precondizioni:}  
Esistono proposte in stato \textit{in valutazione}.

\textbf{Postcondizioni:}  
La proposta risulta \textit{accettata} oppure \textit{rifiutata}.  
La transizione di stato è registrata nello storico.

\textbf{Descrizione:}
\begin{enumerate}
  \item L’amministratore accede all’elenco delle proposte in valutazione.
  \item Seleziona una proposta e ne visualizza contenuto, metadati e storico modifiche.
  \item Analizza revisioni, contributi e sostegni ricevuti.
  \item Decide se approvare la proposta oppure rifiutarla inserendo una motivazione.
  \item Il sistema aggiorna lo stato e registra l’esito della valutazione.
\end{enumerate}

\textbf{Eccezioni:}
\begin{enumerate}
  \item La proposta è stata modificata durante la valutazione: il sistema richiede conferma dell’azione.
  \item La proposta risulta già valutata da un altro amministratore: il sistema impedisce l’operazione.
\end{enumerate}

\subsection{Pubblicare e aggiornare un report amministrativo}

\textbf{Riassunto:}  
Questo caso d’uso descrive come l’amministratore genera, pubblica e, se necessario, aggiorna un report relativo a una proposta. La pubblicazione può comportare un aggiornamento dello stato della proposta.

\textbf{Precondizioni:}  
Esistono proposte che richiedono la redazione o l’aggiornamento di un report.

\textbf{Postcondizioni:}  
Il report risulta pubblicato o aggiornato.  
Lo stato della proposta può essere aggiornato.

\textbf{Descrizione:}
\begin{enumerate}
  \item L’amministratore accede al modulo di report.
  \item Inserisce o modifica i contenuti del report (testo, indicatori, allegati).
  \item Pubblica il report.
  \item Il sistema aggiorna lo stato della proposta, se applicabile.
\end{enumerate}

\textbf{Eccezioni:}
\begin{enumerate}
  \item Il report è in modifica da parte di un altro amministratore: l’operazione viene bloccata.
\end{enumerate}



% ====================================================

% ====================================================
% UC4
% ====================================================

\section{Cittadino}
\begin{enumerate}
	\item Creare una proposta (\ref{sec:creazioneproposta})
	\item Pubblicare una proposta (\ref{sec:pubblicazioneproposta})
	\item Proporre una modifica ad una proposta (\ref{sec:modificaproposta})
	\item Modificare i propri preferiti (\ref{sec:preferiti})
	\item Votare una proposta (\ref{sec:voto})
	\item Segnalare un contenuto (\ref{sec:segnalazione})
\end{enumerate}

\begin{center}
	% \includegraphics[scale=1.1]{img/usecase/1.png}
\end{center}

% ====================================================

% ====================================================
% UC5
% ====================================================

\section{Associazione/Comitato}
\begin{enumerate}
	\item Pubblicare proposte collettive (\ref{sec:creazioneproposta})
	\item Fare endorsement (\ref{sec:endorsement})
\end{enumerate}

\begin{center}
	% \includegraphics[scale=1.1]{img/usecase/1.png}
\end{center}

% ====================================================
% UC6
% ====================================================

\section{Moderatore}
\begin{enumerate}
	\item Visualizzazione coda di revisione (\ref{sec:validazione-contenuti})
	\item Rimuovere un contenuto (\ref{sec:rimozione})
	\item Moderazione sugli utenti (\ref{sec:modcittadini}, \ref{sec:modassociazioni})
\end{enumerate}

\begin{center}
	\includegraphics[scale=0.9]{img/usecase/Moderatore.png}
\end{center}


\subsection{Validare contenuti segnalati}
\textbf{Riassunto:}  
Questo caso d’uso descrive come il moderatore analizza e valida i contenuti segnalati dal sistema o dai cittadini per verificare eventuali violazioni delle policy.

\textbf{Precondizioni:}  
Esistono contenuti segnalati in attesa di revisione.

\textbf{Postcondizioni:}  
Il contenuto risulta confermato (valido) oppure rimosso.

\textbf{Descrizione:}
\begin{enumerate}
  \item Il moderatore accede all’elenco dei contenuti segnalati.
  \item Seleziona un elemento e visualizza la segnalazione e il contenuto associato.
  \item Analizza il testo, gli allegati e i metadati (autore, data, motivo della segnalazione).
  \item Decide se confermare la pubblicazione o rimuovere il contenuto.
  \item Il sistema aggiorna lo stato della segnalazione e registra l’esito.
\end{enumerate}

\textbf{Eccezioni:}
\begin{itemize}
  \item [E1.] Il contenuto è già stato revisionato da un altro moderatore: il sistema segnala la duplicazione e aggiorna la lista.
  \item [E2.] Il contenuto non è più disponibile (es. eliminato dall’autore): il sistema archivia automaticamente la segnalazione.
\end{itemize}


\subsection{Moderare proposte e commenti pubblicati}
\textbf{Riassunto:}  
Questo caso d’uso descrive la moderazione ordinaria dei contenuti pubblici da parte dei moderatori.

\textbf{Precondizioni:}  
Il moderatore è autenticato e ha visibilità sull’elenco dei contenuti pubblicati.

\textbf{Postcondizioni:}  
Il contenuto risulta confermato (valido) o eliminato.

\textbf{Descrizione:}
\begin{enumerate}
  \item Il moderatore visualizza la lista dei contenuti pubblicati o filtrati per stato.
  \item Seleziona un contenuto da analizzare (proposta, commento o allegato).
  \item Verifica la conformità del contenuto rispetto alle linee guida etiche e di moderazione.
  \item In caso di violazione, sospende o elimina il contenuto, inserendo una motivazione.
  \item Il sistema registra l’azione con timestamp, utente moderatore e causa dell’intervento.
\end{enumerate}

\textbf{Eccezioni:}
\begin{itemize}
  \item [E1.] Il contenuto è già stato modificato o rimosso dall’autore: il sistema chiude automaticamente la moderazione.
  \item [E2.] Più moderatori intervengono contemporaneamente: il sistema blocca l’operazione e assegna priorità al primo intervento registrato.
\end{itemize}



\begin{center}
	% \includegraphics[scale=1.1]{img/usecase/1.png}
\end{center}

% ====================================================

