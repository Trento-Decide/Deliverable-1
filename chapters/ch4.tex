\chapter{Use Case Diagram}

\renewcommand{\thesubsection}{UC\arabic{subsection}}
\setlength{\parskip}{0.4em}
\setlength{\parindent}{0pt}

\section{Introduzione}
In questa sezione vengono presentati gli Use Case Diagram del sistema Trento Decide.
Essi rappresentano, in forma grafica, le interazioni tra gli attori e la piattaforma, illustrando le principali funzionalità e relazioni di inclusione o estensione tra i casi d’uso.
Questi diagrammi consentono di visualizzare in modo sintetico come i requisiti funzionali (RF) si traducono in comportamenti concreti del sistema, offrendo una visione d’insieme delle dinamiche operative e dei ruoli coinvolti.

% ====================================================
% UC1
% ====================================================

\section{Utente Anonimo}
\begin{enumerate}
    \item Fare login sulla piattaforma (\ref{sec:login})
    \item Registrare un nuovo cittadino nella piattaforma (\ref{sec:registrazione})
\end{enumerate}

\begin{center}
	\includegraphics[scale=1.1]{img/usecase/1.png}
\end{center}

% ====================================================

\subsection{Login \ref{sec:login}}

\textbf{Riassunto:}  
Questo use case descrive come un utente anonimo accede alla piattaforma tramite credenziali locali oppure tramite SPID o CIE.

\textbf{Precondizioni:}  
L’utente non è autenticato.

\textbf{Postcondizioni:}  
L’utente risulta autenticato e accede alla piattaforma.

\textbf{Descrizione:}
\begin{enumerate}
    \item L’utente visualizza la schermata di login.
    \item L’utente può scegliere una delle seguenti modalità:
    \begin{enumerate}
        \item \textbf{Credenziali locali:} inserisce email e password nei campi dedicati e conferma l’accesso.
        \item \textbf{SPID/CIE:} seleziona l’opzione SPID o CIE; il sistema reindirizza al relativo servizio di autenticazione.
    \end{enumerate}
    \item Se l’autenticazione è valida, l'utente accede al sistema \textbf{[eccezione 1][eccezione 2]}.
\end{enumerate}

\textbf{Eccezioni:}
\begin{itemize}
    \item[(1)] Se email o password non sono valide, il sistema mostra un messaggio di errore e richiede di reinserire i dati.
    \item[(2)] Se l’autenticazione SPID/CIE fallisce, viene annullata o scade, il sistema mostra un messaggio di errore e ritorna alla schermata di login.
\end{itemize}

% ====================================================

\subsection{Registrare un cittadino \ref{sec:registrazione}}

\textbf{Riassunto:}  
Questo use case descrive come un utente anonimo registra un nuovo account cittadino tramite SPID o CIE e definisce le proprie credenziali locali.

\textbf{Precondizioni:}  
L’utente non possiede ancora un account sulla piattaforma.

\textbf{Postcondizioni:}  
Un nuovo account cittadino è creato e attivo; le credenziali locali risultano configurate.

\textbf{Descrizione:}
\begin{enumerate}
    \item L’utente accede alla schermata di registrazione.
    \item L’utente seleziona una modalità di identificazione:
    \begin{enumerate}
        \item \textbf{SPID} — reindirizzamento al provider SPID;
        \item \textbf{CIE} — reindirizzamento al servizio di autenticazione tramite CIE.
    \end{enumerate}
    \item Una volta validata l’identità tramite SPID/CIE \textbf{[eccezione 1]}, il sistema richiede:
    \begin{enumerate}
        \item inserimento dell’email;
        \item definizione della password;
        \item scelta del nome utente.
    \end{enumerate}
    \item L’utente conferma i dati inseriti.
    \item Il sistema crea l’account e invia un’email di conferma all’indirizzo fornito \textbf{[eccezione 2]}.
\end{enumerate}

\textbf{Eccezioni:}
\begin{itemize}
	\item[(1)] Se l’autenticazione SPID/CIE fallisce o viene annullata, la registrazione non procede.
    	\item[(2)] Se l’email non è valida o già associata a un altro account, il sistema notifica l’errore e richiede la correzione del dato. 
\end{itemize}

\textbf{Estensioni:}
\begin{itemize}
    \item[(1)] \textit{Estensione del passo 3:} l’utente può selezionare l’opzione per ricevere notifiche via email; il sistema registra la preferenza.
\end{itemize}

% ====================================================
% UC2
% ====================================================

\section{Utente autenticato}
\begin{enumerate}
	\item Gestire le proprie credenziali (\ref{sec:credenzialihandling})
	\item Visualizzazione proposte (\ref{sec:visualizzazioneproposte}) MHHH NON SONO SICURO QUI!
	\item Ordinare e filtrare le proposte (\ref{sec:sorting}) MHHH NON SONO SICURO QUI!
	\item Cambiare lingua (\ref{sec:cambiolingua})
	\item Visualizzare policy simulator (\ref{sec:vederepolicies}) MHHH NON SONO SICURO QUI!
\end{enumerate}

\begin{center}
	% \includegraphics[scale=1.1]{img/usecase/1.png}
\end{center}

% ====================================================

% ====================================================
% UC3
% ====================================================

\section{Amministratore}
\begin{enumerate}
	\item Creare account moderatore o associazione/comitato (\ref{sec:specialregistrazione})
	\item Pubblicare sondaggi (\ref{sec:sondaggi})
	\item Interazione dashboard amministrativa (\ref{sec:dashboard})
\end{enumerate}

\begin{center}
	% \includegraphics[scale=1.1]{img/usecase/1.png}
\end{center}

% ====================================================

% ====================================================
% UC4
% ====================================================

\section{Cittadino}
\begin{enumerate}
	\item Creare una proposta (\ref{sec:creazioneproposta})
	\item Pubblicare una proposta in stato di bozza (\ref{sec:pubblicazioneproposta})
	\item Proporre una modifica ad una proposta (\ref{sec:modificaproposta})
	\item Modificare i propri preferiti (\ref{sec:preferiti})
	\item Votare una proposta (\ref{sec:voto})
\end{enumerate}

\begin{center}
	% \includegraphics[scale=1.1]{img/usecase/1.png}
\end{center}

% ====================================================

% ====================================================
% UC5
% ====================================================

\section{Associazione/Comitato}
\begin{enumerate}
	\item Creare proposte collettive (\ref{sec:creazioneproposta})
\end{enumerate}

\begin{center}
	% \includegraphics[scale=1.1]{img/usecase/1.png}
	Da capire se possiamo mergiare questo UC5 con Cittadino ma specificandone un' estensione ad hoc per questo
	tipo di accountss
\end{center}

% ====================================================

% ====================================================
% UC6
% ====================================================

\section{Moderatore}
\begin{enumerate}
	\item Fare meglio i RF di moderatore, incompleti
\end{enumerate}

\begin{center}
	% \includegraphics[scale=1.1]{img/usecase/1.png}
\end{center}

% ====================================================


