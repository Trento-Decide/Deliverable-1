\chapter{Use Case Diagram}

\renewcommand{\thesubsection}{UC\arabic{subsection}}
\setlength{\parskip}{0.4em}
\setlength{\parindent}{0pt}

\section{Introduzione}
In questa sezione vengono presentati gli Use Case Diagram del sistema Trento Decide.
Essi rappresentano, in forma grafica, le interazioni tra gli attori e la piattaforma, illustrando le principali funzionalità e relazioni di inclusione o estensione tra i casi d’uso.
Questi diagrammi consentono di visualizzare in modo sintetico come i requisiti funzionali (RF) si traducono in comportamenti concreti del sistema, offrendo una visione d’insieme delle dinamiche operative e dei ruoli coinvolti.

% ====================================================
% UC1
% ====================================================

\section{Utente Anonimo}
\begin{enumerate}
    \item Fare login sulla piattaforma (\ref{sec:login})
    \item Registrare un nuovo cittadino nella piattaforma (\ref{sec:registrazione})
\end{enumerate}

\begin{center}
	\includegraphics[scale=1.1]{img/usecase/1.png}
\end{center}

% ====================================================

\subsection{Login \ref{sec:login}}

\textbf{Riassunto:}
Questo use case descrive come un utente anonimo accede alla piattaforma tramite credenziali locali oppure tramite SPID o CIE.

\textbf{Precondizioni:}
L’utente non è autenticato.

\textbf{Postcondizioni:}
L’utente risulta autenticato e accede alla piattaforma.

\textbf{Descrizione:}
\begin{enumerate}
    \item L’utente visualizza la schermata di login.
    \item L’utente può scegliere una delle seguenti modalità:
    \begin{enumerate}
        \item \textbf{Credenziali locali:} inserisce email e password nei campi dedicati e conferma l’accesso.
        \item \textbf{SPID/CIE:} seleziona l’opzione SPID o CIE; il sistema reindirizza al relativo servizio di autenticazione.
    \end{enumerate}
    \item Se l’autenticazione è valida, l'utente accede al sistema \textbf{[eccezione 1][eccezione 2]}.
\end{enumerate}

\textbf{Eccezioni:}
\begin{itemize}
    \item[(1)] Se email o password non sono valide, il sistema mostra un messaggio di errore e richiede di reinserire i dati.
    \item[(2)] Se l’autenticazione SPID/CIE fallisce, viene annullata o scade, il sistema mostra un messaggio di errore e ritorna alla schermata di login.
\end{itemize}

% ====================================================

\subsection{Registrare un cittadino \ref{sec:registrazione}}

\textbf{Riassunto:}
Questo use case descrive come un utente anonimo registra un nuovo account cittadino tramite SPID o CIE e definisce le proprie credenziali locali.

\textbf{Precondizioni:}
L’utente non possiede ancora un account sulla piattaforma.

\textbf{Postcondizioni:}
Un nuovo account cittadino è creato e attivo; le credenziali locali risultano configurate.

\textbf{Descrizione:}
\begin{enumerate}
    \item L’utente accede alla schermata di registrazione.
    \item L’utente seleziona una modalità di identificazione:
    \begin{enumerate}
        \item \textbf{SPID} — reindirizzamento al provider SPID;
        \item \textbf{CIE} — reindirizzamento al servizio di autenticazione tramite CIE.
    \end{enumerate}
    \item Una volta validata l’identità tramite SPID/CIE \textbf{[eccezione 1]}, il sistema richiede:
    \begin{enumerate}
        \item inserimento dell’email;
        \item definizione della password;
        \item scelta del nome utente.
        \item \textit{facoltativamente} ricezione delle modifiche via email.
    \end{enumerate}
    \item L’utente conferma i dati inseriti.
    \item Il sistema crea l’account e invia un’email di conferma all’indirizzo fornito \textbf{[eccezione 2]}.
\end{enumerate}

\textbf{Eccezioni:}
\begin{itemize}
	\item[(1)] Se l’autenticazione SPID/CIE fallisce o viene annullata, la registrazione non procede.
    	\item[(2)] Se l’email non è valida o già associata a un altro account, il sistema notifica l’errore e richiede la correzione del dato.
\end{itemize}

% ====================================================
% UC2
% ====================================================

\section{Utente autenticato}
\begin{enumerate}
	\item Gestire le proprie credenziali (\ref{sec:credenzialihandling})
	\item Visualizzazione proposte (con ordinamento) e sondaggi (\ref{sec:visualizzapropsond}, \ref{sec:sorting})
	\item Cambiare lingua (\ref{sec:cambiolingua})
\end{enumerate}

\begin{center}
	 \includegraphics[scale=1.25]{img/usecase/2.png}
\end{center}

% ====================================================

\subsection{Gestire le proprie credenziali \ref{sec:credenzialihandling}}

\textbf{Riassunto:}
Questo use case descrive come un utente autenticato gestisce le proprie credenziali di accesso e preferenze personali tramite l’area profilo.

\textbf{Precondizioni:}
L’utente è autenticato nella piattaforma.

\textbf{Postcondizioni:}
Le modifiche alle credenziali o alle preferenze vengono salvate e risultano effettive per le successive sessioni di accesso.

\textbf{Descrizione:}
\begin{enumerate}
	\item L’utente accede alla sezione “Profilo”.
	\item Il sistema mostra i dati attuali: email, username, password mascherata e stato della preferenza notifiche.
	\item L’utente può:
	\begin{enumerate}
		\item modificare l’\textbf{email}, il sistema invia un messaggio di verifica al vecchio e al nuovo indirizzo email \textbf{[eccezione 1]};
		\item modificare l’\textbf{username};
		\item modificare la \textbf{password};
		\item aggiornare la preferenza di \textbf{ricezione notifiche}.
	\end{enumerate}
	\item Se le modifiche sono valide il sistema le salva. \textbf{[eccezione 2]}, \textbf{[eccezione 3]}.
\end{enumerate}

\textbf{Eccezioni:}
\begin{itemize}
		\item[(1)] Se da parte dei due indirizzi email coinvolti il sistema non riceve da entrambi conferma della modifica, il processo di cambio email viene annullato.
	\item[(2)] Se l’email inserita non è valida o già associata a un altro account, il sistema notifica l’errore e annulla la modifica.
		\item[(3)] Se il nome utente inserito non è valido o già associato a un altro account, il sistema notifica l’errore e annulla la modifica.
\end{itemize}

% ====================================================

\subsection{Visualizzare proposte e sondaggi \ref{sec:visualizzapropsond}, \ref{sec:sorting}}

\textbf{Riassunto:}
Questo use case descrive come un utente autenticato consulta le proposte pubblicate e i sondaggi disponibili, applicando criteri di ordinamento e filtraggio.

\textbf{Precondizioni:}
L’utente è autenticato nella piattaforma e sono presenti proposte o sondaggi pubblicati.

\textbf{Postcondizioni:}
Le informazioni visualizzate sono aggiornate in base ai criteri di ordinamento o filtro selezionati.

\textbf{Descrizione:}
\begin{enumerate}
	\item L’utente accede alla sezione “Proposte e Sondaggi”.
	\item Il sistema mostra:
	\begin{enumerate}
		\item le \textbf{proposte} pubblicate, con titolo, autore, stato e categoria;
		\item i \textbf{sondaggi} attivi e chiusi con data di apertura e chiusura.
	\end{enumerate}
	\item L’utente può:
	\begin{enumerate}
		\item ordinare le proposte per rilevanza, numero di voti, data di pubblicazione o categoria;
		\item filtrare i risultati per categoria o stato della proposta;
		\item accedere al dettaglio di una proposta o di un sondaggio selezionato.
	\end{enumerate}
	\item Il sistema aggiorna dinamicamente l’elenco in base ai criteri scelti  \textbf{[eccezione 1]}.
\end{enumerate}

\textbf{Eccezioni:}
\begin{itemize}
	\item[(1)] Se non sono presenti proposte o sondaggi attivi, il sistema mostra un messaggio informativo.
\end{itemize}

% ====================================================

\subsection{Cambiare lingua \ref{sec:cambiolingua}}

\textbf{Riassunto:}
Questo use case descrive come un utente autenticato modifica la lingua dell’interfaccia della piattaforma.

\textbf{Precondizioni:}
L’utente è autenticato nella piattaforma.

\textbf{Postcondizioni:}
La lingua selezionata diventa attiva immediatamente e viene salvata nel profilo utente.

\textbf{Descrizione:}
\begin{enumerate}
	\item L’utente accede all’area “Impostazioni lingua” dal menu del profilo.
	\item Il sistema mostra l’elenco delle lingue disponibili.
	\item L’utente seleziona la lingua desiderata.
	\item Il sistema aggiorna l’interfaccia nella lingua scelta e registra la preferenza nel profilo utente.
\end{enumerate}

% ====================================================
% UC3
% ====================================================

\section{Amministratore}
\begin{enumerate}
	\item Creare account moderatore o associazione/comitato (\ref{sec:specialregistrazione})
	\item Pubblicare sondaggi (\ref{sec:sondaggi})
	\item Interazione dashboard amministrativa (\ref{sec:dashboard})
\end{enumerate}

\begin{center}
	% \includegraphics[scale=1.1]{img/usecase/1.png}
\end{center}

% ====================================================

% ====================================================
% UC4
% ====================================================

\section{Cittadino}

\begin{center}
  \includegraphics[scale=0.75]{img/usecase/3.png}
\end{center}

\subsection{Creare una proposta \ref{sec:creazioneproposta} \ref{sec:pubblicazioneproposta}}

\textbf{Riassunto:} Questo use case descrive come un cittadino autenticato crea una nuova proposta.

\begin{description}
  \item[Precondizioni:]~
  \begin{itemize}
    \item L’utente è autenticato sulla piattaforma con ruolo ``Cittadino'' (Def.~3).
    \item L’utente ha accesso alla sezione per la creazione di nuove proposte.
  \end{itemize}

  \item[Postcondizioni:]~
  \begin{itemize}
    \item Una nuova proposta è creata in stato di ``bozza'', associata all’utente come autore.
    \item Tutti i dati inseriti e i metadati obbligatori (autore, data/ora di creazione, categoria, ecc.) sono salvati nel sistema.
  \end{itemize}

  \item[Descrizione:]~
  \begin{enumerate}
    \item Il cittadino accede alla sezione ``Nuova proposta''.
    \item Il sistema mostra il modulo di creazione con i campi previsti per la categoria selezionata:
    \begin{itemize}
      \item Titolo
      \item Descrizione
      \item Luogo (selezionabile tramite indirizzo o mappa)
      \item Categoria
    \end{itemize}
    \item Il cittadino compila i campi obbligatori e, se disponibili, eventuali campi facoltativi.
    \item Il cittadino sceglie se pubblicare la proposta o salvarla come bozza.
  \end{enumerate}
    Se decide di pubblicarla:
    \begin{enumerate}
      \item Il sistema valida i dati inseriti [eccezione 1].
      \item Se la validazione ha esito positivo, il sistema:
      \begin{itemize}
        \item associa la proposta all’utente come autore;
        \item registra data/ora di creazione e la categoria selezionata;
        \item conferma l’avvenuta pubblicazione all’utente [eccezione 2].
      \end{itemize}
    \end{enumerate}
     Se invece la salva come bozza, alla sua selezione nella lista delle proposte verrà riportato al modulo di creazione, dove avrà la possibilità di modificarla e/o pubblicarla.

  \item[Eccezioni:]~
  \begin{enumerate}
    \item Se uno o più campi sono mancanti o non validi, il sistema segnala gli errori e richiede la correzione dei dati.
    \item In caso di errore di sistema (es. problemi di salvataggio sul database), il sistema notifica un messaggio di errore e la proposta non viene creata o viene richiesto un nuovo tentativo.
  \end{enumerate}
\end{description}

\subsection{Proporre una modifica ad una proposta \ref{sec:modificaproposta}}

\textbf{Riassunto:} Questo use case descrive come un cittadino propone una modifica al contenuto di una proposta esistente, distinta dalla modifica diretta da parte dell’autore originale.

\begin{description}
  \item[Precondizioni:]~
  \begin{itemize}
    \item L’utente è autenticato sulla piattaforma con ruolo ``Cittadino''.
    \item Esiste almeno una proposta pubblicata o visibile all’utente su cui è consentito proporre modifiche.
    \item L’utente non è necessariamente l’autore della proposta.
  \end{itemize}

  \item[Postcondizioni:]~
  \begin{itemize}
    \item Una nuova proposta di modifica è creata e associata alla proposta originale in stato ``in revisione'' (non pubblica).
    \item L’autore della proposta originale può successivamente accettare o rifiutare la modifica.
  \end{itemize}

  \item[Descrizione:]~
  \begin{enumerate}
    \item Il cittadino accede al dettaglio di una proposta pubblicata.
    \item Il sistema mostra le informazioni della proposta e l’opzione ``Proponi una modifica''.
    \item Il cittadino seleziona l’opzione ``Proponi una modifica''.
    \item Il sistema mostra un’interfaccia che consente di:
    \begin{itemize}
      \item modificare il testo dei campi proposti (es. titolo, descrizione, luogo, ecc.);
      \item inserire una breve descrizione della modifica suggerita (motivazione).
    \end{itemize}
    \item Il cittadino apporta le modifiche desiderate e conferma l’invio della proposta di modifica.
    \item Il sistema valida i dati inseriti [eccezione 1].
    \item Se la validazione ha esito positivo, il sistema:
    \begin{itemize}
      \item registra la proposta di modifica come nuova versione in stato ``in revisione'';
      \item associa alla versione proposta autore, data/ora, descrizione delle modifiche e differenze rispetto alla versione corrente;
      \item notifica l’autore originale della proposta (e, se previsto, altri soggetti interessati) [eccezione 2].
    \end{itemize}
  \end{enumerate}

  \item[Eccezioni:]~
  \begin{enumerate}
    \item Se i dati inseriti sono incompleti o non validi (es. testo vuoto, formato non corretto), il sistema segnala gli errori e richiede la correzione.
    \item Se per quella proposta non è consentito proporre modifiche (es. proposta chiusa, stato non modificabile), il sistema nasconde o disabilita l’opzione di modifica e mostra un messaggio informativo.
    \item In caso di errore di sistema nel salvataggio della proposta di modifica, il sistema notifica l’errore e la modifica non viene registrata.
  \end{enumerate}
\end{description}

\subsection{Gestire le proposte di modifica ricevute \ref{sec:modificaproposta}}

\textbf{Riassunto:} Questo use case descrive come un cittadino autore di una proposta gestisce le proposte di modifica ricevute, decidendo se accettarle o rifiutarle.

\begin{description}
  \item[Precondizioni:]~
  \begin{itemize}
    \item L’utente è autenticato sulla piattaforma con ruolo ``Cittadino''.
    \item L’utente è autore di almeno una proposta pubblicata.
    \item Per almeno una delle proposte di cui è autore esistono una o più proposte di modifica in stato ``in revisione'' associate alla proposta originale.
  \end{itemize}

  \item[Postcondizioni:]~
  \begin{itemize}
    \item Ogni proposta di modifica selezionata è posta in stato ``accettata'' oppure ``rifiutata'' in base alla decisione dell’autore.
    \item In caso di accettazione, il contenuto della proposta originale è aggiornato generando una nuova versione corrente che incorpora le modifiche accettate e aggiornando la cronologia delle versioni.
    \item Il proponente della modifica (e, se previsto, altri soggetti interessati) è notificato dell’esito (accettazione/rifiuto).
  \end{itemize}

  \item[Descrizione:]~
  \begin{enumerate}
    \item Il cittadino accede alla sezione ``Le mie proposte''.
    \item Il sistema mostra l’elenco delle proposte di cui l’utente è autore, con l’indicazione della presenza di proposte di modifica in sospeso.
    \item Il cittadino seleziona una proposta per la quale sono presenti modifiche in revisione.
    \item Il sistema mostra l’elenco delle proposte di modifica ricevute per quella proposta, con le principali informazioni (autore della modifica, data/ora, breve descrizione).
    \item Il cittadino seleziona una proposta di modifica dall’elenco.
    \item Il sistema mostra il dettaglio della proposta di modifica, evidenziando le differenze rispetto alla versione corrente della proposta originale.
    \item Il cittadino esamina le modifiche e:
    \begin{itemize}
      \item decide di accettare la proposta di modifica; oppure
      \item decide di rifiutare la proposta di modifica, eventualmente inserendo una motivazione.
    \end{itemize}
    \item In caso di accettazione:
    \begin{enumerate}
      \item Il sistema registra l’esito come ``accettata'', genera una nuova versione della proposta originale che incorpora le modifiche e la imposta come versione corrente.
      \item Il sistema aggiorna la cronologia delle versioni e notifica il proponente della modifica dell’avvenuta accettazione [eccezione 2].
    \end{enumerate}
    \item In caso di rifiuto:
    \begin{enumerate}
      \item Il sistema registra l’esito come ``rifiutata'' e, se presente, memorizza la motivazione fornita dall’autore.
      \item Il sistema mantiene invariata la versione corrente della proposta originale e notifica il proponente della modifica dell’avvenuto rifiuto [eccezione 2].
    \end{enumerate}
  \end{enumerate}

  \item[Eccezioni:]~
  \begin{enumerate}
    \item Se la proposta originale non è più modificabile (ad esempio perché archiviata, chiusa o in stato che non consente modifiche), il sistema disabilita l’accettazione delle modifiche e informa l’utente che le proposte di modifica non possono essere applicate.
    \item In caso di errore di sistema nel salvataggio dell’esito (accettazione o rifiuto) o nell’aggiornamento della versione della proposta, il sistema notifica un messaggio di errore e non modifica lo stato della proposta di modifica.
    \item Se l’utente non risulta più essere l’autore della proposta (ad esempio a causa di inconsistenze o revoche di privilegi), il sistema nega l’accesso alla gestione delle modifiche e mostra un messaggio di errore.
  \end{enumerate}
\end{description}


\subsection{Ripristinare una versione precedente di una proposta \ref{sec:modificaproposta}}

\textbf{Riassunto:} Questo use case descrive come un cittadino autore di una proposta può ripristinare una versione precedente della stessa, rendendola nuovamente la versione corrente.

\begin{description}
  \item[Precondizioni:]~
  \begin{itemize}
    \item L’utente è autenticato sulla piattaforma con ruolo ``Cittadino''.
    \item L’utente è autore della proposta oggetto del ripristino.
    \item Per la proposta esiste una cronologia versioni contenente almeno una versione precedente rispetto a quella corrente.
    \item La proposta si trova in uno stato che consente modifiche e ripristini di versione.
  \end{itemize}

  \item[Postcondizioni:]~
  \begin{itemize}
    \item Una nuova versione della proposta è creata a partire dal contenuto della versione selezionata e impostata come versione corrente.
    \item La cronologia delle versioni è aggiornata, mantenendo traccia del ripristino e dei legami tra versioni.
    \item Gli eventuali soggetti interessati (es. utenti che seguono la proposta) sono notificati del cambiamento, se previsto.
  \end{itemize}

  \item[Descrizione:]~
  \begin{enumerate}
    \item Il cittadino accede alla sezione ``Le mie proposte''.
    \item Il sistema mostra l’elenco delle proposte di cui l’utente è autore.
    \item Il cittadino seleziona la proposta per la quale desidera consultare la cronologia delle versioni.
    \item Il sistema mostra la cronologia delle versioni della proposta, ordinata temporalmente, con le principali informazioni (data/ora, autore della modifica, breve descrizione delle modifiche).
    \item Il cittadino seleziona una versione precedente da ripristinare.
    \item Il sistema mostra il dettaglio della versione selezionata e, se utile, le differenze rispetto alla versione corrente.
    \item Il cittadino conferma l’azione di ripristino della versione selezionata.
    \item Il sistema verifica che la proposta sia ancora in uno stato che consente modifiche e ripristini [eccezione 1].
    \item Se la verifica ha esito positivo, il sistema:
    \begin{itemize}
      \item crea una nuova versione della proposta, copiando il contenuto della versione selezionata;
      \item imposta la nuova versione come versione corrente della proposta;
      \item aggiorna la cronologia delle versioni registrando l’operazione di ripristino;
      \item notifica, se previsto, gli utenti interessati del cambiamento [eccezione 2].
    \end{itemize}
  \end{enumerate}

  \item[Eccezioni:]~
  \begin{enumerate}
    \item Se la proposta non è più modificabile (ad esempio perché chiusa, archiviata o in un processo di voto che non consente modifiche), il sistema impedisce il ripristino e mostra un messaggio informativo all’utente.
    \item In caso di errore di sistema nel salvataggio della nuova versione o nell’aggiornamento della cronologia, il sistema notifica un messaggio di errore e non modifica la versione corrente della proposta.
    \item Se la versione selezionata non è più disponibile (ad esempio a causa di inconsistenze nella cronologia), il sistema segnala l’errore e invita l’utente a selezionare un’altra versione.
  \end{enumerate}
\end{description}

\subsection{Modificare i propri preferiti \ref{sec:preferiti}}

\textbf{Riassunto:} Questo use case descrive come un cittadino gestisce l’elenco dei propri elementi preferiti (proposte e sondaggi).

\begin{description}
  \item[Precondizioni:]~
  \begin{itemize}
    \item L’utente è autenticato sulla piattaforma con ruolo ``Cittadino''.
    \item Sono presenti proposte o sondaggi visualizzabili.
  \end{itemize}

  \item[Postcondizioni:]~
  \begin{itemize}
    \item L’elenco dei preferiti del cittadino è aggiornato in base alle azioni di aggiunta/rimozione.
    \item Le eventuali notifiche collegate agli elementi preferiti sono aggiornate (RF4.2).
  \end{itemize}

  \item[Descrizione:]~
  \begin{enumerate}
    \item Il cittadino accede alla lista delle proposte e/o dei sondaggi, oppure al dettaglio di un singolo elemento.
    \item Per ogni proposta/sondaggio il sistema mostra lo stato di preferito con l'icona di una stella che, se colorata, indica l'aggiunta ai preferiti.
    \item L'utente può cliccare sulla stella per modificarne lo stato.
    \item Il sistema aggiorna l’elenco dei preferiti dell’utente e conferma l’operazione [eccezione 1].
    \item Il cittadino può accedere alla sezione ``Preferiti'' per visualizzare l’elenco aggiornato delle proposte e dei sondaggi contrassegnati.
  \end{enumerate}

  \item[Eccezioni:]~
  \begin{enumerate}
    \item Se l’elemento selezionato non è più disponibile (es. rimosso, oscurato, archiviato), il sistema informa l’utente che l’operazione non può essere completata e, se necessario, lo rimuove automaticamente dall’elenco dei preferiti.
    \item In caso di errore tecnico nella registrazione dell’operazione, il sistema segnala l’errore e l’elenco dei preferiti rimane invariato.
  \end{enumerate}
\end{description}

\subsection{Votare una proposta \ref{sec:voto}}

\textbf{Riassunto:} Questo use case descrive come un cittadino eleggibile esprime un voto positivo o negativo su una proposta.

\begin{description}
  \item[Precondizioni:]~
  \begin{itemize}
    \item L’utente è autenticato sulla piattaforma con ruolo ``Cittadino''.
    \item La proposta su cui si intende votare è in uno stato che consente la votazione (es. pubblicata, in valutazione, non ancora accettata o rifiutata).
  \end{itemize}

  \item[Postcondizioni:]~
  \begin{itemize}
    \item Il voto del cittadino è registrato in forma anonima.
    \item Il conteggio dei voti e/o il punteggio della proposta è aggiornato secondo le regole di calcolo previste.
    \item L’utente può, se consentito, modificare il proprio voto fino alla chiusura della votazione o al cambio di stato della proposta.
  \end{itemize}

  \item[Descrizione:]~
  \begin{enumerate}
    \item Il cittadino accede al dettaglio di una proposta su cui è possibile votare.
    \item Il sistema mostra lo stato corrente della proposta e le opzioni di voto (es. ``Voto favorevole'', ``Voto contrario''), nonché l’eventuale voto già espresso dall’utente.
    \item Il cittadino seleziona l’opzione di voto desiderata o modifica un voto precedentemente espresso.
    \item Se i controlli hanno esito positivo, il sistema:
    \begin{itemize}
      \item registra il voto in forma anonima;
      \item aggiorna il conteggio totale e, se previsto, gli altri indicatori associati;
      \item conferma l’avvenuta registrazione del voto all’utente [eccezione 1].
    \end{itemize}
  \end{enumerate}

  \item[Eccezioni:]~
  \begin{enumerate}
    \item Se la proposta si trova in uno stato che non consente più la votazione (es. accettata, rifiutata, chiusa), il sistema disabilita l’azione di voto e notifica che la votazione non è più disponibile.
    \item In caso di errore tecnico durante la registrazione del voto, il sistema segnala l’errore e il voto non viene considerato valido fino a nuovo tentativo dell’utente.
  \end{enumerate}
\end{description}

% ====================================================

% ====================================================
% UC5
% ====================================================

\section{Associazione/Comitato}
\begin{enumerate}
	\item Creare proposte collettive (\ref{sec:creazioneproposta})
\end{enumerate}

\begin{center}
	% \includegraphics[scale=1.1]{img/usecase/1.png}
	Da capire se possiamo mergiare questo UC5 con Cittadino ma specificandone un' estensione ad hoc per questo
	tipo di accountss
\end{center}

% ====================================================

% ====================================================
% UC6
% ====================================================

\section{Moderatore}
\begin{enumerate}
	\item Fare meglio i RF di moderatore, incompleti
\end{enumerate}

\begin{center}
	% \includegraphics[scale=1.1]{img/usecase/1.png}
\end{center}

% ====================================================


