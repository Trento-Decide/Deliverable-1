\chapter{Use Case Diagram}

\renewcommand{\thesubsection}{UC\arabic{subsection}}
\setlength{\parskip}{0.4em}
\setlength{\parindent}{0pt}

\section{Introduzione}
In questa sezione vengono presentati gli Use Case Diagram del sistema Trento Decide.
Essi rappresentano, in forma grafica, le interazioni tra gli attori e la piattaforma, illustrando le principali funzionalità e relazioni di inclusione o estensione tra i casi d’uso.
Questi diagrammi consentono di visualizzare in modo sintetico come i requisiti funzionali (RF) si traducono in comportamenti concreti del sistema, offrendo una visione d’insieme delle dinamiche operative e dei ruoli coinvolti.

% ====================================================
% UC1
% ====================================================

\section{Utente Anonimo}
\begin{enumerate}
    \item Fare login sulla piattaforma (\ref{sec:login})
    \item Registrare un nuovo cittadino nella piattaforma (\ref{sec:registrazione})
\end{enumerate}

\begin{center}
	\includegraphics[scale=1.1]{img/usecase/1.png}
\end{center}

% ====================================================

\subsection{Login \ref{sec:login}}

\textbf{Riassunto:}  
Questo use case descrive come un utente anonimo accede alla piattaforma tramite credenziali locali oppure tramite SPID o CIE.

\textbf{Precondizioni:}  
L’utente non è autenticato.

\textbf{Postcondizioni:}  
L’utente risulta autenticato e accede alla piattaforma.

\textbf{Descrizione:}
\begin{enumerate}
    \item L’utente visualizza la schermata di login.
    \item L’utente può scegliere una delle seguenti modalità:
    \begin{enumerate}
        \item \textbf{Credenziali locali:} inserisce email e password nei campi dedicati e conferma l’accesso.
        \item \textbf{SPID/CIE:} seleziona l’opzione SPID o CIE; il sistema reindirizza al relativo servizio di autenticazione.
    \end{enumerate}
    \item Se l’autenticazione è valida, l'utente accede al sistema \textbf{[eccezione 1][eccezione 2]}.
\end{enumerate}

\textbf{Eccezioni:}
\begin{itemize}
    \item[(1)] Se email o password non sono valide, il sistema mostra un messaggio di errore e richiede di reinserire i dati.
    \item[(2)] Se l’autenticazione SPID/CIE fallisce, viene annullata o scade, il sistema mostra un messaggio di errore e ritorna alla schermata di login.
\end{itemize}

% ====================================================

\subsection{Registrare un cittadino \ref{sec:registrazione}}

\textbf{Riassunto:}  
Questo use case descrive come un utente anonimo registra un nuovo account cittadino tramite SPID o CIE e definisce le proprie credenziali locali.

\textbf{Precondizioni:}  
L’utente non possiede ancora un account sulla piattaforma.

\textbf{Postcondizioni:}  
Un nuovo account cittadino è creato e attivo; le credenziali locali risultano configurate.

\textbf{Descrizione:}
\begin{enumerate}
    \item L’utente accede alla schermata di registrazione.
    \item L’utente seleziona una modalità di identificazione:
    \begin{enumerate}
        \item \textbf{SPID} — reindirizzamento al provider SPID;
        \item \textbf{CIE} — reindirizzamento al servizio di autenticazione tramite CIE.
    \end{enumerate}
    \item Una volta validata l’identità tramite SPID/CIE \textbf{[eccezione 1]}, il sistema richiede:
    \begin{enumerate}
        \item inserimento dell’email;
        \item definizione della password;
        \item scelta del nome utente.
        \item \textit{facoltativamente} ricezione delle modifiche via email.
    \end{enumerate}
    \item L’utente conferma i dati inseriti.
    \item Il sistema crea l’account e invia un’email di conferma all’indirizzo fornito \textbf{[eccezione 2]}.
\end{enumerate}

\textbf{Eccezioni:}
\begin{itemize}
	\item[(1)] Se l’autenticazione SPID/CIE fallisce o viene annullata, la registrazione non procede.
    	\item[(2)] Se l’email non è valida o già associata a un altro account, il sistema notifica l’errore e richiede la correzione del dato. 
\end{itemize}

% ====================================================
% UC2
% ====================================================

\section{Utente autenticato}
\begin{enumerate}
	\item Gestire le proprie credenziali (\ref{sec:credenzialihandling})
	\item Visualizzazione proposte (con ordinamento) e sondaggi (\ref{sec:visualizzapropsond}, \ref{sec:sorting})
	\item Cambiare lingua (\ref{sec:cambiolingua})
\end{enumerate}

\begin{center}
	 \includegraphics[scale=1.25]{img/usecase/2.png}
\end{center}

% ====================================================

\subsection{Gestire le proprie credenziali \ref{sec:credenzialihandling}}

\textbf{Riassunto:}  
Questo use case descrive come un utente autenticato gestisce le proprie credenziali di accesso e preferenze personali tramite l’area profilo.

\textbf{Precondizioni:}  
L’utente è autenticato nella piattaforma.

\textbf{Postcondizioni:}  
Le modifiche alle credenziali o alle preferenze vengono salvate e risultano effettive per le successive sessioni di accesso.

\textbf{Descrizione:}
\begin{enumerate}
	\item L’utente accede alla sezione “Profilo”.
	\item Il sistema mostra i dati attuali: email, username, password mascherata e stato della preferenza notifiche.
	\item L’utente può:
	\begin{enumerate}
		\item modificare l’\textbf{email}, il sistema invia un messaggio di verifica al vecchio e al nuovo indirizzo email \textbf{[eccezione 1]};
		\item modificare l’\textbf{username};
		\item modificare la \textbf{password};
		\item aggiornare la preferenza di \textbf{ricezione notifiche}.
	\end{enumerate}
	\item Se le modifiche sono valide il sistema le salva. \textbf{[eccezione 2]}, \textbf{[eccezione 3]}.
\end{enumerate}

\textbf{Eccezioni:}
\begin{itemize}
		\item[(1)] Se da parte dei due indirizzi email coinvolti il sistema non riceve da entrambi conferma della modifica, il processo di cambio email viene annullato.
	\item[(2)] Se l’email inserita non è valida o già associata a un altro account, il sistema notifica l’errore e annulla la modifica.
		\item[(3)] Se il nome utente inserito non è valido o già associato a un altro account, il sistema notifica l’errore e annulla la modifica.
\end{itemize}

% ====================================================

\subsection{Visualizzare proposte e sondaggi \ref{sec:visualizzapropsond}, \ref{sec:sorting}}

\textbf{Riassunto:}  
Questo use case descrive come un utente autenticato consulta le proposte pubblicate e i sondaggi disponibili, applicando criteri di ordinamento e filtraggio.

\textbf{Precondizioni:}  
L’utente è autenticato nella piattaforma e sono presenti proposte o sondaggi pubblicati.

\textbf{Postcondizioni:}  
Le informazioni visualizzate sono aggiornate in base ai criteri di ordinamento o filtro selezionati.

\textbf{Descrizione:}
\begin{enumerate}
	\item L’utente accede alla sezione “Proposte e Sondaggi”.
	\item Il sistema mostra:
	\begin{enumerate}
		\item le \textbf{proposte} pubblicate, con titolo, autore, stato e categoria;
		\item i \textbf{sondaggi} attivi e chiusi con data di apertura e chiusura.
	\end{enumerate}
	\item L’utente può:
	\begin{enumerate}
		\item ordinare le proposte per rilevanza, numero di voti, data di pubblicazione o categoria;
		\item filtrare i risultati per categoria o stato della proposta;
		\item accedere al dettaglio di una proposta o di un sondaggio selezionato.
	\end{enumerate}
	\item Il sistema aggiorna dinamicamente l’elenco in base ai criteri scelti  \textbf{[eccezione 1]}.
\end{enumerate}

\textbf{Eccezioni:}
\begin{itemize}
	\item[(1)] Se non sono presenti proposte o sondaggi attivi, il sistema mostra un messaggio informativo.
\end{itemize}

% ====================================================

\subsection{Cambiare lingua \ref{sec:cambiolingua}}

\textbf{Riassunto:}  
Questo use case descrive come un utente autenticato modifica la lingua dell’interfaccia della piattaforma.

\textbf{Precondizioni:}  
L’utente è autenticato nella piattaforma.

\textbf{Postcondizioni:}  
La lingua selezionata diventa attiva immediatamente e viene salvata nel profilo utente.

\textbf{Descrizione:}
\begin{enumerate}
	\item L’utente accede all’area “Impostazioni lingua” dal menu del profilo.
	\item Il sistema mostra l’elenco delle lingue disponibili.
	\item L’utente seleziona la lingua desiderata.
	\item Il sistema aggiorna l’interfaccia nella lingua scelta e registra la preferenza nel profilo utente.
\end{enumerate}

% ====================================================
% UC3
% ====================================================

\section{Amministratore}
\begin{enumerate}
	\item Creare account moderatore o associazione/comitato (\ref{sec:specialregistrazione})
	\item Pubblicare sondaggi (\ref{sec:sondaggi})
	\item Interazione dashboard amministrativa (\ref{sec:dashboard})
\end{enumerate}

\begin{center}
	% \includegraphics[scale=1.1]{img/usecase/1.png}
\end{center}

% ====================================================

% ====================================================
% UC4
% ====================================================

\section{Cittadino}
\begin{enumerate}
	\item Creare una proposta (\ref{sec:creazioneproposta})
	\item Pubblicare una proposta in stato di bozza (\ref{sec:pubblicazioneproposta})
	\item Proporre una modifica ad una proposta (\ref{sec:modificaproposta})
	\item Modificare i propri preferiti (\ref{sec:preferiti})
	\item Votare una proposta (\ref{sec:voto})
\end{enumerate}

\begin{center}
	% \includegraphics[scale=1.1]{img/usecase/1.png}
\end{center}

% ====================================================

% ====================================================
% UC5
% ====================================================

\section{Associazione/Comitato}
\begin{enumerate}
	\item Creare proposte collettive (\ref{sec:creazioneproposta})
\end{enumerate}

\begin{center}
	% \includegraphics[scale=1.1]{img/usecase/1.png}
	Da capire se possiamo mergiare questo UC5 con Cittadino ma specificandone un' estensione ad hoc per questo
	tipo di accountss
\end{center}

% ====================================================

% ====================================================
% UC6
% ====================================================

\section{Moderatore}
\begin{enumerate}
	\item Fare meglio i RF di moderatore, incompleti
\end{enumerate}

\begin{center}
	% \includegraphics[scale=1.1]{img/usecase/1.png}
\end{center}

% ====================================================


