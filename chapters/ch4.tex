% !TeX spellcheck = it_IT
\chapter{Use Case Diagram}

\renewcommand{\thesubsection}{\textbf{UC\arabic{ucsection}.\arabic{subsection}}}
\renewcommand{\thesubsubsection}{\textbf{\thesubsection.\arabic{subsubsection}}}

\setlength{\parskip}{0.4em}
\setlength{\parindent}{0pt}

\section{Introduzione}
In questa sezione sono riportati gli Use Case Diagram di \emph{Trento Decide}.
Questi diagrammi mostrano come gli attori interagiscono con la piattaforma e mettono in evidenza le principali funzionalità e le loro relazioni.
Gli Use Case Diagram aiutano a concretizzare graficamente i requisiti funzionali, offrendo una visione più chiara e immediata delle dinamiche del sistema rispetto ai capitoli precedenti.

\begin{ucscope}

\section{Utente Anonimo}

\begin{center}
	\includegraphics[scale=1.0]{img/usecase/UtenteAnonimo.png}
\end{center}

\subsection{Login~\ref{rf:login}}

\begin{description}
	\item[Riassunto:]
	Questo use case descrive come un utente anonimo accede alla piattaforma tramite credenziali locali oppure tramite SPID o CIE\@.

	\item[Precondizioni:]
	L’utente non è autenticato.

	\item[Postcondizioni:]
	L’utente risulta autenticato e accede alla piattaforma.

	\item[Descrizione:]~
	\begin{enumerate}
		\item L’utente visualizza la schermata di login.
		\item L’utente può scegliere una delle seguenti modalità:
		\begin{enumerate}
			\item \textbf{Credenziali locali:} inserisce email e password nei campi dedicati e conferma l’accesso.
			\item \textbf{SPID/CIE:} seleziona l’opzione SPID o CIE\@; il sistema reindirizza al relativo servizio di autenticazione.
		\end{enumerate}
		\item L'utente può decidere di far "ricordare" la propria sessione al sistema, salvando così un cookie di autenticazione in locale.
		\item Se l’autenticazione è valida, l'utente accede al sistema \textbf{[eccezione 1]} \textbf{[eccezione 2]}.
	\end{enumerate}

	\item[Estensioni:]~
	\begin{itemize}
		\item[(0)] Se l'utente possiede un cookie di autenticazione nella propria macchina, l'utente accede senza dover reinserire le credenziali.  \textbf{[eccezione 3]}
	\end{itemize}

	\item[Eccezioni:]~
	\begin{itemize}
		\item[(1)] Se email o password non sono valide, il sistema mostra un messaggio di errore e richiede di reinserire i dati.
		\item[(2)] Se l’autenticazione SPID/CIE fallisce, viene annullata o scade, il sistema mostra un messaggio di errore e ritorna alla schermata di login.
		\item[(3)] Se il sistema non riconosce il cookie di autenticazione, o il cookie è invalidato, allora l'utente è reindirizzato alla schermata di login.
	\end{itemize}
\end{description}

\subsection{Registrazione del cittadino~\ref{rf:registrazionecittadini}}

\begin{description}
	\item[Riassunto:]
	Questo use case descrive come un utente anonimo registra un nuovo account cittadino tramite SPID o CIE e definisce le proprie credenziali locali.

	\item[Precondizioni:]
	L’utente non possiede ancora un account sulla piattaforma.

	\item[Postcondizioni:]
	Un nuovo account cittadino è creato e attivo; le credenziali locali risultano configurate.

	\item[Descrizione:]~
	\begin{enumerate}
		\item L’utente visualizza la schermata di registrazione.
		\item L’utente seleziona una modalità di identificazione:
		\begin{enumerate}
			\item \textbf{SPID} — reindirizzamento al provider SPID\@;
			\item \textbf{CIE} — reindirizzamento al servizio di autenticazione tramite CIE\@.
		\end{enumerate}
		\item Una volta validata l’identità tramite SPID/CIE \textbf{[eccezione 1]}, il sistema richiede:
		\begin{enumerate}
			\item Il comune di residenza dell'utente tramite ANPR
		\end{enumerate}
		\item Una volta che il sistema verifica la residenza nel comune di Trento \textbf{[eccezione 2]} dell'utente il sistema richiede:
		\begin{enumerate}
			\item inserimento dell’email;
			\item definizione della password;
			\item definizione del nome utente.
			\item \emph{facoltativamente} ricezione delle modifiche via email.
		\end{enumerate}
		\item L’utente conferma i dati inseriti.
		\item Il sistema invia una email di verifica all'indirizzo immesso dall'utente, dove è richiesta la conferma della creazione dell'account \textbf{[eccezione 3]} \textbf{[eccezione 4]}.
		\item Il sistema invia una email di conferma.
	\end{enumerate}

	\item[Eccezioni:]~
	\begin{itemize}
		\item[(1)] Se l’autenticazione SPID/CIE fallisce o viene annullata, la registrazione non procede.
		\item[(2)] Se l'utente che tenta la registrazione non risiede a Trento, la registrazione non procede.
		%lo stesso per nome utente%
		\item[(3)] Se l’email non è valida o già associata a un altro account, il sistema notifica l’errore e richiede la correzione del dato.
		\item[(4)] Se non viene confermata la creazione dell'account entro 10 minuti la registrazione non procede.
	\end{itemize}
\end{description}

\section{Utente Autenticato}

\begin{center}
	 \includegraphics[scale=1.25]{img/usecase/UtenteAutenticato2.png}
\end{center}

\subsection{Logout~\ref{rf:logout}}

\begin{description}
	\item[Riassunto:]
	Questo use case descrive come un utente autenticato possa eseguire il logout dalla piattaforma, ovvero terminare la propria sessione attuale e, se esistente, invalidare il cookie di autenticazione.

	\item[Precondizioni:]
	L’utente è autenticato.

	\item[Postcondizioni:]
	L'utente non è autenticato e non possiede cookie di autenticazione validi.

	\item[Descrizione:]~
	\begin{enumerate}
		\item L’utente visualizza la funzione di logout attraverso il menu utente.
		\item L'utente conferma il logout e il sistema invalida la sessione attuale ed eventuale cookie di autenticazione.
	\end{enumerate}
\end{description}

\subsection{Visualizzazione proposte e sondaggi~\ref{rf:ordinamentoproposte},~\ref{rf:filtroproposte}}

\begin{description}
	\item[Riassunto:]
	Questo use case descrive come un utente autenticato consulta le proposte pubblicate e i sondaggi disponibili, applicando criteri di ordinamento e filtraggio.

	\item[Precondizioni:]
	L’utente è autenticato nella piattaforma.

	\item[Postcondizioni:]
	Le informazioni visualizzate sono aggiornate in base ai criteri di ordinamento o filtro selezionati.

	\item[Descrizione:]~
	\begin{enumerate}
		\item L’utente accede alla sezione ``Proposte" o "Sondaggi''.
		\item Il sistema mostra:
		\begin{enumerate}
			\item le \textbf{proposte} pubblicate, con titolo, autore, stato e categoria;
			\item oppure i \textbf{sondaggi} attivi e chiusi con data di apertura e chiusura.
		\end{enumerate}
		\item L’utente può:
		\begin{enumerate}
			\item ordinare le proposte per rilevanza, numero di voti, data di pubblicazione o categoria;
			\item filtrare i risultati per categoria o stato della proposta;
			\item accedere al dettaglio di una proposta o di un sondaggio selezionato.
		\end{enumerate}
		\item Il sistema aggiorna dinamicamente l’elenco in base ai criteri scelti.
	\end{enumerate}

	\item[Estensioni:]~
	\begin{itemize}
		\item[(1)] Se non sono presenti proposte o sondaggi attivi, il sistema mostra un messaggio informativo.
	\end{itemize}
\end{description}

\subsection{Cambio di lingua~\ref{rf:cambiolingua}}

\begin{description}
	\item[Riassunto:]
	Questo use case descrive come un utente autenticato modifica la lingua dell’interfaccia della piattaforma.

	\item[Precondizioni:]
	L’utente è autenticato nella piattaforma.

	\item[Postcondizioni:]
	La lingua selezionata diventa attiva immediatamente e viene salvata nel profilo utente.

	\item[Descrizione:]~
	\begin{enumerate}
		\item L’utente accede all’area ``Impostazioni lingua'' dal menu utente.
		\item Il sistema mostra l’elenco delle lingue disponibili.
		\item L’utente seleziona la lingua desiderata.
		\item Il sistema aggiorna l’interfaccia nella lingua scelta e registra la preferenza nel profilo utente.
	\end{enumerate}
\end{description}

\begin{center}
	\includegraphics[scale=1.25]{img/usecase/UtenteAutenticato1.png}
\end{center}

\subsection{Gestione profilo e dati personali~\ref{rf:datiecred},~\ref{rf:prefnotifiche}}

\begin{description}
	\item[Riassunto:]
	Questo use case descrive come un utente autenticato gestisce le proprie credenziali di accesso e preferenze personali tramite l’area profilo.

	\item[Precondizioni:]
	L’utente è autenticato nella piattaforma.

	\item[Postcondizioni:]
	Le modifiche alle credenziali o alle preferenze vengono salvate e risultano effettive per le successive sessioni di accesso.

	\item[Descrizione:]~
	\begin{enumerate}
		\item L’utente accede alla sezione ``Profilo'' tramite menu utente.
		\item Il sistema mostra i dati attuali: email, username e stato della preferenza notifiche.
		\item L’utente può:
		\begin{enumerate}
			\item modificare l’\textbf{email}, il sistema invia un email di verifica al vecchio e al nuovo indirizzo \textbf{[eccezione 1]};
			\item modificare l’\textbf{username} \textbf{[eccezione 2]};
			\item modificare la \textbf{password}, inviando una email di verifica all'utente;
			\item aggiornare la preferenza di \textbf{ricezione notifiche}.
		\end{enumerate}
		\item Se le modifiche sono valide il sistema le salva. \textbf{[eccezione 3]} \textbf{[eccezione 4]}.
	\end{enumerate}

	\item[Eccezioni:]~
	\begin{itemize}
		\item[(1)] Se da parte dei due indirizzi email coinvolti il sistema non riceve da entrambi conferma della modifica, il processo di cambio email viene annullato.
		\item[(2)] Se l’username inserito non è valido o già associato a un altro account, il sistema notifica l’errore e annulla la modifica.
		\item[(3)] Se l’email inserita non è valida o già associata a un altro account, il sistema notifica l’errore e annulla la modifica.
		\item[(4)] Se il nome utente inserito non è valido o già associato a un altro account, il sistema notifica l’errore e annulla la modifica.
	\end{itemize}
\end{description}

% ====================================================
% UC3
% ====================================================

\section{Amministratore}

\begin{center}
	\includegraphics[scale=1]{img/usecase/Amministratore1.png}
\end{center}

\subsection{Creare account moderatore o associazione~\ref{rf:registrazionemodass}}
\begin{description}
	\item[Riassunto:]
	Questo caso d’uso descrive la procedura di creazione di un nuovo account di tipo moderatore o associazione tramite l’interfaccia amministrativa.

	\item[Precondizioni:]
	L’amministratore è autenticato e ha i permessi di gestione utenti.

	\item[Postcondizioni:]
	L’account risulta creato, registrato e notificato via email.

	\item[Descrizione:]~
	\begin{enumerate}
		\item L’amministratore accede alla sezione di gestione utenti.
		\item Seleziona il tipo di account da creare (moderatore o associazione).
		\item Inserisce i dati richiesti e conferma l’operazione.
		\item Il sistema verifica la correttezza e completezza dei dati inseriti. \textbf{[eccezione 1]} \textbf{[eccezione 2]}
		\item Il sistema invia credenziali email all’utente interessato. \textbf{[eccezione 3]}
	\end{enumerate}

	\item[Eccezioni:]~
	\begin{itemize}
		\item[(1)] Dati mancanti o non validi: il sistema mostra un messaggio d’errore e richiede la correzione.
		\item[(2)] Indirizzo email già associato a un altro account: il sistema notifica il conflitto e annulla la creazione.
		\item[(3)] Errore nell’invio dell’email: il sistema registra l’errore e notifica l’amministratore.
	\end{itemize}
\end{description}

\subsection{Pubblicare un sondaggio~\ref{rf:sondaggi}}
\begin{description}
	\item[Riassunto:]
	Questo caso d’uso descrive la pubblicazione di un sondaggio rivolto ai cittadini.

	\item[Precondizioni:]
	L’amministratore è autenticato e dispone dei permessi di gestione sondaggi.

	\item[Postcondizioni:]
	Il sondaggio viene pubblicato.

	\item[Descrizione:]~
	\begin{enumerate}
		\item L’amministratore accede al modulo di gestione sondaggi.
		\item Seleziona ``Crea nuovo sondaggio''.
		\item Compila titolo, descrizione, categoria, periodo e quesiti.
		\item Conferma la pubblicazione. \textbf{[eccezione 1]} \textbf{[eccezione 2]}
		\item Il sistema invia notifiche agli utenti che hanno attivato le notifiche. \textbf{[eccezione 3]}
	\end{enumerate}

	\item[Eccezioni:]~
	\begin{itemize}
		\item[(1)] Campi obbligatori mancanti: il sistema richiede di completare i dati.
		\item[(2)] Periodo di validità non coerente (data fine antecedente all’inizio): il sistema segnala l’errore.
		\item[(3)] Errore durante l’invio delle notifiche: il sistema completa la pubblicazione ma registra l’errore di comunicazione.
	\end{itemize}
\end{description}

\begin{center}
	\includegraphics[scale=1.10]{img/usecase/Amministratore2.png}
\end{center}

\subsection{Consultare ed esportare dati dalla dashboard amministrativa~\ref{rf:dashboard}}
\begin{description}
	\item[Riassunto:]
	Questo caso d’uso descrive l’utilizzo della dashboard amministrativa per consultare, filtrare ed esportare i dati aggregati del sistema.

	\item[Precondizioni:]
	L’amministratore è autenticato.

	\item[Postcondizioni:]
	I dati richiesti sono visualizzati o esportati.

	\item[Descrizione:]~
	\begin{enumerate}
		\item L’amministratore accede alla dashboard amministrativa.
		\item Visualizza indicatori generali (utenti, voti, attività, categorie). [\textbf{eccezione 1}]
		\item Applica filtri e ordinamenti sui dati visualizzati.
		\item Esporta dati in un formato disponibile (es. CSV). [\textbf{eccezione 2}]
	\end{enumerate}

	\item[Eccezioni:]~
	\begin{itemize}
		\item[(1)] Errore nel caricamento dei dati: il sistema mostra un messaggio di errore e consente di riprovare.
		\item[(2)] Errore di esportazione (ad es.\ permessi insufficienti o formato non supportato): il sistema notifica l’errore e annulla l’operazione.
	\end{itemize}
\end{description}

\begin{center}
	\includegraphics[scale=1]{img/usecase/Amministratore3.png}
\end{center}

\subsection{Approvare/rifiutare una proposta}

\begin{description}
	\item[Riassunto:]
	Questo caso d’uso descrive come l’amministratore esamina una proposta in stato \emph{in valutazione} e decide se approvarla oppure rifiutarla.

	\item[Precondizioni:]
	La proposta è in stato \emph{in valutazione}.

	\item[Postcondizioni:]
	La proposta risulta \emph{accettata} oppure \emph{rifiutata}.
	La motivazione viene resa visibile al pubblico.

	\item[Descrizione:]~
	\begin{enumerate}
		\item L’amministratore accede all’elenco delle proposte in valutazione.
		\item Seleziona una proposta.
		\item Analizza contenuto, storia modifiche, contributi e voti.
		\item Decide se approvare la proposta oppure rifiutarla.
		\item Inserisce una motivazione.
		\item Pubblica la valutazione. [\textbf{eccezione 1}] [\textbf{eccezione 2}]
		\item Il sistema aggiorna lo stato e registra l’esito della valutazione.
	\end{enumerate}

	\item[Eccezioni:]~
	\begin{enumerate}
		\item[(1)] La proposta è stata modificata durante la valutazione: il sistema richiede conferma dell’azione.
		\item[(2)] La proposta risulta già valutata da un altro amministratore: il sistema impedisce l’operazione.
	\end{enumerate}
\end{description}

\subsection{Pubblicare un report di valutazione su una proposta}

\begin{description}
	\item[Riassunto:]
	Questo caso d’uso descrive come l’amministratore redige o aggiorna un report relativo a una proposta, eventualmente modificandone lo stato.

	\item[Precondizioni:]
	La proposta richiede un report o un suo aggiornamento.

	\item[Postcondizioni:]
	Il report risulta pubblicato o aggiornato.
	Lo stato della proposta può essere modificato.

	\item[Descrizione:]~
	\begin{enumerate}
		\item L’amministratore accede al modulo di report della proposta.
		\item Scrive o modifica testo e allegati.
		\item Pubblica il report. [\textbf{eccezione 1}]
		\item Il sistema aggiorna lo stato della proposta (se previsto).
	\end{enumerate}

	\item[Eccezioni:]~
	\begin{enumerate}
		\item[(1)] Il report è in modifica da parte di un altro amministratore: l’operazione viene bloccata.
	\end{enumerate}
\end{description}

\section{Cittadino}

\begin{center}
  \includegraphics{img/usecase/Cittadino1.png}
\end{center}

\subsection{Creazione proposta~\ref{rf:creazioneproposta}}

\begin{description}
  \item[Riassunto:]: Questo use case descrive come un cittadino crea una nuova proposta.

  \item[Descrizione:]~
  \begin{enumerate}
    \item Il cittadino seleziona ``Crea proposta''.

    \item Il sistema mostra il modulo di creazione con i campi previsti per la categoria selezionata:
    \begin{itemize}
      \item Titolo

      \item Descrizione

      \item Luogo (selezionabile tramite indirizzo o mappa)

      \item Categoria
    \end{itemize}

    \item Il cittadino compila i campi obbligatori e, se disponibili, eventuali campi facoltativi.

    \item Il cittadino sceglie se pubblicare la proposta o salvarla come bozza:
    \begin{itemize}
      \item Se decide di pubblicarla il sistema valida i dati inseriti [eccezione 1] e conferma l’avvenuta pubblicazione all’utente.

      \item Se invece la salva come bozza, alla sua selezione nella lista delle proposte verrà riportato al modulo di creazione, dove avrà la possibilità di modificarla e/o pubblicarla.
    \end{itemize}
  \end{enumerate}

  \item[Eccezioni:]~
  \begin{enumerate}
    \item Uno o più campi sono mancanti o non validi: il sistema segnala gli errori.
  \end{enumerate}
\end{description}

\subsection{Eliminazione proposta~\ref{rf:eliminazioneproposta}}

\begin{description}
  \item[Riassunto:] Questo use case descrive come un cittadino autore di una proposta pubblicata procede alla sua eliminazione.

  \item[Descrizione:]~
  \begin{enumerate}
    \item Il cittadino accede alla sezione delle proprie proposte.

    \item Il cittadino seleziona una proposta in stato pubblicata che desidera eliminare.

    \item Il cittadino seleziona il comando ``Elimina proposta'' associato alla proposta.

    \item Il sistema mostra un messaggio di conferma che riassume gli effetti dell’operazione e richiede una conferma esplicita [eccezione 1].

    \item Il cittadino conferma l’eliminazione.

    \item Il sistema:
    \begin{itemize}
      \item aggiorna lo stato della proposta per indicarne l’eliminazione [eccezione 2];

      \item mostra un messaggio di conferma dell’avvenuta eliminazione al cittadino.
    \end{itemize}
  \end{enumerate}

  \item[Eccezioni:]~
  \begin{enumerate}
    \item L’utente annulla la conferma di eliminazione: il sistema chiude la finestra di conferma e non applica alcuna modifica.

    \item La proposta non è più nello stato ``pubblicata'' al momento della conferma: il sistema annulla l’operazione, non effettua l’eliminazione e informa il cittadino che la proposta non è più eliminabile.
  \end{enumerate}
\end{description}

\begin{center}
  \includegraphics{img/usecase/Cittadino3.png}
\end{center}

\subsection{Modifica proposta~\ref{rf:modificaproposta}}
\begin{description}
  \item[Riassunto:] Questo use case descrive come un cittadino modifica una proposta, sia come autore che proponendo una modifica a una proposta creata da un altro utente.

  \item[Descrizione:]~
  \begin{enumerate}
    \item Il cittadino seleziona una proposta.

    \item Il cittadino seleziona il comando ``Modifica'' contrassegnato da un'icona di una matita.

    \item Il sistema mostra il modulo di modifica precompilato con i dati correnti.

    \item Il cittadino modifica uno o più campi del modulo.

    \item Il cittadino compila il campo ``descrizione'' della modifica.

    \item Il cittadino conferma il salvataggio della modifica.

    \item Il sistema verifica la validità dei dati inseriti [eccezione 1].

    \item Il sistema verifica se il cittadino è l’autore della proposta:
    \begin{itemize}
      \item Se il cittadino è l'autore della proposta:
      \begin{enumerate}
        \item imposta la nuova versione come versione principale visibile al pubblico.

        \item aggiorna lo storico delle versioni e conferma al cittadino l’avvenuto salvataggio della modifica [eccezione 2].
      \end{enumerate}

      \item Se il cittadino non è l'autore della proposta:
      \begin{enumerate}
        \item registra la proposta di modifica associandola alla proposta originale e al cittadino proponente, in stato di revisione.

        \item conferma al cittadino l’avvenuto invio della proposta di modifica [eccezione 2].

        \item L’autore della proposta potrà poi accettare o rifiutare la proposta di modifica ricevuta.
      \end{enumerate}
    \end{itemize}
  \end{enumerate}

  \item[Eccezioni:]~
  \begin{enumerate}
    \item Uno o più campi del modulo non compilati correttamente o descrizione della modifica (nel caso B) assente: il sistema evidenzia gli errori e non procede.

    \item La proposta non è più in stato ``pubblicata'': il sistema informa il cittadino che la proposta non è più modificabile.
  \end{enumerate}
\end{description}

\begin{center}
  \includegraphics{img/usecase/Cittadino2.png}
\end{center}

\subsection{Espressione del voto (proposta) \ref{rf:voto}}

\textbf{Riassunto:} Questo use case descrive come un cittadino esprime un voto positivo o negativo su una proposta.

\begin{description}
  \item[Descrizione:]~
  \begin{enumerate}
    \item Il cittadino accede al dettaglio di una proposta su cui è possibile votare.

    \item Il sistema mostra le opzioni di voto, nonché l’eventuale voto già espresso dall’utente.

    \item Il cittadino seleziona l’opzione di voto desiderata.

    \item Se i controlli hanno esito positivo, il sistema conferma l’avvenuta registrazione del voto all’utente [eccezione 1].
  \end{enumerate}

  \item[Eccezioni:]~
  \begin{enumerate}
    \item La proposta si trova in uno stato che non consente più la votazione: il sistema disabilita l’azione di voto e notifica che la votazione non è più disponibile.
  \end{enumerate}
\end{description}

\subsection{Gestione dei preferiti~\ref{rf:preferiti}}

\textbf{Riassunto:} Questo use case descrive come un cittadino gestisce i propri preferiti (proposte e sondaggi).

\begin{description}
  \item[Descrizione:]~
  \begin{enumerate}
    \item Il cittadino accede alla lista delle proposte e/o dei sondaggi, oppure al dettaglio di un singolo elemento.

    \item Per ogni proposta/sondaggio il sistema mostra lo stato di preferito con l'icona di una stella che, se colorata, indica l'aggiunta ai preferiti.

    \item L'utente può cliccare sulla stella per modificarne lo stato.

    \item Il sistema aggiorna l’elenco dei preferiti dell’utente e conferma l’operazione [eccezione 1].
  \end{enumerate}

  \item[Eccezioni:]~
  \begin{enumerate}
    \item Elemento selezionato non più disponibile: il sistema informa all’utente che l’operazione non può essere completata e, se necessario, lo rimuove automaticamente dall’elenco dei preferiti.
  \end{enumerate}
\end{description}

\subsection{Segnalazione di un contenuto~\ref{rf:segnalazione}}

\begin{description}
	\item[Riassunto:]
	Questo caso d’uso descrive come un cittadino segnala un contenuto pubblicato (proposta o modifica) che ritiene inappropriato, inserendolo nella coda di revisione dei moderatori (\ref{rf:validazionecontenuti}).

	\item[Descrizione:]~
	\begin{enumerate}
		\item Il cittadino apre una proposta o una modifica pubblicata.

		\item Il cittadino seleziona ``Segnala contenuto''.

		\item Il sistema mostra un modulo che include:
		\begin{enumerate}
			\item identificativo del contenuto;

			\item un elenco di motivazioni:
			\begin{enumerate}
				\item incitamento all’odio;

				\item contenuti offensivi o volgari;

				\item informazioni false;

				\item spam;

				\item contenuto duplicato;

				\item violazione della privacy;

				\item altro.
			\end{enumerate}

			\item un campo di testo libero per una descrizione opzionale del problema.
		\end{enumerate}

		\item Il cittadino compila il modulo e conferma l’invio della segnalazione.

		\item Il sistema verifica i dati inseriti [eccezione 1] e invia la segnalazione [eccezione 2].

		\item In caso di esito positivo, il sistema mostra un messaggio di conferma.
	\end{enumerate}

	\item[Eccezioni:]~
	\begin{enumerate}
		\item Campi obbligatori mancanti o non validi: il sistema evidenzia i campi da correggere.

		\item Contenuto non più disponibile: il sistema annulla la procedura e informa il che il contenuto non è più disponibile.
	\end{enumerate}
\end{description}

\section{Associazione}

\begin{center}
	\includegraphics[scale=1.1]{img/usecase/Associazione.png}
\end{center}

\subsection{Pubblicazione proposte collettive~\ref{rf:propostecollettive}}

\begin{description}
	\item[Riassunto:]
	Questo use case descrive come un’associazione pubblica una proposta collettiva, che viene etichettata come tale dal sistema.

	\item[Precondizioni:]
	L’associazione è autenticata nel sistema.

	\item[Postcondizioni:]
	La proposta risulta pubblicata e marcata come \emph{proposta collettiva}.

	\item[Descrizione:]~
	\begin{enumerate}
		\item L’associazione accede alla sezione di creazione proposta.
		\item L’associazione seleziona l’opzione \emph{Proposta collettiva}.
		\item L’associazione compila tutti i campi obbligatori previsti per la categoria selezionata.
		\item L’associazione conferma la pubblicazione.
		\item Il sistema registra la proposta e la etichetta come \emph{proposta collettiva}, mostrando il nome dell’associazione. \textbf{[eccezione 1]}
	\end{enumerate}

	\item[Eccezioni:]~
	\begin{itemize}
		\item[(1)] Se alcuni campi obbligatori risultano mancanti o non validi, il sistema impedisce la pubblicazione e segnala gli errori.
	\end{itemize}
\end{description}

\subsection{Endorsement~\ref{rf:endorsement}}

\begin{description}
	\item[Riassunto:]
	Questo use case descrive come un’associazione esprime un endorsement digitale su una proposta pubblicata.

	\item[Precondizioni:]~
	\begin{itemize}
		\item L’associazione è autenticata.
		\item La proposta è nello stato \emph{pubblicata}.
		\item L’associazione non ha già espresso un endorsement su quella proposta.
	\end{itemize}

	\item[Postcondizioni:]
	L’endorsement risulta registrato e la proposta è contrassegnata con un’etichetta riportante il nome dell’associazione.

	\item[Descrizione:]~
	\begin{enumerate}
		\item L’associazione accede alla pagina della proposta pubblicata.
		\item L’associazione seleziona l’opzione \emph{Esprimi endorsement}. \textbf{[eccezione 1]}
		\item L’associazione conferma l’azione.
		\item Il sistema registra l’endorsement.
		\item Il sistema applica l’etichetta identificativa con il nome dell’associazione.
	\end{enumerate}

	\item[Eccezioni:]~
	\begin{itemize}
		\item[(1)] Se l’associazione ha già espresso un endorsement per quella proposta, il sistema impedisce l’azione e mostra un messaggio di errore.
	\end{itemize}
\end{description}

\subsection{Rimozione Endorsement~\ref{rf:rimozioneendoresement}}

\begin{description}
	\item[Riassunto:]
	Questo use case descrive come un’associazione rimuove un endorsement precedentemente espresso.

	\item[Precondizioni:]~
	\begin{itemize}
		\item L’associazione è autenticata.
		\item L’associazione ha espresso un endorsement su quella proposta.
	\end{itemize}

	\item[Postcondizioni:]
	L’endorsement risulta rimosso e l’etichetta non è più visibile sulla proposta.

	\item[Descrizione:]~
	\begin{enumerate}
		\item L’associazione accede alla proposta su cui ha espresso endorsement.
		\item L’associazione seleziona l’opzione \emph{Rimuovi endorsement}.
		\item L’associazione conferma l’azione.
		\item Il sistema elimina l’endorsement e l'etichetta associata.
	\end{enumerate}
\end{description}

\section{Moderatore}


\begin{center}
	\includegraphics[scale=0.9]{img/usecase/Moderatore.png}
\end{center}

\subsection{Moderazione tramite coda di revisione~\ref{rf:validazionecontenuti}} \label{uc:modcoda}

\begin{description}
	\item[Riassunto:]
	Questo use case descrive come il moderatore analizza contenuti segnalati dai cittadini o rilevati automaticamente dal sistema e decide l’esito della revisione.

	\item[Precondizioni:]
	Esistono contenuti presenti nella coda di revisione.

	\item[Postcondizioni:]
	Il contenuto risulta confermato oppure rimosso.
	Eventuali limitazioni all’autore possono essere applicate (\emph{estensione}~\ref{uc:modlimitazioni}).

	\item[Descrizione:]~
	\begin{enumerate}
		\item Il moderatore accede alla coda di revisione.
		\item Seleziona un contenuto segnalato o rilevato automaticamente.
		\item Analizza il testo e i metadati (autore, data, motivo della segnalazione).
		\item Valuta la conformità rispetto alle policy della piattaforma e decide tra:
		\begin{itemize}
			\item confermare il contenuto;
			\item rimuovere il contenuto (include~\ref{uc:rimozione});
			\item applicare una limitazione all’autore (estende~\ref{uc:modlimitazioni}).
		\end{itemize}
		\item Il sistema aggiorna lo stato della segnalazione e registra l’esito. [\textbf{eccezione 1}] [\textbf{eccezione 2}]
	\end{enumerate}

	\item[Eccezioni:]~
	\begin{itemize}
		\item[(1)] L’elemento è stato già revisionato da un altro moderatore: il sistema aggiorna la lista e notifica la situazione.
		\item[(2)] Il contenuto non è più disponibile (eliminato dall’autore): il sistema archivia automaticamente la segnalazione.
	\end{itemize}
\end{description}


\subsection{Moderazione spontanea dei contenuti} \label{uc:modspontanea}

\begin{description}
	\item[Riassunto:]
	Questo use case descrive come il moderatore interviene autonomamente su contenuti pubblicati senza passare dalla coda di revisione.

	\item[Precondizioni:]
	Il moderatore è autenticato e ha accesso ai contenuti pubblici.

	\item[Postcondizioni:]
	Il contenuto risulta confermato o rimosso.
	Eventuali limitazioni possono essere applicate (\emph{estensione}~\ref{uc:modlimitazioni}).

	\item[Descrizione:]~
	\begin{enumerate}
		\item Il moderatore accede alla sezione di ricerca dei contenuti pubblicati.
		\item Filtra o individua un contenuto da analizzare.
		\item Visualizza testo, allegati, metadati e contesto.
		\item Valuta la conformità rispetto alle policy della piattaforma e decide tra:
		\begin{itemize}
			\item non intervenire;
			\item rimuovere il contenuto (include~\ref{uc:rimozione});
			\item applicare una limitazione (estende~\ref{uc:modlimitazioni}).
		\end{itemize}
		\item Il sistema registra l’azione eventualmente intrapresa. [\textbf{eccezione 1}] [\textbf{eccezione 2}]
	\end{enumerate}

	\item[Eccezioni:]~
	\begin{itemize}
		\item[(1)] Il contenuto è stato rimosso dall’autore: il sistema notifica l’incongruenza.
		\item[(2)] Un altro moderatore sta intervenendo sul contenuto: il sistema blocca l’operazione.
	\end{itemize}
\end{description}

\subsection{Rimozione di un contenuto} \label{uc:rimozione}

\begin{description}
	\item[Riassunto:]
	Questo use case descrive come il moderatore elimina un contenuto che viola le policy.

	\item[Descrizione:]~
	\begin{enumerate}
		\item Il moderatore seleziona l’opzione di rimozione e di eventuale limitazione (estensione~\ref{uc:modlimitazioni}).
		\item Inserisce la motivazione dell’intervento.
		\item Il sistema elimina il contenuto dalla piattaforma.
		\item Il sistema registra: motivazione, timestamp e identificativo del moderatore.
	\end{enumerate}
\end{description}

\subsection{Applicazione limitazioni all’autore} \label{uc:modlimitazioni}

\begin{description}
	\item[Attore principale:] Moderatore
	\item[Attore secondario:] Amministrazione

	\item[Riassunto:]
	Questo use case descrive come il moderatore gestisce eventuali limitazioni all’autore del contenuto, secondo due flussi distinti per cittadini e associazioni.
\end{description}

\subsubsection*{Caso A — Autore cittadino: applicazione diretta}

\begin{description}
	\item[Descrizione:]~
	\begin{enumerate}
		\item Il moderatore seleziona tipo e durata della limitazione.
		\item Il sistema applica la limitazione all’account del cittadino.
		\item Il sistema invia automaticamente all’utente un’email con:
		\begin{itemize}
			\item durata,
			\item riferimenti alle norme violate,
			\item modalità di ricorso.
		\end{itemize}
		\item L’operazione viene registrata.
	\end{enumerate}
\end{description}

\subsubsection*{Caso B — Autore associazione/comitato: escalation verso amministrazione}

\begin{description}
	\item[Descrizione:]~
	\begin{enumerate}
		\item Il moderatore seleziona ``Proponi limitazione''.
		\item Il sistema invia la richiesta all’amministrazione.
		\item Il sistema applica un blocco temporaneo delle interazioni fino alla decisione.
		\item Il sistema invia un’email all’associazione, indicando:
		\begin{itemize}
			\item la contestazione,
			\item che la decisione finale è in carico all’amministrazione.
		\end{itemize}
		\item L’amministrazione valuterà il provvedimento secondo il processo definito nel Requito Funzionale~\ref{rf:moderazioneassociazioni}.
	\end{enumerate}
\end{description}

\end{ucscope}
