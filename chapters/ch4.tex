\chapter{Use Case Diagram}

\renewcommand{\thesubsection}{\textbf{UC\arabic{subsection}}}

\setlength{\parskip}{0.4em}
\setlength{\parindent}{0pt}

\section*{Introduzione}
In questa sezione sono riportati gli Use Case Diagram di \textit{Trento Decide}.
Questi diagrammi mostrano come gli attori interagiscono con la piattaforma e mettono in evidenza le principali funzionalità e le loro relazioni.
Gli Use Case Diagram aiutano a concretizzare graficamente i requisiti funzionali, offrendo una visione più chiara e immediata delle dinamiche del sistema rispetto ai capitoli precedenti.

\section{Utente Anonimo}
\localtableofcontentswithrelativedepth{+1}

\begin{center}
	\includegraphics[scale=1.0]{img/usecase/UtenteAnonimo.png}
\end{center}

\subsection{Login \ref{rf:login}}

\textbf{Riassunto:}
Questo use case descrive come un utente anonimo accede alla piattaforma tramite credenziali locali oppure tramite SPID o CIE.

\textbf{Precondizioni:}
L’utente non è autenticato.

\textbf{Postcondizioni:}
L’utente risulta autenticato e accede alla piattaforma.

\textbf{Descrizione:}
\begin{enumerate}
    \item L’utente visualizza la schermata di login.
    \item L’utente può scegliere una delle seguenti modalità:
    \begin{enumerate}
        \item \textbf{Credenziali locali:} inserisce email e password nei campi dedicati e conferma l’accesso.
        \item \textbf{SPID/CIE:} seleziona l’opzione SPID o CIE; il sistema reindirizza al relativo servizio di autenticazione.
    \end{enumerate}
    \item L'utente può decidere di far "ricordare" la propria sessione al sistema, salvando così un cookie di autenticazione in locale.
    \item Se l’autenticazione è valida, l'utente accede al sistema \textbf{[eccezione 1 - 2]}.
\end{enumerate}

\textbf{Estensioni:}
\begin{itemize}
	\item[(0)] Se l'utente possiede un cookie di autenticazione nella propria macchina, l'utente accede senza dover reinserire le credenziali.  \textbf{[eccezione 3]}
\end{itemize}

\textbf{Eccezioni:}
\begin{itemize}
    \item[(1)] Se email o password non sono valide, il sistema mostra un messaggio di errore e richiede di reinserire i dati.
    \item[(2)] Se l’autenticazione SPID/CIE fallisce, viene annullata o scade, il sistema mostra un messaggio di errore e ritorna alla schermata di login.
    \item[(3)] Se il sistema non riconosce il cookie di autenticazione, o il cookie è invalidato, allora l'utente è reindirizzato alla schermata di login.
\end{itemize}

\subsection{Registrazione del cittadino \ref{rf:registrazionecittadini}}

\textbf{Riassunto:}
Questo use case descrive come un utente anonimo registra un nuovo account cittadino tramite SPID o CIE e definisce le proprie credenziali locali.

\textbf{Precondizioni:}
L’utente non possiede ancora un account sulla piattaforma.

\textbf{Postcondizioni:}
Un nuovo account cittadino è creato e attivo; le credenziali locali risultano configurate.

\textbf{Descrizione:}
\begin{enumerate}
    \item L’utente visualizza la schermata di registrazione.
    \item L’utente seleziona una modalità di identificazione:
    \begin{enumerate}
        \item \textbf{SPID} — reindirizzamento al provider SPID;
        \item \textbf{CIE} — reindirizzamento al servizio di autenticazione tramite CIE.
    \end{enumerate}
    \item Una volta validata l’identità tramite SPID/CIE \textbf{[eccezione 1]}, il sistema richiede: 
    \begin{enumerate}
        \item Il comune di residenza dell'utente tramite ANPR
    \end{enumerate}
    \item Una volta che il sistema verifica la residenza nel comune di Trento \textbf{[eccezione 2]} dell'utente il sistema richiede:
    \begin{enumerate}
        \item inserimento dell’email;
        \item definizione della password;
        \item definizione del nome utente.
        \item \textit{facoltativamente} ricezione delle modifiche via email.
    \end{enumerate}
    \item L’utente conferma i dati inseriti.
    \item Il sistema invia una email di verifica all'indirizzo immesso dall'utente, dove è richiesta la conferma della creazione dell'account \textbf{[eccezione 3 - 4]}.
\end{enumerate}

\textbf{Eccezioni:}
\begin{itemize}
	\item[(1)] Se l’autenticazione SPID/CIE fallisce o viene annullata, la registrazione non procede.
  \item[(2)] Se l'utente che tenta la registrazione non risiede a Trento, la registrazione non procede.
  %lo stesso per nome utente%
  \item[(3)] Se l’email non è valida o già associata a un altro account, il sistema notifica l’errore e richiede la correzione del dato.
  \item[(4)] Se non viene confermata la creazione dell'account entro 10 minuti la registrazione non procede.
\end{itemize}

\section{Utente Autenticato}
\localtableofcontentswithrelativedepth{+1}

\begin{center}
	 \includegraphics[scale=1.25]{img/usecase/utenteautenticato.png}
\end{center}

\subsection{Logout \ref{rf:logout}}

\textbf{Riassunto:}
Questo use case descrive come un utente autenticato possa eseguire il logout dalla piattaforma, ovvero terminare la propria sessione attuale e, se esistente, invalidare il cookie di autenticazione.

\textbf{Precondizioni:}
L’utente è autenticato.

\textbf{Postcondizioni:}
L'utente non è autenticato e non possiede cookie di autenticazione validi.

\textbf{Descrizione:}
\begin{enumerate}
	\item L’utente visualizza la schermata di login.
	\item L’utente può scegliere una delle seguenti modalità:
	\begin{enumerate}
		\item \textbf{Credenziali locali:} inserisce email e password nei campi dedicati e conferma l’accesso.
		\item \textbf{SPID/CIE:} seleziona l’opzione SPID o CIE; il sistema reindirizza al relativo servizio di autenticazione.
	\end{enumerate}
	\item L'utente può decidere di far "ricordare" la propria sessione al sistema, salvando così un cookie di autenticazione in locale.
	\item Se l’autenticazione è valida, l'utente accede al sistema \textbf{[eccezione 1 - 2]}.
\end{enumerate}

\textbf{Estensioni:}
\begin{itemize}
	\item[(0)] Se l'utente possiede un cookie di autenticazione nella propria macchina, l'utente accede senza dover reinserire le credenziali.  \textbf{[eccezione 3]}
\end{itemize}

\textbf{Eccezioni:}
\begin{itemize}
	\item[(1)] Se email o password non sono valide, il sistema mostra un messaggio di errore e richiede di reinserire i dati.
	\item[(2)] Se l’autenticazione SPID/CIE fallisce, viene annullata o scade, il sistema mostra un messaggio di errore e ritorna alla schermata di login.
	\item[(3)] Se il sistema non riconosce il cookie di autenticazione, o il cookie è invalidato, allora l'utente è reindirizzato alla schermata di login.
\end{itemize}

\subsection{Gestione profilo e dati personali \ref{rf:gestioneprofiloedati}}

\textbf{Riassunto:}
Questo use case descrive come un utente autenticato gestisce le proprie credenziali di accesso e preferenze personali tramite l’area profilo.

\textbf{Precondizioni:}
L’utente è autenticato nella piattaforma.

\textbf{Postcondizioni:}
Le modifiche alle credenziali o alle preferenze vengono salvate e risultano effettive per le successive sessioni di accesso.

\textbf{Descrizione:}
\begin{enumerate}
	\item L’utente accede alla sezione “Profilo”.
	\item Il sistema mostra i dati attuali: email, username, password mascherata e stato della preferenza notifiche.
	\item L’utente può:
	\begin{enumerate}
		\item modificare l’\textbf{email}, il sistema invia un messaggio di verifica al vecchio e al nuovo indirizzo email \textbf{[eccezione 1]};
		\item modificare l’\textbf{username};
		\item modificare la \textbf{password};
		\item aggiornare la preferenza di \textbf{ricezione notifiche}.
	\end{enumerate}
	\item Se le modifiche sono valide il sistema le salva. \textbf{[eccezione 2]}, \textbf{[eccezione 3]}.
\end{enumerate}

\textbf{Eccezioni:}
\begin{itemize}
		\item[(1)] Se da parte dei due indirizzi email coinvolti il sistema non riceve da entrambi conferma della modifica, il processo di cambio email viene annullato.
	\item[(2)] Se l’email inserita non è valida o già associata a un altro account, il sistema notifica l’errore e annulla la modifica.
		\item[(3)] Se il nome utente inserito non è valido o già associato a un altro account, il sistema notifica l’errore e annulla la modifica.
\end{itemize}

\subsection{Visualizzazione proposte e sondaggi \ref{rf:ordinamentoproposte}, \ref{rf:filtroproposte}}

\textbf{Riassunto:}
Questo use case descrive come un utente autenticato consulta le proposte pubblicate e i sondaggi disponibili, applicando criteri di ordinamento e filtraggio.

\textbf{Precondizioni:}
L’utente è autenticato nella piattaforma e sono presenti proposte o sondaggi pubblicati.

\textbf{Postcondizioni:}
Le informazioni visualizzate sono aggiornate in base ai criteri di ordinamento o filtro selezionati.

\textbf{Descrizione:}
\begin{enumerate}
	\item L’utente accede alla sezione “Proposte e Sondaggi”.
	\item Il sistema mostra:
	\begin{enumerate}
		\item le \textbf{proposte} pubblicate, con titolo, autore, stato e categoria;
		\item i \textbf{sondaggi} attivi e chiusi con data di apertura e chiusura.
	\end{enumerate}
	\item L’utente può:
	\begin{enumerate}
		\item ordinare le proposte per rilevanza, numero di voti, data di pubblicazione o categoria;
		\item filtrare i risultati per categoria o stato della proposta;
		\item accedere al dettaglio di una proposta o di un sondaggio selezionato.
	\end{enumerate}
	\item Il sistema aggiorna dinamicamente l’elenco in base ai criteri scelti  \textbf{[eccezione 1]}.
\end{enumerate}

\textbf{Eccezioni:}
\begin{itemize}
	\item[(1)] Se non sono presenti proposte o sondaggi attivi, il sistema mostra un messaggio informativo.
\end{itemize}

\subsection{Cambio di lingua \ref{rf:cambiolingua}}

\textbf{Riassunto:}
Questo use case descrive come un utente autenticato modifica la lingua dell’interfaccia della piattaforma.

\textbf{Precondizioni:}
L’utente è autenticato nella piattaforma.

\textbf{Postcondizioni:}
La lingua selezionata diventa attiva immediatamente e viene salvata nel profilo utente.

\textbf{Descrizione:}
\begin{enumerate}
	\item L’utente accede all’area “Impostazioni lingua” dal menu del profilo.
	\item Il sistema mostra l’elenco delle lingue disponibili.
	\item L’utente seleziona la lingua desiderata.
	\item Il sistema aggiorna l’interfaccia nella lingua scelta e registra la preferenza nel profilo utente.
\end{enumerate}

% ====================================================
% UC3
% ====================================================

\section{Amministratore}
\localtableofcontentswithrelativedepth{+1}

\begin{enumerate}
	\item Registrazione di moderatori e associazioni (\ref{rf:registrazionemodass})
	\item Pubblicazione sondaggi (\ref{rf:sondaggi})
	\item Visualizzare policy simulation (\ref{rf:visualizzazionesimulazione})
	\item Download report sull’attività del sistema (\ref{rf:report})
	\item Gestione avanzamento stato proposta (\ref{rf:modificastatoproposta})
\end{enumerate}

\begin{center}
	\includegraphics[scale=0.53]{img/usecase/Amministratore.png}
\end{center}

\subsection{Creare account moderatore o associazione/comitato \ref{rf:registrazionemodass}}
\textbf{Riassunto:}
Questo caso d’uso descrive la procedura di creazione di un nuovo account di tipo moderatore o associazione/comitato tramite l’interfaccia amministrativa.

\textbf{Precondizioni:}
L’amministratore è autenticato e ha i permessi di gestione utenti.

\textbf{Postcondizioni:}
L’account risulta creato, registrato e notificato via email.

\textbf{Descrizione:}
\begin{enumerate}
  \item L’amministratore accede alla sezione di gestione utenti.
  \item Seleziona il tipo di account da creare (moderatore o associazione/comitato).
  \item Inserisce i dati richiesti e conferma l’operazione.
  \item Il sistema verifica la correttezza e completezza dei dati inseriti.
  \item In caso di esito positivo, il sistema crea il nuovo account.
  \item Il sistema invia una comunicazione email all’utente interessato.
\end{enumerate}

\textbf{Eccezioni:}
\begin{itemize}
  \item [E1.] Dati mancanti o non validi: il sistema mostra un messaggio d’errore e richiede la correzione.
  \item [E2.] Indirizzo email già associato a un altro account: il sistema notifica il conflitto e annulla la creazione.
  \item [E3.] Errore nell’invio dell’email: il sistema registra l’errore e notifica l’amministratore.
\end{itemize}

\subsection{Gestire sondaggi \ref{rf:sondaggi}}
\textbf{Riassunto:}
Questo caso d’uso descrive la pubblicazione di un sondaggio rivolto ai cittadini.

\textbf{Precondizioni:}
L’amministratore è autenticato e dispone dei permessi di gestione sondaggi.

\textbf{Postcondizioni:}
Il sondaggio viene pubblicato.

\textbf{Descrizione:}
\begin{enumerate}
  \item L’amministratore accede al modulo di gestione sondaggi.
  \item Seleziona “Crea nuovo sondaggio”.
  \item Compila titolo, descrizione, categoria, periodo e quesiti.
  \item Conferma la pubblicazione.
  \item Il sistema invia notifiche agli utenti che hanno attivato le notifiche.
\end{enumerate}

\textbf{Eccezioni:}
\begin{itemize}
  \item [E1.] Campi obbligatori mancanti: il sistema richiede di completare i dati.
  \item [E2.] Periodo di validità non coerente (data fine antecedente all’inizio): il sistema segnala l’errore.
  \item [E3.] Errore durante l’invio delle notifiche: il sistema completa la pubblicazione ma registra l’errore di comunicazione.
\end{itemize}

\subsection{Consultare e scaricare report\ref{rf:dashboard}}
\textbf{Riassunto:}
Questo caso d’uso descrive l’utilizzo della dashboard amministrativa per consultare, filtrare ed esportare i dati aggregati del sistema.

\textbf{Precondizioni:}
L’amministratore è autenticato.

\textbf{Postcondizioni:}
I dati richiesti sono visualizzati o esportati.

\textbf{Descrizione:}
\begin{enumerate}
  \item L’amministratore accede alla dashboard amministrativa.
  \item Visualizza indicatori generali (utenti, voti, attività, categorie).
  \item Applica filtri e ordinamenti sui dati visualizzati.
  \item Esporta dati in un formato disponibile (es. CSV).
\end{enumerate}

\textbf{Eccezioni:}
\begin{itemize}
  \item [E1.] Nessun dato disponibile per i criteri selezionati: il sistema mostra un messaggio “Nessun risultato trovato”.
  \item [E2.] Errore di esportazione (ad es. permessi insufficienti o formato non supportato): il sistema notifica l’errore e annulla l’operazione.
  \item [E3.] Errore nel caricamento dei dati: il sistema mostra un messaggio di errore e consente di riprovare.
\end{itemize}

\subsection{Approvare/rifiutare una proposta}

\textbf{Riassunto:}
Questo caso d’uso descrive come l’amministratore esamina una proposta in stato \textit{in valutazione} e decide se approvarla oppure rifiutarla.

\textbf{Precondizioni:}
La proposta è in stato \textit{in valutazione}.

\textbf{Postcondizioni:}
La proposta risulta \textit{accettata} oppure \textit{rifiutata}.

\textbf{Descrizione:}
\begin{enumerate}
  \item L’amministratore accede all’elenco delle proposte in valutazione.
  \item Seleziona una proposta.
  \item Analizza contenuto, storia modifiche, contributi e voti.
  \item Decide se approvare la proposta oppure rifiutarla.
  \item Inserisce una motivazione.
  \item Il sistema aggiorna lo stato e registra l’esito della valutazione.
\end{enumerate}

\textbf{Eccezioni:}
\begin{enumerate}
  \item La proposta è stata modificata durante la valutazione: il sistema richiede conferma dell’azione.
  \item La proposta risulta già valutata da un altro amministratore: il sistema impedisce l’operazione.
\end{enumerate}

\subsection{Pubblicare o aggiornare un report amministrativo}

\textbf{Riassunto:}
Questo caso d’uso descrive come l’amministratore redige o aggiorna un report relativo a una proposta, eventualmente modificandone lo stato.

\textbf{Precondizioni:}
La proposta richiede un report o un suo aggiornamento.

\textbf{Postcondizioni:}
Il report risulta pubblicato o aggiornato.
Lo stato della proposta può essere modificato.

\textbf{Descrizione:}
\begin{enumerate}
  \item L’amministratore accede al modulo di report della proposta.
  \item Scrive o modifica testo, allegati, indicatori.
  \item Pubblica il report.
  \item Il sistema aggiorna lo stato della proposta (se previsto).
\end{enumerate}

\textbf{Eccezioni:}
\begin{enumerate}
  \item Il report è in modifica da parte di un altro amministratore: l’operazione viene bloccata.
\end{enumerate}

\section{Cittadino}
\localtableofcontentswithrelativedepth{+1}

\begin{center}
  \includegraphics[scale=0.55]{img/usecase/Cittadino.png}
\end{center}

\subsection{Eseguire una simulazione di policy} \ref{rf:policysimulator}

\textbf{Riassunto:}
Questo use case descrive come un amministratore configura ed esegue un modello di simulazione sui dati relativi alle proposte.

\textbf{Precondizioni}
Dataset e parametri sono disponibili nel sistema.

\textbf{Postcondizioni}
Gli indicatori di simulazione vengono generati.

\textbf{Descrizione}
\begin{enumerate}
  \item L’amministratore apre il modulo “Policy Simulator”.
  \item Configura parametri, coefficienti e modello da utilizzare.
  \item Avvia la simulazione.
  \item Il sistema calcola gli indicatori di simulazione.
\end{enumerate}

\textbf{Eccezioni:}
\begin{enumerate}
  \item I parametri sono incompleti o non validi, viene richiesta una correzione.
  \item Dataset mancante: la simulazione non è avviabile.
\end{enumerate}

\subsection{Visualizzare il risultato di una simulazione}

\textbf{Riassunto:}
Questo use case descrive come un amministratore consulta il risultato di una simulazione tramite grafici e tabelle.

\textbf{Precondizioni:}
È stata eseguita almeno una simulazione.

\textbf{Postcondizioni:}
I risultati della simulazione sono visualizzati.

\textbf{Descrizione:}
\begin{enumerate}
  \item L’amministratore accede alla sezione “Risultati simulazione”.
  \item Visualizza indicatori numerici, grafici e confronti.
  \item Può esportare i dati.
\end{enumerate}

\textbf{Eccezioni:}
\begin{enumerate}
  \item Nessuna simulazione disponibile: viene mostrato messaggio di avviso.
\end{enumerate}

\subsection{Creazione proposta \ref{rf:creazioneproposta}

\textbf{Riassunto:} Questo use case descrive come un cittadino autenticato crea una nuova proposta.

\begin{description}
  \item[Precondizioni:]~
  \begin{itemize}
    \item L’utente è autenticato sulla piattaforma con ruolo ``Cittadino''.
    \item L’utente ha accesso alla sezione per la creazione di nuove proposte.
  \end{itemize}

  \item[Postcondizioni:]~
  \begin{itemize}
    \item Una nuova proposta è pubblicata o creata in stato di ``bozza'', associata all’utente come autore.
    \item Tutti i dati inseriti e i metadati obbligatori (autore, data/ora di creazione, categoria, campi categoria) sono salvati nel sistema.
  \end{itemize}

  \item[Descrizione:]~
  \begin{enumerate}
    \item Il cittadino accede alla sezione ``Crea proposta''.
    \item Il sistema mostra il modulo di creazione con i campi previsti per la categoria selezionata:
    \begin{itemize}
      \item Titolo
      \item Descrizione
      \item Luogo (selezionabile tramite indirizzo o mappa)
      \item Categoria
    \end{itemize}
    \item Il cittadino compila i campi obbligatori e, se disponibili, eventuali campi facoltativi.
    \item Il cittadino sceglie se pubblicare la proposta o salvarla come bozza.
  \end{enumerate}
    Se decide di pubblicarla:
    \begin{enumerate}
      \item Il sistema valida i dati inseriti [eccezione 1].
      \item Se la validazione ha esito positivo, il sistema:
      \begin{itemize}
        \item associa la proposta all’utente come autore;
        \item registra data/ora di creazione e la categoria selezionata;
        \item conferma l’avvenuta pubblicazione all’utente [eccezione 2].
      \end{itemize}
    \end{enumerate}
     Se invece la salva come bozza, alla sua selezione nella lista delle proposte verrà riportato al modulo di creazione, dove avrà la possibilità di modificarla e/o pubblicarla.

  \item[Eccezioni:]~
  \begin{enumerate}
    \item Se uno o più campi sono mancanti o non validi, il sistema segnala gli errori e richiede la correzione dei dati.
    \item In caso di errore di sistema (es. problemi di salvataggio sul database), il sistema notifica un messaggio di errore e la proposta non viene creata o viene richiesto un nuovo tentativo.
  \end{enumerate}
\end{description}



\subsection{Pubblicazione proposta \ref{rf:creazioneproposta}}

\subsection{Eliminazione proposta \ref{rf: ?}}

\subsection{Modifiche ad una proposta \ref{rf:modificacollaborativa}}

Rifare non con chat, piu slim

\subsection{Espressione del voto \ref{rf:voto}}


\subsection{Gestione dei preferiti\ref{rf:preferiti}}

\textbf{Riassunto:} Questo use case descrive come un cittadino gestisce l’elenco dei propri elementi preferiti (proposte e sondaggi).

\begin{description}
  \item[Precondizioni:]~
  \begin{itemize}
    \item L’utente è autenticato sulla piattaforma con ruolo ``Cittadino''.
    \item Sono presenti proposte o sondaggi visualizzabili.
  \end{itemize}

  \item[Postcondizioni:]~
  \begin{itemize}
    \item L’elenco dei preferiti del cittadino è aggiornato in base alle azioni di aggiunta/rimozione.
    \item Le eventuali notifiche collegate agli elementi preferiti sono aggiornate (\ref{rf:notifiche}).
  \end{itemize}

  \item[Descrizione:]~
  \begin{enumerate}
    \item Il cittadino accede alla lista delle proposte e/o dei sondaggi, oppure al dettaglio di un singolo elemento.
    \item Per ogni proposta/sondaggio il sistema mostra lo stato di preferito con l'icona di una stella che, se colorata, indica l'aggiunta ai preferiti.
    \item L'utente può cliccare sulla stella per modificarne lo stato.
    \item Il sistema aggiorna l’elenco dei preferiti dell’utente e conferma l’operazione [eccezione 1].
    \item Il cittadino può accedere alla sezione ``Preferiti'' per visualizzare l’elenco aggiornato delle proposte e dei sondaggi contrassegnati.
  \end{enumerate}

  \item[Eccezioni:]~
  \begin{enumerate}
    \item Se l’elemento selezionato non è più disponibile (es. rimosso, oscurato, archiviato), il sistema informa l’utente che l’operazione non può essere completata e, se necessario, lo rimuove automaticamente dall’elenco dei preferiti.
    \item In caso di errore tecnico nella registrazione dell’operazione, il sistema segnala l’errore e l’elenco dei preferiti rimane invariato.
  \end{enumerate}
\end{description}



\textbf{Riassunto:} Questo use case descrive come un cittadino eleggibile esprime un voto positivo o negativo su una proposta.

\begin{description}
  \item[Precondizioni:]~
  \begin{itemize}
    \item L’utente è autenticato sulla piattaforma con ruolo ``Cittadino''.
    \item La proposta su cui si intende votare è in uno stato che consente la votazione (es. pubblicata, in valutazione, non ancora accettata o rifiutata).
  \end{itemize}

  \item[Postcondizioni:]~
  \begin{itemize}
    \item Il voto del cittadino è registrato in forma anonima.
    \item Il conteggio dei voti e/o il punteggio della proposta è aggiornato secondo le regole di calcolo previste.
    \item L’utente può, se consentito, modificare il proprio voto fino alla chiusura della votazione o al cambio di stato della proposta.
  \end{itemize}

  \item[Descrizione:]~
  \begin{enumerate}
    \item Il cittadino accede al dettaglio di una proposta su cui è possibile votare.
    \item Il sistema mostra lo stato corrente della proposta e le opzioni di voto (es. ``Voto favorevole'', ``Voto contrario''), nonché l’eventuale voto già espresso dall’utente.
    \item Il cittadino seleziona l’opzione di voto desiderata o modifica un voto precedentemente espresso.
    \item Se i controlli hanno esito positivo, il sistema:
    \begin{itemize}
      \item registra il voto in forma anonima;
      \item aggiorna il conteggio totale e, se previsto, gli altri indicatori associati;
      \item conferma l’avvenuta registrazione del voto all’utente [eccezione 1].
    \end{itemize}
  \end{enumerate}

  \item[Eccezioni:]~
  \begin{enumerate}
    \item Se la proposta si trova in uno stato che non consente più la votazione (es. accettata, rifiutata, chiusa), il sistema disabilita l’azione di voto e notifica che la votazione non è più disponibile.
    \item In caso di errore tecnico durante la registrazione del voto, il sistema segnala l’errore e il voto non viene considerato valido fino a nuovo tentativo dell’utente.
  \end{enumerate}
\end{description}

\subsection{Segnalare un contenuto \ref{rf:segnalazione}} 

\textbf{Riassunto:}
Questo caso d’uso descrive come un cittadino segnala un contenuto pubblicato (proposta o modifica) ritenuto inappropriato, inserendolo nella coda di revisione dei moderatori in conformità al requisito \ref{rf:validazionecontenuti}.

\textbf{Precondizioni:}
\begin{itemize}
  \item L’utente è autenticato sulla piattaforma con ruolo ``Cittadino''.
  \item Sta visualizzando un contenuto pubblicato (proposta o modifica).
\end{itemize}

\textbf{Postcondizioni:}
\begin{itemize}
  \item È stata registrata una segnalazione associata al contenuto e al cittadino.
  \item Il contenuto risulta presente nella coda di revisione dei moderatori.
\end{itemize}

\textbf{Descrizione:}
\begin{enumerate}
  \item Il cittadino visualizza una proposta o una modifica pubblicata che ritiene inappropriata.
  \item Il cittadino seleziona il comando \textit{``Segnala contenuto''} associato al contenuto.
  \item Il sistema mostra un modulo di segnalazione che include:
  \begin{enumerate}
    \item identificativo e anteprima del contenuto;
    \item un elenco di motivazioni predefinite, ad esempio:
    \begin{enumerate}
      \item incitamento all’odio;
      \item contenuti offensivi o volgari;
      \item informazioni false;
      \item spam;
      \item contenuto duplicato;
      \item violazione della privacy;
      \item altro.
    \end{enumerate}
    \item un campo di testo libero per una descrizione opzionale del problema.
  \end{enumerate}
  \item Il cittadino seleziona una motivazione dall’elenco.
  \item \textit{Facoltativamente}, il cittadino inserisce una descrizione aggiuntiva nel campo di testo.
  \item Il cittadino conferma l’invio della segnalazione.
  \item Il sistema verifica che sia stata selezionata almeno una motivazione e che i campi obbligatori siano compilati correttamente [eccezione 1].
  \item Il sistema verifica che il contenuto sia ancora disponibile e non sia già stato rimosso [eccezione 2].
  \item In caso di esito positivo dei controlli, il sistema:
  \begin{enumerate}
    \item registra la segnalazione associandola al contenuto e al cittadino;
    \item inserisce il contenuto nella coda di revisione dei moderatori;
    \item mostra un messaggio di conferma al cittadino sull’avvenuta registrazione della segnalazione.
  \end{enumerate}
\end{enumerate}

\textbf{Eccezioni:}
\begin{enumerate}
  \item Campi obbligatori mancanti o non validi (ad esempio nessuna motivazione selezionata): il sistema evidenzia i campi da correggere e non conferma l’invio della segnalazione.
  \item Contenuto non più disponibile (ad esempio rimosso da un moderatore o dall’autore): il sistema annulla la procedura e informa il cittadino che il contenuto non è più accessibile.
\end{enumerate}

\section{Associazione/Comitato}

\subsection{Pubblicazione proposte collettive (\ref{rf:propostecollettive})}
\subsection{Endorsement (\ref{rf:endorsement})}

\section{Moderatore}
\localtableofcontentswithrelativedepth{+1}

\begin{center}
	\includegraphics[scale=0.9]{img/usecase/Moderatore.png}
\end{center}

\section{Moderatore}
\localtableofcontentswithrelativedepth{+1}

\subsection{Moderazione tramite coda di revisione} \label{UC:modcoda}

\textbf{Riassunto:}  
Questo use case descrive come il moderatore analizza contenuti segnalati dai cittadini o rilevati automaticamente dal sistema e decide l’esito della revisione.

\textbf{Precondizioni:}  
Esistono contenuti presenti nella coda di revisione.

\textbf{Postcondizioni:}  
Il contenuto risulta confermato oppure rimosso.  
Eventuali limitazioni all’autore possono essere applicate (\textit{estensione} \ref{UC:modlimitazioni}).

\textbf{Descrizione:}
\begin{enumerate}
  \item Il moderatore accede alla coda di revisione.
  \item Seleziona un contenuto segnalato o rilevato automaticamente.
  \item Analizza il testo e i metadati (autore, data, motivo della segnalazione).
  \item Valuta la conformità rispetto alle policy della piattaforma e decide tra:
  \begin{itemize}
    \item confermare il contenuto;
    \item rimuovere il contenuto (include \ref{UC:rimozione});
    \item applicare una limitazione all’autore (estende \ref{UC:modlimitazioni}).
  \end{itemize}
  \item Il sistema aggiorna lo stato della segnalazione e registra l’esito.
\end{enumerate}

\textbf{Eccezioni:}
\begin{itemize}
  \item[E1.] L’elemento è stato già revisionato da un altro moderatore: il sistema aggiorna la lista e notifica la situazione.
  \item[E2.] Il contenuto non è più disponibile (eliminato dall’autore): il sistema archivia automaticamente la segnalazione.
\end{itemize}


\subsection{Moderazione spontanea dei contenuti} \label{UC:modspontanea}

\textbf{Riassunto:}  
Questo use case descrive come il moderatore interviene autonomamente su contenuti pubblicati senza passare dalla coda di revisione.

\textbf{Precondizioni:}  
Il moderatore è autenticato e ha accesso ai contenuti pubblici.

\textbf{Postcondizioni:}  
Il contenuto risulta confermato o rimosso.  
Eventuali limitazioni possono essere applicate (\textit{estensione} \ref{UC:modlimitazioni}).

\textbf{Descrizione:}
\begin{enumerate}
  \item Il moderatore accede alla sezione di ricerca dei contenuti pubblicati.
  \item Filtra o individua un contenuto da analizzare.
  \item Visualizza testo, allegati, metadati e contesto.
  \item Valuta la conformità rispetto alle policy della piattaforma e decide tra:
  \begin{itemize}
    \item non intervenire;
    \item rimuovere il contenuto (include \ref{UC:rimozione});
    \item applicare una limitazione (estende \ref{UC:mod-limitazioni}).
  \end{itemize}
  \item Il sistema registra l’azione eventualmente intrapresa.
\end{enumerate}

\textbf{Eccezioni:}
\begin{itemize}
  \item[E1.] Il contenuto è stato rimosso dall’autore: il sistema notifica l’incongruenza.
  \item[E2.] Un altro moderatore sta intervenendo sul contenuto: il sistema blocca l’operazione.
\end{itemize}



\subsection{Rimozione di un contenuto} \label{UC:rimozione}

\textbf{Riassunto:}  
Questo use case descrive come il moderatore elimina un contenuto che viola le policy.

\textbf{Descrizione:}
\begin{enumerate}
  \item Il moderatore seleziona l’opzione di rimozione.
  \item Inserisce la motivazione dell’intervento.
  \item Il sistema elimina il contenuto dalla piattaforma.
  \item Il sistema registra: motivazione, timestamp e identificativo del moderatore.
  \item Il sistema offre la possibilità di applicare una limitazione all’autore (estensione \ref{UC:modlimitazioni}).
\end{enumerate}



\subsection{Applicazione limitazioni all’autore} \label{UC:modlimitazioni}

\textbf{Attore principale:} Moderatore  
\textbf{Attore secondario:} Amministrazione

\textbf{Riassunto:}  
Questo use case descrive come il moderatore gestisce eventuali limitazioni all’autore del contenuto, secondo due flussi distinti per cittadini e associazioni.

---

\subsubsection*{Caso A — Autore cittadino: applicazione diretta}

\textbf{Descrizione:}
\begin{enumerate}
  \item Il moderatore seleziona tipo e durata della limitazione.
  \item Il sistema applica la limitazione all’account del cittadino.
  \item Il sistema invia automaticamente all’utente un’email con:
  \begin{itemize}
    \item durata,
    \item riferimenti alle norme violate,
    \item modalità di ricorso.
  \end{itemize}
  \item L’operazione viene registrata.
\end{enumerate}

---

\subsubsection*{Caso B — Autore associazione/comitato: escalation verso amministrazione}

\textbf{Descrizione:}
\begin{enumerate}
  \item Il moderatore seleziona “Proponi limitazione”.
  \item Il sistema invia la richiesta all’amministrazione.
  \item Il sistema applica un blocco temporaneo delle interazioni fino alla decisione.
  \item Il sistema invia un’email all’associazione, indicando:
  \begin{itemize}
    \item la contestazione,
    \item che la decisione finale è in carico all’amministrazione.
  \end{itemize}
  \item L’amministrazione valuterà il provvedimento secondo il processo definito nel Requito Funzionale \ref{rf:modoerazioneprocamm}.
\end{enumerate}