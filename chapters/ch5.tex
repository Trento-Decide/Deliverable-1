% !TeX spellcheck = it_IT
\chapter{User Stories}

\renewcommand{\thesection}{US\arabic{section}}

\newcommand{\userstory}[3]{
  #1

  \begin{description}
    \item[Criteri di accettazione:]~
    \begin{itemize}
      #2
    \end{itemize}

    \item[Tasks:]~
    \begin{itemize}
      #3
    \end{itemize}
  \end{description}
}

\section{Registrazione del cittadino \ref{rf:registrazionecittadini}}
\userstory
{
  Come utente anonimo, voglio registrarmi alla piattaforma verificando la mia identità tramite SPID o CIE e configurando le mie credenziali locali.
}
{
  \item Il sistema deve richiedere l'autenticazione iniziale tramite SPID o CIE.
  \item Il sistema deve verificare automaticamente la residenza nel Comune di Trento tramite ANPR; se l'utente non è residente, la registrazione deve essere bloccata.
  \item L'utente deve poter inserire un indirizzo email, una password e un nome utente per i futuri accessi.
  \item L'utente deve poter selezionare la preferenza per la ricezione delle notifiche email (abilita/disabilita).
  \item Il sistema deve inviare una email di verifica contenente un link per l'attivazione dell'account.
  \item L'attivazione dell'account deve essere completata entro 10 minuti dalla ricezione dell'email, pena l'invalidazione della richiesta.
}
{
  \item Integrazione dei servizi di autenticazione SPID e CIE.
  \item Implementazione del connettore ANPR per la verifica della residenza.
  \item Sviluppo del modulo frontend per l'inserimento di email, password, username e preferenze notifiche.
  \item Implementazione controlli di unicità per email e username nel database.
  \item Configurazione del servizio SMTP per l'invio delle email transazionali.
  \item Implementazione della logica di scadenza del token di verifica (10 minuti).
}

\section{Espressione del voto \ref{rf:voto}} \label{us:voto}
\userstory
{
  Come cittadino, voglio esprimere un voto (favorevole o contrario) sulle proposte e partecipare ai sondaggi attivi, per contribuire alla definizione delle priorità comunali.
}
{
  \item Il sistema deve consentire di esprimere un voto positivo o negativo sulle proposte in stato "pubblicata" o "in valutazione".
  \item Il sistema deve consentire di rispondere ai quesiti presenti nei sondaggi attivi.
  \item Ogni utente può esprimere un solo voto per ciascuna proposta o per ciascun quesito dei sondaggi.
  \item Il voto deve poter essere modificato dall'utente fino alla chiusura del sondaggio o alla definizione dell'esito della proposta.
  \item Il sistema deve registrare il voto in forma anonima, rendendo impossibile risalire all'identità del votante tramite i report pubblici.
  \item Il sistema deve aggiornare il conteggio dei voti e visualizzare l'eventuale feedback all'utente.
}
{
  \item Implementazione componenti UI per la votazione (bottoni favorevole/contrario e opzioni sondaggio).
  \item Sviluppo API backend per la ricezione e validazione del voto (controllo stato proposta).
  \item Progettazione dello schema database per garantire l'unicità del voto per utente mantenendo l'anonimato (separazione identità/preferenza).
  \item Implementazione logica per la modifica del voto.
  \item Aggiornamento real-time dei contatori visibili sulla proposta.
}

\section{Gestione dei preferiti \ref{rf:preferiti}} \label{us:preferiti}
\userstory
{
  Come cittadino, voglio aggiungere o rimuovere proposte e sondaggi dai miei preferiti, per poter accedere rapidamente ai contenuti di mio interesse e ricevere notifiche automatiche sui loro aggiornamenti di stato.
}
{
  \item Il sistema deve consentire di aggiungere o rimuovere una proposta o un sondaggio dai preferiti agendo su un apposito indicatore visivo.
  \item L'azione di aggiunta/rimozione deve essere disponibile sia dalla lista generale delle proposte/sondaggi che dalla pagina di dettaglio del singolo elemento.
  \item Il sistema deve fornire una sezione dedicata "Preferiti" dove l'utente può consultare l'elenco aggiornato di tutti gli elementi salvati.
  \item Se un elemento nei preferiti viene rimosso o archiviato dal sistema, esso deve essere automaticamente rimosso dalla lista dei preferiti dell'utente, eventualmente con una notifica informativa.
  \item L'aggiornamento dello stato "preferito" deve essere immediato e persistente tra le sessioni.
}
{
  \item Implementazione del componente UI (attivo/disattivo) sulle card e nelle pagine di dettaglio.
  \item Sviluppo API backend per il toggle dello stato di preferito (add/remove).
  \item Sviluppo della pagina frontend "I miei Preferiti".
  \item Implementazione della logica di pulizia dei preferiti in caso di cancellazione della proposta originale.
}

\section{Segnalazione contenuti \ref{rf:segnalazione}} \label{us:segnalazione}
\userstory
{
  Come \textbf{cittadino}, voglio poter segnalare i contenuti pubblicati (proposte o modifiche) ritenuti inappropriati specificandone la motivazione, affinché i moderatori possano verificarli e garantire un ambiente di discussione sicuro e rispettoso.
}
{
  \item Il sistema deve mostrare un comando "Segnala contenuto" su ogni proposta o modifica pubblicata.
  \item Il sistema deve presentare un modulo che richiede obbligatoriamente la selezione di una motivazione da un elenco predefinito (es. incitamento all'odio, informazioni false, spam, violazione privacy).
  \item L'utente deve poter inserire facoltativamente un testo descrittivo aggiuntivo per chiarire il motivo della segnalazione.
  \item Il sistema deve verificare che il contenuto segnalato sia ancora disponibile prima di registrare l'operazione.
  \item Una volta confermata, la segnalazione deve essere inserita automaticamente nella coda di revisione dei moderatori.
  \item L'utente deve ricevere un feedback visivo di conferma dell'avvenuto invio della segnalazione.
}
{
  \item Implementazione del bottone "Segnala" nell'interfaccia delle proposte.
  \item Creazione della modale con form contenente la lista predefinita di motivazioni (enum) e campo testo libero.
  \item Sviluppo API backend per la creazione dell'oggetto "Report" collegato a User e Content.
  \item Implementazione della logica di inserimento nella coda di revisione.
  \item Gestione degli errori nel caso in cui il contenuto sia stato rimosso concorrentemente.
}

\section{Espressione Endorsement \ref{rf:endorsement}}
\userstory
{
  Come rappresentante di un'associazione, voglio poter esprimere o rimuovere un endorsement ufficiale sulle proposte pubblicate, per manifestare pubblicamente il sostegno della mia organizzazione accrescendone la rilevanza agli occhi della cittadinanza.
}
{
  \item Il sistema deve consentire alle sole utenze di tipo "Associazione" di esprimere un endorsement sulle proposte in stato "pubblicata".
  \item Ogni associazione può esprimere un solo endorsement per ciascuna proposta.
  \item L'azione di endorsement deve generare visibilmente un'etichetta sulla proposta riportante il nome dell'associazione.
  \item L'associazione deve poter rimuovere un endorsement precedentemente assegnato; la rimozione deve comportare la cancellazione immediata dell'etichetta visuale.
}
{
  \item Implementazione verifica permessi (solo ruolo Associazione).
  \item Sviluppo API backend per gestione endorsement.
  \item Aggiornamento modello dati Proposta per collegare la lista degli endorser.
  \item Sviluppo componente UI "Etichetta Endorsement" nella visualizzazione proposta.
  \item Implementazione pulsante toggle "Esprimi/Rimuovi Endorsement".
}
