% !TeX spellcheck = it_IT
\chapter{Il Progetto Trento Decide}

\section{Introduzione}

Il progetto si fonda sull’assunto che il Comune di Trento, pur disponendo di competenze e strumenti amministrativi avanzati, non possa individuare e affrontare con assoluta completezza l’insieme delle problematiche presenti sul territorio comunale.
Tale limite non deriva da inefficienze, ma è una conseguenza fisiologica della complessità urbana e della distanza che spesso si crea tra amministrazione e cittadinanza.

\emph{Trento Decide} nasce con l’obiettivo di ridurre tale divario, fungendo da ponte diretto tra cittadini e istituzioni.

La piattaforma rende il processo decisionale più inclusivo, trasparente e reattivo, offrendo uno spazio digitale in cui proporre, discutere e votare iniziative di interesse pubblico. In questo senso, \emph{Trento Decide} si qualifica come un’iniziativa di e-democracy, applicando i principi della civic technology per modernizzare i processi partecipativi e promuovere una governance più aperta e collaborativa.

I cittadini possono pubblicare nuove iniziative, modificarle o proporre revisioni in collaborazione con altri partecipanti.
Inoltre, possono esprimere il proprio voto sulle proposte esistenti, contribuendo così a determinarne la priorità e la rilevanza all’interno del processo decisionale civico.

Le proposte vengono periodicamente analizzate dall’amministrazione comunale, che ne verifica la fattibilità tecnico-scientifica.

L’intero processo di valutazione è tracciato nel sistema e reso trasparente alla cittadinanza attraverso la pubblicazione delle principali fasi di analisi.
Al termine, l’amministrazione comunica l’esito — accettazione con avvio dell’attuazione o rifiuto motivato — visibile direttamente nella pagina della proposta.

Il sistema è implementato come applicazione web accessibile via browser, senza necessità di installazioni.
È concepito per favorire un’interazione continua e verificabile tra cittadini e amministrazione, garantendo trasparenza, chiarezza e sicurezza lungo l’intero ciclo di vita delle proposte.

\section{Definizioni}

Le seguenti definizioni stabiliscono la terminologia di riferimento adottata nel documento, assicurando coerenza e chiarezza tra le sezioni funzionali e tecniche.

\begin{description}

\item[Definizione \refstepcounter{definition}\thedefinition\label{def:categoria} – Categoria:]
Ambito tematico utilizzato per classificare le proposte in fase di creazione e pubblicazione.
Le categorie rappresentano macro-aree di interesse pubblico  e consentono di indirizzare la proposta verso gli uffici competenti.
Le categorie previste sono:

\begin{itemize}
	\item \textbf{Urbanistica} — interventi sulla pianificazione e trasformazione urbana (campi tipici: tipologia intervento, destinazione d’uso).
	\item \textbf{Ambiente} — proposte a tutela dell’ambiente e della sostenibilità (campi tipici: ambito ambientale, impatto previsto).
	\item \textbf{Sicurezza} — segnalazioni o interventi su rischi e criticità pubbliche (campi tipici: tipo di criticità, priorità).
	\item \textbf{Cultura} — iniziative culturali e artistiche (campi tipici: ambito culturale, target).
	\item \textbf{Istruzione} — interventi per scuole e servizi formativi (campi tipici: livello educativo, benefici).
	\item \textbf{Innovazione e Digitale} — progetti tecnologici o di digitalizzazione (campi tipici: tecnologia proposta, processo coinvolto).
	\item \textbf{Sociale e Inclusione} — iniziative rivolte al benessere e all’inclusione sociale (campi tipici: destinatari, problema affrontato).
	\item \textbf{Mobilità e Trasporti} — proposte sul sistema di mobilità urbana (campi tipici: tipologia intervento, impatto sulla mobilità).
	\item \textbf{Welfare} — interventi socio-assistenziali (campi tipici: destinatari, impatto sociale).
	\item \textbf{Sport e Tempo Libero} — attività e infrastrutture sportive o ricreative (campi tipici: tipo attività, struttura coinvolta).
	\item \textbf{Patrimonio e Lavori Pubblici} — manutenzione o interventi su edifici e infrastrutture pubbliche (campi tipici: tipo intervento, urgenza).
\end{itemize}

\item[Definizione \refstepcounter{definition}\thedefinition\label{def:utente} – Utente:]
  Soggetto registrato e autenticato alla piattaforma.
  Comprende tutti i ruoli predefiniti previsti dal sistema: cittadino, moderatore, amministratore comunale e rappresentante di associazione \defref{def:ruoli}.

\item[Definizione \refstepcounter{definition}\thedefinition\label{def:ruoli} – Ruoli utente:]
La piattaforma prevede quattro ruoli distinti, utilizzati per regolare l’accesso alle funzionalità del sistema.

\textbf{Cittadino}:
Persona fisica di età pari o superiore a 16 anni, residente nel Comune di Trento.  
L’identità e l’eleggibilità al voto sono verificate tramite SPID o CIE\@.
La residenza è verificata tramite integrazione con ANPR\@.

\textbf{Associazione}:
Entità giuridica registrata che presenta proposte collettive.
La rappresentanza è certificata tramite atto costitutivo e delega del legale rappresentante, caricati in piattaforma e soggetti a verifica dell’amministrazione comunale.
Le associazioni dispongono di tutte le funzionalità previste per i cittadini, con in aggiunta quelle relative alla creazione delle proposte collettive ed endorsement \defref{def:propostacollettiva}, \defref{def:endorsement}.

\textbf{Moderatore}:
Utente incaricato della verifica di conformità dei contenuti generati dai cittadini e della gestione delle segnalazioni.

\textbf{Amministratore}:
Utente con privilegi completi di gestione della piattaforma.
È responsabile della pubblicazione di report e sondaggi, della validazione delle proposte e dell’aggiornamento dei relativi stati di avanzamento, nonché della configurazione dei parametri operativi del sistema e della supervisione dei processi di moderazione.

\item[Definizione \refstepcounter{definition}\thedefinition\label{def:statoproposta} – Stato di una proposta:]
Ogni proposta è associata a uno stato che ne descrive l’avanzamento nel processo amministrativo.
Gli stati previsti sono: \emph{bozza}, \emph{pubblicata}, \emph{in valutazione}, \emph{accettata}, \emph{rifiutata}, \emph{in attuazione}, \emph{completata}.

\item[Definizione \refstepcounter{definition}\thedefinition\label{def:propostacollettiva} – Proposta collettiva:]
Proposta presentata da un’associazione riconosciuta, distinta dalle proposte individuali in quanto esprime una posizione formalmente attribuibile a un soggetto collettivo e può avere un peso specifico diverso nei processi di valutazione e decisione rispetto alle proposte dei singoli cittadini.

\item[Definizione \refstepcounter{definition}\thedefinition\label{def:endorsement} – Endorsement:]
Azione con cui un’associazione esprime sostegno formale a una proposta pubblicata. Gli endorsement sono visibili pubblicamente e contribuiscono alla percezione di rilevanza dell’iniziativa.

\end{description}

\section{Vantaggi}

\subsection{Per il Comune}

\begin{itemize}
  \item \textbf{Prioritizzazione basata sui dati}: il sistema di voto aiuta a individuare rapidamente le iniziative più rilevanti per la comunità, riducendo il divario tra bisogni territoriali e agenda amministrativa.

  \item \textbf{Flusso operativo tracciabile}: il flusso operativo è tracciabile dalla creazione della proposta all’inoltro all’ufficio competente, garantendo cronologia pubblica e regole chiare, per una maggiore trasparenza e una minore frammentazione procedurale.

  \item \textbf{Canale unico e verificabile}: ogni proposta segue un flusso chiaro con registri pubblici, agevolando audit interni e garantendo trasparenza verso la cittadinanza.

  \item \textbf{Riduzione del carico sugli sportelli}: la centralizzazione delle proposte diminuisce il volume di email, PEC e richieste frammentate agli URP, generando risparmi operativi e di tempo.
\end{itemize}

\subsection{Per i cittadini}

\begin{itemize}
  \item \textbf{Partecipazione effettiva}: la possibilità di pubblicare, votare e proporre modifiche rende la cittadinanza protagonista nella definizione delle priorità pubbliche.

  \item \textbf{Chiarezza sul percorso}: ogni proposta riporta in modo trasparente il proprio stato
  (\emph{bozza}, \emph{pubblicata}, \emph{in valutazione}, \emph{accettata}, \emph{rifiutata}, \emph{in attuazione}, \emph{completata}), garantendo tracciabilità e riducendo l’asimmetria informativa.

  \item \textbf{Accesso semplice e universale}: la piattaforma è immediatamente accessibile via web, non richiede installazioni e garantisce un utilizzo universale da parte di tutti i cittadini.

  \item \textbf{Tracciabilità personale}: ogni cittadino può facilmente seguire le proprie iniziative, modifiche apportate e voti espressi, in un'area riservata e dedicata.
\end{itemize}

\section{Limiti e mitigazioni}

Il progetto \emph{Trento Decide}, pur garantendo un elevato grado di trasparenza e partecipazione, presenta alcuni limiti intrinseci di natura tecnica, organizzativa e sociale.
La seguente sezione identifica i principali rischi e descrive le strategie di mitigazione previste.

\begin{itemize}

  \item \textbf{Definizione dei criteri di priorità}:
  Stabilire parametri oggettivi per la valutazione delle proposte è complesso.

  \item \textbf{Neutralità e bias cognitivi}:
  Le decisioni automatizzate e i comportamenti collettivi possono introdurre distorsioni involontarie.
  Si prevede la revisione annuale degli algoritmi e l’audit di neutralità per garantire imparzialità e trasparenza dei processi.

  \item \textbf{Rischio di manipolazione o brigading}:
  Coordinamenti di gruppo o attività fraudolente possono alterare voti e ranking.
  Il sistema applica limiti di frequenza, verifica d’identità tramite SPID/CIE e rilevamento automatico di anomalie, integrato con la moderazione (\ref{rnf:moderazione}).

  \item \textbf{Digital divide}:
  Una parte della popolazione può trovarsi in difficoltà nell’uso di strumenti digitali.
  Sono previsti punti di facilitazione presso biblioteche e sportelli comunali, supporto telefonico e assistenza tramite delega per l’inserimento delle proposte.

  \item \textbf{Competenza multi-ente}:
  Alcune proposte ricadono su enti diversi dal Comune di Trento (Provincia, Stato, gestori di servizi pubblici).
  Il sistema deve prevedere un flusso chiaro e tracciabile di reindirizzamento e presa in carico verso l’ente competente.

  \item \textbf{Aspettative giuridiche ambigue}:
  L’approvazione o la priorità elevata di una proposta non implicano un diritto automatico alla sua attuazione.
  L’amministrazione pubblica deve fornire motivazioni chiare e documentate per ogni decisione finale.

  \item \textbf{Costi ricorrenti e sostenibilità}:
  Il mantenimento della piattaforma richiede risorse continuative per moderazione, assistenza, comunicazione, sicurezza e aggiornamenti software.
  È raccomandata la previsione di un piano di sostenibilità tecnica ed economica pluriennale.
\end{itemize}

% TODO: trovare soluzione migliore x estetica
\newpage
\section*{Slides Pitch}
Di seguito sono riportate le quattro slide principali del pitch introduttivo, accompagnate da una breve descrizione del contenuto.

\vspace{0.6cm}

% --- SLIDE 1 ---
\noindent
\begin{minipage}[t]{0.48\textwidth} 
    \vspace{0pt} 
    \setlength\fboxsep{0pt}
    \setlength\fboxrule{0.5pt}
    \color{grayline}\fbox{\includegraphics[width=\textwidth]{img/D0/1.png}}
\end{minipage}
\hfill
\begin{minipage}[t]{0.48\textwidth} 
    \vspace{0pt} 
    {\large\textbf{\textcolor{trentogreen}{Slide 1 — Il problema}}}\\[3mm]
    {\small 
        \textbf{Bassa partecipazione attiva}: mancano strumenti utili e motivanti.\\[1mm]
        \textbf{Processi decisionali poco accessibili}: strumenti complessi o poco utilizzati; serve renderli semplici e con impatti visibili.
    }
\end{minipage}

\vspace{0.6cm}
{\color{grayline}\hrule} 
\vspace{0.6cm}

% --- SLIDE 2 ---
\noindent
\begin{minipage}[t]{0.48\textwidth}
    \vspace{0pt}
    \setlength\fboxsep{0pt}
    \setlength\fboxrule{0.5pt}
    \color{grayline}\fbox{\includegraphics[width=\textwidth]{img/D0/2.png}}
\end{minipage}
\hfill
\begin{minipage}[t]{0.48\textwidth}
    \vspace{0pt}
    {\large\textbf{\textcolor{trentogreen}{Slide 2 — La soluzione}}}\\[3mm]
    {\small 
        \textbf{Trento Decide} avvicina cittadini e amministrazione: propone, discute e vota iniziative con tracciabilità completa e feedback chiari.
    }
\end{minipage}

\vspace{0.6cm}
{\color{grayline}\hrule}
\vspace{0.6cm}

% --- SLIDE 3 ---
\noindent
\begin{minipage}[t]{0.48\textwidth}
    \vspace{0pt}
    \setlength\fboxsep{0pt}
    \setlength\fboxrule{0.5pt}
    \color{grayline}\fbox{\includegraphics[width=\textwidth]{img/D0/3.png}}
\end{minipage}
\hfill
\begin{minipage}[t]{0.48\textwidth}
    \vspace{0pt}
    {\large\textbf{\textcolor{trentogreen}{Slide 3 — Vantaggi}}}\\[3mm]
    {\small 
        \textbf{Democrazia diretta}: ogni cittadino partecipa senza filtri.\\[1mm]
        \textbf{Attenzione alla vita pubblica}: più coinvolgimento, più responsabilità.\\[1mm]
        \textbf{Divario di percezione}: l’amministrazione non vede sempre i problemi quotidiani vissuti dai cittadini.
    }
\end{minipage}

\vspace{0.6cm}
{\color{grayline}\hrule}
\vspace{0.6cm}

% --- SLIDE 4 ---
\noindent
\begin{minipage}[t]{0.48\textwidth}
    \vspace{0pt}
    \setlength\fboxsep{0pt}
    \setlength\fboxrule{0.5pt}
    \color{grayline}\fbox{\includegraphics[width=\textwidth]{img/D0/4.png}}
\end{minipage}
\hfill
\begin{minipage}[t]{0.48\textwidth}
    \vspace{0pt}
    {\large\textbf{\textcolor{trentogreen}{Slide 4 — Che dicono i dati?}}}\\[3mm]
    {\small 
        Tasso di istruzione: \textbf{75,3\%} (+9,8\%).\\[1mm]
        Associazioni per abitante: \textbf{1 ogni 83} (+102,8\%).\\[1mm]
        Partecipazione e volontariato: \textbf{18,4\%} (+12,2\%).\\[1mm]
        Relazioni definite rispetto alla media Nazionale.\\[2mm]
        \textit{\textcolor{gray}{Fonti: Trentino Open Data, Istat}}
    }
\end{minipage}
\vspace{0.8cm}