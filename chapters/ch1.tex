\chapter{Il Progetto Trento Decide}

\section{Introduzione}

Il progetto si fonda sull’assunto che il Comune di Trento, pur disponendo di competenze e strumenti amministrativi avanzati, non sia in grado di individuare e affrontare con assoluta precisione e completezza l’insieme delle problematiche presenti sul territorio comunale. Tale limite non deriva da inefficienze specifiche, ma è una conseguenza fisiologica della complessità urbana e della distanza che spesso si crea tra amministrazione e cittadinanza.
Trento Decide nasce proprio con l’obiettivo di ridurre il divario tra l’efficienza amministrativa attuale e un’ideale efficienza assoluta, fungendo da ponte diretto tra cittadini e istituzioni. La piattaforma si propone di rendere il processo decisionale più inclusivo, trasparente e reattivo, offrendo ai cittadini uno spazio digitale in cui proporre, discutere e votare iniziative di pubblico interesse.
Gli utenti, una volta autenticati, possono pubblicare nuove iniziative, corredandole di titolo e descrizione, e modificarle o proporre revisioni nel caso in cui non ne condividano pienamente i contenuti. Inoltre, hanno la possibilità di esprimere il proprio voto sulle proposte esistenti, contribuendo così a determinarne la priorità e la rilevanza all’interno del processo decisionale.
Quando una proposta supera la soglia minima di voti prevista (def. 1.2), essa viene automaticamente segnalata ai tecnici comunali competenti, che hanno il compito di avviare una valutazione di fattibilità tecnico-scientifica. Nel caso di iniziative appartenenti a particolari categorie di rilievo (def. 1.1), il sistema fornisce già in fase preliminare delle analisi tecniche automatiche, così da agevolare e velocizzare il processo di esame.
Il personale tecnico, una volta ricevuta la notifica di una nuova proposta, provvede a inoltrarla all’ufficio competente e a garantire la massima trasparenza del processo di valutazione, rendendo disponibili alla cittadinanza tutte le informazioni relative alle fasi di analisi. Al termine di tale processo, l’amministrazione comunica pubblicamente l’esito, che può consistere nell’accettazione e conseguente attuazione dell’iniziativa oppure nel suo rifiuto motivato.
L’intero sistema è implementato come applicazione web accessibile direttamente tramite browser, senza la necessità di installare software aggiuntivo da parte degli utenti. La piattaforma è progettata per garantire trasparenza, chiarezza e sicurezza, ponendosi come uno strumento al servizio sia della cittadinanza, che può contribuire attivamente al miglioramento del territorio, sia dell’amministrazione, che beneficia di un canale di comunicazione diretta, strutturata e partecipata.

\section{Definizioni}

\begin{description}
  \item[Definizione 1.1 – Categoria notevole:] 
  Per \textit{categoria notevole} si intende un ambito tematico di particolare rilevanza pubblica, la cui trattazione comporta un impatto significativo sul territorio comunale o sulla qualità della vita collettiva.  
  Le iniziative pubblicate all’interno di tali categorie vengono sottoposte dal sistema a una prima analisi tecnico-scientifica automatica, utile a supportare l’attività decisionale dell’amministrazione comunale.  
  Le principali categorie notevoli previste nella piattaforma sono:
  \begin{itemize}
    \item \textbf{Urbanistica} – Interventi su spazi pubblici, pianificazione urbana, mobilità e viabilità, arredo urbano e infrastrutture cittadine.
    \item \textbf{Ambiente} – Tutela del verde pubblico, riduzione dell’inquinamento, gestione dei rifiuti, risparmio energetico e promozione di pratiche sostenibili.
    \item \textbf{Sicurezza} – Installazione di sistemi di sorveglianza, illuminazione pubblica, segnaletica, percorsi pedonali e strategie di prevenzione del degrado urbano.
    \item \textbf{Cultura e istruzione} – Realizzazione di eventi culturali, potenziamento di biblioteche, spazi per attività educative, percorsi formativi e valorizzazione del patrimonio storico-artistico.
    \item \textbf{Innovazione e digitale} – Progetti di digitalizzazione dei servizi comunali, creazione di piattaforme online, installazione di sensori e infrastrutture per città intelligenti.
    \item \textbf{Sociale e inclusione} – Servizi di sostegno alle fasce vulnerabili, spazi comunitari, iniziative per l’inclusione sociale e il dialogo interculturale.
    \item \textbf{Mobilità sostenibile} – Piste ciclabili, trasporto pubblico ecologico, infrastrutture per veicoli elettrici e strategie per ridurre l’uso dell’auto privata.
  \end{itemize}
  L’elenco potrà essere esteso nel tempo per includere ulteriori categorie di interesse strategico per la cittadinanza.
  
  \vspace{1em}
  \item[Definizione 1.2 - Soglia di voto:] Numero minimo di voti necessario affinché una proposta venga inoltrata all’amministrazione per la valutazione di fattibilità.
\end{description}

\section{Vantaggi}

\subsection{Comune}

\begin{itemize}
  \item Prioritizzazione data-driven dei problemi: Il sistema di pubblicazione e 
  voto aiuta a individuare rapidamente le iniziative più rilevanti per la 
  comunità, riducendo il divario tra i bisogni percepiti sul territorio e l'agenda 
  amministrativa.

  \item Flusso operativo standardizzato e tracciabile: dall'arrivo della proposta 
  all'inoltro all'ufficio competente fino all'esito, con linea temporale pubblica e 
  regole chiare; meno email disperse e più ordine procedurale.

  \item Migliore allocazione delle risorse tecniche: per le "categorie notevoli" 
  il sistema propone valutazioni tecnico-scientifiche preventive, riducendo 
  analisi ridondanti e concentrando i tecnici sulle pratiche più impattanti.

  \item Canale unico e tracciabile: Ogni proposta segue un flusso chiaro 
  con registri pubblici che agevolano gli audit interni e garantiscono la 
  trasparenza verso l'esterno.

  \item Riduzione del carico sugli sportelli: Centralizzando segnalazioni e 
  proposte, diminuiscono email, PEC e richieste frammentate agli uffici/URP, 
  con risparmi operativi e di tempo.
\end{itemize}

\subsection{Cittadini}

\begin{itemize}
  \item Partecipazione effettiva: Pubblicare, votare e proporre modifiche rende 
  la cittadinanza protagonista nella definizione delle priorità pubbliche.

  \item Chiarezza sul percorso: Ogni iniziativa ha uno stato visibile e aggiornato 
  (in valutazione, accettata, rifiutata), riducendo l'asimmetria informativa.

  \item Accesso semplice via web: Nessuna installazione richiesta; la piattaforma 
  è usabile da browser e orientata a trasparenza, chiarezza e sicurezza.

  \item Tracciabilità personale: Ogni cittadino può seguire le proprie iniziative, 
  le modifiche proposte e i voti espressi.
\end{itemize}

\section{Limiti}

\begin{itemize}
  \item Definizione della soglia (def 1.2): Un limite troppo alto scoraggia; 
  troppo basso satura gli uffici. La taratura richiede monitoraggio e possibili 
  revisioni.

  \item Aspettative giuridiche ambigue: superare la soglia di voto può essere 
  interpretato come "diritto all'attuazione"; rischio di contenzioso se l'esito 
  è negativo.

  \item Competenza amministrativa: molte proposte ricadono su enti non comunali 
  (Provincia, Stato, gestori di servizi); serve un flusso di re-indirizzamento 
  chiaro.

  \item Costi ricorrenti: moderazione, comunicazione, supporto 
  utenti, osservabilità e test di sicurezza vanno finanziati in modo 
  continuativo.
\end{itemize}
