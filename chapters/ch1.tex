\chapter{Il Progetto Trento Decide}

\section{Introduzione}

Il progetto si fonda sull’assunto che il Comune di Trento, pur disponendo di competenze e strumenti amministrativi avanzati, non possa individuare e affrontare con assoluta completezza l’insieme delle problematiche presenti sul territorio comunale.  
Tale limite non deriva da inefficienze, ma è una conseguenza fisiologica della complessità urbana e della distanza che spesso si crea tra amministrazione e cittadinanza.

\textit{Trento Decide} nasce con l’obiettivo di ridurre tale divario, fungendo da ponte diretto tra cittadini e istituzioni.  

La piattaforma rende il processo decisionale più inclusivo, trasparente e reattivo, offrendo uno spazio digitale in cui proporre, discutere e votare iniziative di interesse pubblico. In questo senso, \textit{Trento Decide} si qualifica come un'iniziativa di e-democracy, applicando i principi della civic technology per modernizzare i processi partecipativi e promuovere una governance più aperta e collaborativa.

Gli utenti autenticati possono pubblicare nuove iniziative corredandole di titolo e descrizione, modificarle o proporre revisioni in collaborazione con altri partecipanti.  
Inoltre, possono esprimere il proprio voto sulle proposte esistenti, contribuendo così a determinarne la priorità e la rilevanza all’interno del processo decisionale civico.

Le proposte vengono periodicamente analizzate dagli amministratori comunali competenti, che avviano una valutazione di fattibilità tecnico-scientifica.  
Nel caso di iniziative appartenenti a particolari categorie notevoli (vedi Definizione~\ref{def:categoria-notevole}), il sistema fornisce in fase preliminare analisi tecniche automatiche utili a supportare e velocizzare l’esame.

Ogni proposta è automaticamente classificata in base alla propria categoria tematica e resa immediatamente accessibile agli assessorati e agli uffici competenti, che possono consultarla e valutarne la fattibilità.
L’intero processo di valutazione è tracciato nel sistema e reso trasparente alla cittadinanza attraverso la pubblicazione delle principali fasi di analisi.
Al termine, l’amministrazione comunica l’esito — accettazione con avvio dell’attuazione o rifiuto motivato — visibile direttamente nella pagina della proposta.

Il sistema è implementato come applicazione web accessibile via browser, senza necessità di installazioni.  
È concepito per favorire un’interazione continua e verificabile tra cittadini e amministrazione, garantendo trasparenza, chiarezza e sicurezza lungo l’intero ciclo di vita delle proposte.

\section{Definizioni}

Le seguenti definizioni stabiliscono la terminologia di riferimento adottata nel documento, assicurando coerenza e chiarezza tra le sezioni funzionali e tecniche.

\begin{description}

   \item[Definizione \refstepcounter{definition}\thedefinition\label{def:categoria-notevole} – Categoria notevole:]
  Ambito tematico di particolare rilevanza pubblica, la cui trattazione comporta un impatto significativo sul territorio comunale o sulla qualità della vita collettiva.  
  Le iniziative pubblicate in tali categorie sono sottoposte a una prima analisi tecnico-scientifica automatica a supporto dell’attività decisionale dell’amministrazione.  
  L’elenco e i criteri di eleggibilità sono definiti con deliberazione dell’amministrazione comunale.  
  Le principali categorie notevoli comprendono: Urbanistica, Ambiente, Sicurezza, Cultura e Istruzione, Innovazione e Digitale, Sociale e Inclusione, Mobilità Sostenibile.

  \vspace{1em}
  \item[Definizione \refstepcounter{definition}\thedefinition\label{def:utente} – Utente:] 
  Soggetto registrato e autenticato alla piattaforma \textit{Trento Decide} tramite SPID, CIE o credenziali locali.  
  Comprende tutti i ruoli predefiniti previsti dal sistema — cittadino, moderatore, amministratore comunale e rappresentante di associazione o comitato — come specificato nel Requisito Funzionale~\ref{sec:ruoli}.

  \item[Definizione \refstepcounter{definition}\thedefinition\label{def:cittadino} – Cittadino:] 
  Persona fisica di età pari o superiore a 16 anni, residente nel Comune di Trento o ivi domiciliata per motivi di studio o lavoro.  
  La verifica dell’identità e dell’eleggibilità al voto avviene tramite SPID o CIE.  
  La residenza è verificata tramite integrazione con ANPR, con procedure alternative in caso di indisponibilità del servizio.
  Per i non residenti, il legame territoriale è dichiarato tramite autocertificazione soggetta a verifiche a campione da parte dell’amministrazione comunale.  
  Le modalità di trattamento dei dati personali sono conformi alla normativa vigente sulla protezione dei dati (vedi Requisito Non Funzionale \ref{ref:privacy}).
  
  \item[Definizione \refstepcounter{definition}\thedefinition\label{def:associazione-comitato} – Associazione/Comitato:]  
  Entità giuridica registrata che presenta proposte collettive.  
  La rappresentanza è certificata tramite atto costitutivo e delega del legale rappresentante, caricati in piattaforma e soggetti a verifica amministrativa.  
  Le deleghe devono essere rinnovate annualmente o alla loro scadenza naturale, con notifica automatica di rinnovo.

  \item[Definizione \refstepcounter{definition}\thedefinition\label{def:proposta-collettiva} – Proposta collettiva:] 
  Proposta promossa da un gruppo di cittadini o da un’associazione riconosciuta.  
  Le regole di ponderazione del loro peso nel ranking sono pubbliche e documentate; il sistema rileva automaticamente eventuali pattern anomali (concentrazioni temporali, account recenti) e li segnala ai moderatori.

  \item[Definizione \refstepcounter{definition}\thedefinition\label{def:endorsement} – Endorsement:] 
  Azione di sostegno digitale espressa da cittadini o enti collettivi a favore di una proposta.  
  Il numero e il peso degli endorsement concorrono al ranking secondo coefficienti configurabili e trasparenti.

  \item[Definizione \refstepcounter{definition}\thedefinition\label{def:voto} – Voto:] 
  Espressione anonima \footnote{Il sistema garantisce l’anonimato del voto mediante separazione procedurale tra (a) verifica dell’eleggibilità dell’elettore e (b) registrazione del voto. Entro la finestra temporale definita l’utente può modificare il proprio voto senza che ciò consenta di ricollegarlo alla sua identità. Non sono adottate decisioni automatizzate aventi effetti giuridici o similmente significativi ai sensi dell’art. 22 GDPR.} di consenso o dissenso da parte di un cittadino autenticato tramite SPID o CIE.
 
\end{description}

\section{Vantaggi}

\subsection{Per il Comune}

\begin{itemize}
  \item \textbf{Prioritizzazione data-driven}: il sistema di pubblicazione, voto e ranking aiuta a individuare rapidamente le iniziative più rilevanti per la comunità, riducendo il divario tra bisogni territoriali e agenda amministrativa.

  \item \textbf{Flusso operativo tracciabile}: dalla creazione della proposta all’inoltro all’ufficio competente, con cronologia pubblica e regole chiare; maggiore trasparenza e meno frammentazione procedurale.

  \item \textbf{Allocazione efficiente delle risorse}: per le categorie notevoli il sistema fornisce pre-analisi tecnico-scientifiche, concentrando i tecnici sulle pratiche più impattanti.

  \item \textbf{Canale unico e verificabile}: ogni proposta segue un flusso chiaro con registri pubblici, agevolando audit interni e garantendo trasparenza verso la cittadinanza.

  \item \textbf{Riduzione del carico sugli sportelli}: la centralizzazione delle proposte diminuisce email, PEC e richieste frammentate agli URP, con risparmi operativi e di tempo.
\end{itemize}

\subsection{Per i cittadini}

\begin{itemize}
  \item \textbf{Partecipazione effettiva}: pubblicare, votare e proporre modifiche rende la cittadinanza protagonista nella definizione delle priorità pubbliche.

  \item \textbf{Chiarezza sul percorso}: ogni proposta riporta in modo trasparente il proprio stato
  (\textit{bozza}, \textit{in valutazione}, \textit{accettata}, \textit{rifiutata}),
  in conformità al Requisito Funzionale \ref{sec:stato-tracciabilita}, garantendo tracciabilità e riducendo l’asimmetria informativa.

  \item \textbf{Accesso semplice e universale}: la piattaforma è accessibile via web senza installazioni.

  \item \textbf{Tracciabilità personale}: ogni cittadino può seguire le proprie iniziative, modifiche e voti espressi.
\end{itemize}

\section{Limiti e mitigazioni}

Il progetto \textit{Trento Decide}, pur garantendo un elevato grado di trasparenza e partecipazione, presenta alcuni limiti intrinseci di natura tecnica, organizzativa e sociale.  
La seguente sezione identifica i principali rischi e descrive le strategie di mitigazione previste.

\begin{itemize}

  \item \textbf{Definizione dei criteri di priorità}:  
  Stabilire parametri oggettivi per la valutazione delle proposte è complesso.  
  La formula di ranking richiede un monitoraggio continuo e revisioni periodiche, come previsto dal requisito non funzionale sulla neutralità algoritmica (\ref{ref:neutralita}).

  \item \textbf{Neutralità e bias cognitivi}:  
  Le decisioni automatizzate e i comportamenti collettivi possono introdurre distorsioni involontarie.  
  Si prevede la revisione annuale degli algoritmi e l’audit di neutralità per garantire imparzialità e trasparenza dei processi.

  \item \textbf{Rischio di manipolazione o brigading}:  
  Coordinamenti di gruppo o attività fraudolente possono alterare voti e ranking.  
  Il sistema applica limiti di frequenza, verifica d’identità tramite SPID/CIE e rilevamento automatico di anomalie, integrato con la moderazione (\ref{ref:moderazione}).

  \item \textbf{Digital divide}:  
  Una parte della popolazione può trovarsi in difficoltà nell’uso di strumenti digitali.  
  Sono previsti punti di facilitazione presso biblioteche e sportelli comunali, supporto telefonico e assistenza tramite delega per l’inserimento delle proposte.

  \item \textbf{Competenza multi-ente}:  
  Alcune proposte ricadono su enti diversi dal Comune di Trento (Provincia, Stato, gestori di servizi pubblici).  
  Il sistema deve prevedere un flusso chiaro e tracciabile di reindirizzamento e presa verso l’ente competente.

  \item \textbf{Aspettative giuridiche ambigue}:  
  L’approvazione o la priorità elevata di una proposta non implicano un diritto automatico alla sua attuazione.  
  L’amministrazione pubblica deve fornire motivazioni chiare e documentate per ogni decisione finale.

  \item \textbf{Costi ricorrenti e sostenibilità}:  
  Il mantenimento della piattaforma richiede risorse continuative per moderazione, assistenza, comunicazione, sicurezza e aggiornamenti software.  
  È raccomandata la previsione di un piano di sostenibilità tecnica ed economica pluriennale.
\end{itemize}