\chapter{Il Progetto Trento Decide}

\section{Introduzione}

Il progetto si fonda sull’assunto che il Comune di Trento, pur disponendo di competenze e strumenti amministrativi avanzati, non possa individuare e affrontare con assoluta completezza l’insieme delle problematiche presenti sul territorio comunale.  
Tale limite non deriva da inefficienze, ma è una conseguenza fisiologica della complessità urbana e della distanza che spesso si crea tra amministrazione e cittadinanza.

\textit{Trento Decide} nasce con l’obiettivo di ridurre tale divario, fungendo da ponte diretto tra cittadini e istituzioni.  

La piattaforma rende il processo decisionale più inclusivo, trasparente e reattivo, offrendo uno spazio digitale in cui proporre, discutere e votare iniziative di interesse pubblico. In questo senso, \textit{Trento Decide} si qualifica come un'iniziativa di e-democracy, applicando i principi della civic technology per modernizzare i processi partecipativi e promuovere una governance più aperta e collaborativa.

Gli utenti autenticati possono pubblicare nuove iniziative corredandole di titolo e descrizione, modificarle o proporre revisioni in collaborazione con altri partecipanti.  
Inoltre, possono esprimere il proprio voto sulle proposte esistenti, contribuendo così a determinarne la priorità e la rilevanza all’interno del processo decisionale civico.

Le proposte vengono periodicamente analizzate dagli amministratori comunali competenti, che avviano una valutazione di fattibilità tecnico-scientifica.  
Nel caso di iniziative appartenenti a particolari categorie notevoli (vedi Definizione~\ref{def:categoria-notevole}), il sistema fornisce in fase preliminare analisi tecniche automatiche utili a supportare e velocizzare l’esame.

Ogni proposta è automaticamente classificata in base alla propria categoria tematica e resa immediatamente accessibile agli assessorati e agli uffici competenti, che possono consultarla e valutarne la fattibilità.
L’intero processo di valutazione è tracciato nel sistema e reso trasparente alla cittadinanza attraverso la pubblicazione delle principali fasi di analisi.
Al termine, l’amministrazione comunica l’esito — accettazione con avvio dell’attuazione o rifiuto motivato — visibile direttamente nella pagina della proposta.

Il sistema è implementato come applicazione web accessibile via browser, senza necessità di installazioni.  
È concepito per favorire un’interazione continua e verificabile tra cittadini e amministrazione, garantendo trasparenza, chiarezza e sicurezza lungo l’intero ciclo di vita delle proposte.

\section{Definizioni}

Le seguenti definizioni stabiliscono la terminologia di riferimento adottata nel documento, assicurando coerenza e chiarezza tra le sezioni funzionali e tecniche.

\begin{description}

   \item[Definizione \refstepcounter{definition}\thedefinition\label{def:categoria-notevole} – Categoria notevole:]
  Ambito tematico di particolare rilevanza pubblica, la cui trattazione comporta un impatto significativo sul territorio comunale o sulla qualità della vita collettiva.  
  Le iniziative pubblicate in tali categorie sono sottoposte a una prima analisi tecnico-scientifica automatica a supporto dell’attività decisionale dell’amministrazione.  
  L’elenco e i criteri di eleggibilità sono definiti con deliberazione dell’amministrazione comunale.  
  Le principali categorie notevoli comprendono: Urbanistica, Ambiente, Sicurezza, Cultura e Istruzione, Innovazione e Digitale, Sociale e Inclusione, Mobilità Sostenibile.

  \vspace{1em}
  \item[Definizione \refstepcounter{definition}\thedefinition\label{def:utente} – Utente:] 
  Soggetto registrato e autenticato alla piattaforma \textit{Trento Decide} tramite SPID, CIE o credenziali locali.  
  Comprende tutti i ruoli predefiniti previsti dal sistema — cittadino, moderatore, amministratore comunale e rappresentante di associazione o comitato — come specificato nel Requisito Funzionale~\ref{sec:ruoli}.

  \item[Definizione \refstepcounter{definition}\thedefinition\label{def:cittadino} – Cittadino:] 
  Persona fisica di età pari o superiore a 16 anni \footnote{Soglia 16 anni: scelta di policy per garantire partecipazione giovanile matura; distinta dal limite di 14 anni per il consenso ai servizi digitali (D.lgs. 101/2018 su art. 8 GDPR). Base giuridica del trattamento: art. 6(1)(e) GDPR.}, residente nel Comune di Trento o ivi domiciliata per motivi di studio o lavoro.  
  La dichiarazione del legame territoriale per i non residenti avviene tramite autocertificazione ai sensi del DPR 445/2000, soggetta a verifica a campione da parte dell’amministrazione comunale.
  L’identificazione avviene tramite SPID o CIE e i dati raccolti sono trattati esclusivamente per la verifica dell’eleggibilità al voto. 

  \item[Definizione \refstepcounter{definition}\thedefinition\label{def:residente} – Residente:] 
  Soggetto iscritto all’ANPR con indirizzo nel Comune di Trento.  
  La verifica della residenza avviene in tempo reale tramite integrazione con ANPR o, in caso di indisponibilità del servizio, tramite procedura documentale differita (batch).

  \item[Definizione \refstepcounter{definition}\thedefinition\label{def:associazione-comitato} – Associazione/Comitato:]  
  Entità giuridica registrata che presenta proposte collettive.  
  La rappresentanza è certificata tramite atto costitutivo e delega del legale rappresentante, caricati in piattaforma e soggetti a verifica amministrativa.  
  Le deleghe devono essere rinnovate annualmente o alla loro scadenza naturale, con notifica automatica di rinnovo.

  \item[Definizione \refstepcounter{definition}\thedefinition\label{def:proposta-collettiva} – Proposta collettiva:] 
  Proposta promossa da un gruppo di cittadini o da un’associazione riconosciuta.  
  Le regole di ponderazione del loro peso nel ranking sono pubbliche e documentate; il sistema rileva automaticamente eventuali pattern anomali (concentrazioni temporali, account recenti) e li segnala ai moderatori.

  \item[Definizione \refstepcounter{definition}\thedefinition\label{def:endorsement} – Endorsement:] 
  Azione di sostegno digitale espressa da cittadini o enti collettivi a favore di una proposta.  
  Il numero e il peso degli endorsement concorrono al ranking secondo coefficienti configurabili e trasparenti.

  \item[Definizione \refstepcounter{definition}\thedefinition\label{def:voto} – Voto:] 
  Espressione anonima \footnote{Il sistema garantisce l’anonimato del voto mediante separazione procedurale tra (a) verifica dell’eleggibilità dell’elettore e (b) registrazione del voto. Entro la finestra temporale definita l’utente può modificare il proprio voto senza che ciò consenta di ricollegarlo alla sua identità. Non sono adottate decisioni automatizzate aventi effetti giuridici o similmente significativi ai sensi dell’art. 22 GDPR.} di consenso o dissenso da parte di un cittadino autenticato tramite SPID o CIE.
 
\end{description}

\section{Vantaggi}

\subsection{Per il Comune}

\begin{itemize}
  \item \textbf{Prioritizzazione data-driven}: il sistema di pubblicazione, voto e ranking aiuta a individuare rapidamente le iniziative più rilevanti per la comunità, riducendo il divario tra bisogni territoriali e agenda amministrativa.

  \item \textbf{Flusso operativo tracciabile}: dalla creazione della proposta all’inoltro all’ufficio competente, con cronologia pubblica e regole chiare; maggiore trasparenza e meno frammentazione procedurale.

  \item \textbf{Allocazione efficiente delle risorse}: per le categorie notevoli il sistema fornisce pre-analisi tecnico-scientifiche, concentrando i tecnici sulle pratiche più impattanti.

  \item \textbf{Canale unico e verificabile}: ogni proposta segue un flusso chiaro con registri pubblici, agevolando audit interni e garantendo trasparenza verso la cittadinanza.

  \item \textbf{Riduzione del carico sugli sportelli}: la centralizzazione delle proposte diminuisce email, PEC e richieste frammentate agli URP, con risparmi operativi e di tempo.
\end{itemize}

\subsection{Per i cittadini}

\begin{itemize}
  \item \textbf{Partecipazione effettiva}: pubblicare, votare e proporre modifiche rende la cittadinanza protagonista nella definizione delle priorità pubbliche.

  \item \textbf{Chiarezza sul percorso}: ogni proposta riporta in modo trasparente il proprio stato
  (\textit{bozza}, \textit{in valutazione}, \textit{accettata}, \textit{rifiutata}),
  in conformità al Requisito Funzionale \ref{sec:stato-tracciabilita}, garantendo tracciabilità e riducendo l’asimmetria informativa.

  \item \textbf{Accesso semplice e universale}: la piattaforma è accessibile via web senza installazioni.

  \item \textbf{Tracciabilità personale}: ogni cittadino può seguire le proprie iniziative, modifiche e voti espressi.
\end{itemize}

\section{Limiti e mitigazioni}

\begin{itemize}
  \item \textbf{Definizione dei criteri di priorità}: è complesso stabilire parametri oggettivi per determinare l’ordine di rilevanza delle proposte; la taratura del ranking richiede monitoraggio costante e revisione periodica da parte dell’amministrazione.

  \item \textbf{Rischio di manipolazione o brigading}: per prevenire comportamenti coordinati, il sistema applica limiti di frequenza configurabili, CAPTCHA accessibili e rilevamento di anomalie basato su impronte dispositivo pseudonimizzate.
  È inoltre previsto il monitoraggio dei pattern di traffico e di comportamento in collaborazione con i presidi di sicurezza comunali, per individuare eventuali campagne coordinate o attività di disinformazione.
  Le misure adottate sono proporzionate e trasparenti. L’eventuale uso di impronte di dispositivo pseudonimizzate è subordinato a DPIA e adeguata informativa, ed avviene solo se strettamente necessario a fronte di evidenze di attacchi coordinati.

  \item \textbf{Neutralità e bias cognitivi}: 
  non è possibile garantire la completa assenza di bias politici o sociali, poiché la definizione dei criteri di valutazione e ranking implica scelte di policy.  
  Il rischio viene mitigato attraverso la trasparenza algoritmica (\ref{ref:neutralita}) e l’applicazione delle linee guida etiche definite nel requisito~\ref{ref:etica}.

  \item \textbf{Digital divide}: per garantire inclusività sono previsti punti di facilitazione digitale presso biblioteche e sportelli comunali, canali telefonici di supporto e la possibilità di inserimento assistito delle proposte tramite delega.

  \item \textbf{Competenza multi-ente}: alcune proposte possono ricadere su enti diversi dal Comune di Trento (ad esempio la Provincia Autonoma di Trento, lo Stato o gestori di servizi pubblici).
  Il sistema deve quindi prevedere un flusso chiaro e tracciabile di re-indirizzamento verso l’ente competente.

  \item \textbf{Aspettative giuridiche ambigue}: l’ottenimento di alta priorità o approvazione può essere erroneamente interpretato come diritto all’attuazione; l’amministrazione pubblica pertanto motivazioni chiare e comprensibili per ogni decisione.

  \item \textbf{Costi ricorrenti e sostenibilità}: moderazione, comunicazione, supporto utenti, osservabilità e sicurezza devono essere mantenuti e finanziati in modo continuativo.

  \item \textbf{Rischio di saturazione partecipativa}: l'entusiasmo iniziale dei cittadini potrebbe diminuire nel tempo, portando a un calo della partecipazione attiva e riducendo la vitalità della piattaforma. Per mitigare questo rischio, sono previste strategie di coinvolgimento proattivo, come campagne di comunicazione periodiche, la gamification di alcune attività (es. badge per i "super-partecipanti") e la promozione di consultazioni su temi di forte impatto mediatico.
\end{itemize}