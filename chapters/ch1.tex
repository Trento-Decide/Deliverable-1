\chapter{Il Progetto Trento Decide}

\section{Introduzione}

Il progetto si basa sull'assunto che il comune di Trento non sia in grado di 
determinare con precisione in maniera assolutamente ottimale la totalità dei 
problemi apprezzabili nel territorio comunale. Trento Decide si pone come 
piattaforma che mira a ridurre il gap tra l'efficienza comunale attuale, e un' 
utopistica efficienza assoluta. Nel software Trento Decide i cittadini, 
autenticatisi come tali, possono: pubblicare, modificare e votare iniziative di 
carattere pubblico, nei più svariati ambiti utili alla cittadinanza, come 
urbanistica, ambiente, cultura, etc.
I cittadini votando una proposta pubblicata, hanno la possibilità di portarla 
all'attenzione del comune, che si occuperà a quel punto di mettere in atto una 
breve analisi di fattibilità tecnica, al termine della quale verrà pubblicata 
insieme all'esito riguardo l'iniziativa, che può essere rifiutata o accettata e 
dunque concretizzata. \\
Il progetto consiste in pratica in un' applicazione web, accessibile via browser
e destinata a due tipologie di utenti principali: cittadini e tecnici comunali.
Gli utenti nello specifico potranno: pubblicare iniziative, motivandole attraverso 
titolo e descrizione, se tale iniziativa viene pubblicata in una categoria 
notevole (def 1.1) il sistema proporrà automaticamente delle preventive 
valutazioni tecnico-scientifiche. Un cittadino in disaccordo o parziale accordo 
con un' iniziativa potrà proporre di modificarla, giustificandone il motivo. Un 
cittadino in accordo con un'iniziativa potrà votarla, per portarla all'attenzione 
del comune. Un tecnico del comune, una volta notificatagli un'iniziativa la cui 
somma dei voti abbia superato il limite previsto di voto (def 1.2), dovrà 
inoltrarla all'ufficio di competenza, e fornire in maniera trasparente tutte le 
informazioni sul processo di valutazione al pubblico. Una volta completata 
l'analisi tecnica, il tecnico riporterà i dati alla cittadinanza, de facto 
notificando il rifiuto o l'accettazione della proposta. \\
La piattaforma sarà totalmente disponibile e usabile via browser, senza la 
necessità di installare alcun pacchetto aggiuntivo, progettato per garantire: 
trasparenza, chiarezza e sicurezza.

\section{Vantaggi}

\subsection{Comune}

\begin{itemize}
  \item Prioritizzazione data-driven dei problemi: Il sistema di pubblicazione e 
  voto aiuta a individuare rapidamente le iniziative più rilevanti per la 
  comunità, riducendo il divario tra i bisogni percepiti sul territorio e l'agenda 
  amministrativa.

  \item Flusso operativo standardizzato e tracciabile: dall'arrivo della proposta 
  all'inoltro all'ufficio competente fino all'esito, con linea temporale pubblica e 
  regole chiare; meno email disperse e più ordine procedurale.

  \item Migliore allocazione delle risorse tecniche: per le "categorie notevoli" 
  il sistema propone valutazioni tecnico-scientifiche preventive, riducendo 
  analisi ridondanti e concentrando i tecnici sulle pratiche più impattanti.

  \item Canale unico e tracciabile: Ogni proposta segue un flusso chiaro 
  con registri pubblici che agevolano gli audit interni e garantiscono la 
  trasparenza verso l'esterno.

  \item Riduzione del carico sugli sportelli: Centralizzando segnalazioni e 
  proposte, diminuiscono email, PEC e richieste frammentate agli uffici/URP, 
  con risparmi operativi e di tempo.
\end{itemize}

\subsection{Cittadini}

\begin{itemize}
  \item Partecipazione effettiva: Pubblicare, votare e proporre modifiche rende 
  la cittadinanza protagonista nella definizione delle priorità pubbliche.

  \item Chiarezza sul percorso: Ogni iniziativa ha uno stato visibile e aggiornato 
  (in valutazione, accettata, rifiutata), riducendo l'asimmetria informativa.

  \item Accesso semplice via web: Nessuna installazione richiesta; la piattaforma 
  è usabile da browser e orientata a trasparenza, chiarezza e sicurezza.

  \item Tracciabilità personale: Ogni cittadino può seguire le proprie iniziative, 
  le modifiche proposte e i voti espressi.
\end{itemize}

\section{Limiti}

\begin{itemize}
  \item Definizione della soglia (def 1.2): Un limite troppo alto scoraggia; 
  troppo basso satura gli uffici. La taratura richiede monitoraggio e possibili 
  revisioni.

  \item Aspettative giuridiche ambigue: superare la soglia di voto può essere 
  interpretato come "diritto all'attuazione"; rischio di contenzioso se l'esito 
  è negativo.

  \item Competenza amministrativa: molte proposte ricadono su enti non comunali 
  (Provincia, Stato, gestori di servizi); serve un flusso di re-indirizzamento 
  chiaro.

  \item Costi ricorrenti: moderazione, comunicazione, supporto 
  utenti, osservabilità e test di sicurezza vanno finanziati in modo 
  continuativo.
\end{itemize}
