\chapter{Il Progetto Trento Decide}

\section{Introduzione}

Il progetto si fonda sull’assunto che il Comune di Trento, pur disponendo di competenze e strumenti amministrativi avanzati, non possa individuare e affrontare con assoluta completezza l’insieme delle problematiche presenti sul territorio comunale.  
Tale limite non deriva da inefficienze, ma è una conseguenza fisiologica della complessità urbana e della distanza che spesso si crea tra amministrazione e cittadinanza.

\textit{Trento Decide} nasce con l’obiettivo di ridurre tale divario, fungendo da ponte diretto tra cittadini e istituzioni.  

La piattaforma rende il processo decisionale più inclusivo, trasparente e reattivo, offrendo uno spazio digitale in cui proporre, discutere e votare iniziative di interesse pubblico. In questo senso, \textit{Trento Decide} si qualifica come un'iniziativa di e-democracy, applicando i principi della civic technology per modernizzare i processi partecipativi e promuovere una governance più aperta e collaborativa.

I cittadini possono pubblicare nuove iniziative, modificarle o proporre revisioni in collaborazione con altri partecipanti.  
Inoltre, possono esprimere il proprio voto sulle proposte esistenti, contribuendo così a determinarne la priorità e la rilevanza all’interno del processo decisionale civico.

Le proposte vengono periodicamente analizzate dall'amministrazione comunale, che ne verifica la fattibilità tecnico-scientifica.  
Nel caso di iniziative appartenenti a particolari categorie "notevoli" \defref{def:categoria-notevole}, il sistema fornisce in fase preliminare analisi tecniche automatiche utili a supportare e velocizzare l’esame.

L’intero processo di valutazione è tracciato nel sistema e reso trasparente alla cittadinanza attraverso la pubblicazione delle principali fasi di analisi.
Al termine, l’amministrazione comunica l’esito — accettazione con avvio dell’attuazione o rifiuto motivato — visibile direttamente nella pagina della proposta.

Il sistema è implementato come applicazione web accessibile via browser, senza necessità di installazioni.  
È concepito per favorire un’interazione continua e verificabile tra cittadini e amministrazione, garantendo trasparenza, chiarezza e sicurezza lungo l’intero ciclo di vita delle proposte.

\section{Definizioni}

Le seguenti definizioni stabiliscono la terminologia di riferimento adottata nel documento, assicurando coerenza e chiarezza tra le sezioni funzionali e tecniche.

\begin{description}

\item[Definizione \refstepcounter{definition}\thedefinition\label{def:categoria} – Categoria:] 
Ambito tematico utilizzato per classificare le proposte in fase di creazione e pubblicazione.  
Ogni categoria può richiedere campi aggiuntivi specifici necessari alla valutazione tecnico-amministrativa.  
Le categorie previste sono:

\begin{itemize}
	\item \textbf{Urbanistica} — interventi sulla pianificazione e trasformazione urbana (campi tipici: tipologia intervento, destinazione d’uso).
	\item \textbf{Ambiente} — proposte a tutela dell’ambiente e della sostenibilità (campi tipici: ambito ambientale, impatto previsto).
	\item \textbf{Sicurezza} — segnalazioni o interventi su rischi e criticità pubbliche (campi tipici: tipo di criticità, priorità).
	\item \textbf{Cultura} — iniziative culturali e artistiche (campi tipici: ambito culturale, target).
	\item \textbf{Istruzione} — interventi per scuole e servizi formativi (campi tipici: livello educativo, benefici).
	\item \textbf{Innovazione e Digitale} — progetti tecnologici o di digitalizzazione (campi tipici: tecnologia proposta, processo coinvolto).
	\item \textbf{Sociale e Inclusione} — iniziative rivolte al benessere e all’inclusione sociale (campi tipici: destinatari, problema affrontato).
	\item \textbf{Mobilità e Trasporti} — proposte sul sistema di mobilità urbana (campi tipici: tipologia intervento, impatto sulla mobilità).
	\item \textbf{Welfare} — interventi socio-assistenziali (campi tipici: destinatari, impatto sociale).
	\item \textbf{Sport e Tempo Libero} — attività e infrastrutture sportive o ricreative (campi tipici: tipo attività, struttura coinvolta).
	\item \textbf{Patrimonio e Lavori Pubblici} — manutenzione o interventi su edifici e infrastrutture pubbliche (campi tipici: tipo intervento, urgenza).
\end{itemize}
  
\item[Definizione \refstepcounter{definition}\thedefinition\label{def:classificazione} – Classificazione:] 
Meccanismo interno che consente all’amministrazione comunale di ordinare e filtrare le proposte sulla base di un indicatore sintetico di “valore”, inteso come rilevanza amministrativa o impatto potenziale.  
Ogni proposta è associata a un punteggio calcolato tramite coefficienti moltiplicativi configurabili dall’amministrazione e non visibili al pubblico.

  \item[Definizione \refstepcounter{definition}\thedefinition\label{def:utente} – Utente:] 
  Soggetto registrato e autenticato alla piattaforma \textit{Trento Decide}. 
  Comprende tutti i ruoli predefiniti previsti dal sistema: cittadino, moderatore, amministratore comunale e rappresentante di associazione \defref{def:ruoli}.

\item[Definizione \refstepcounter{definition}\thedefinition\label{def:ruoli} – Ruoli utente:] 
La piattaforma \textit{Trento Decide} prevede quattro ruoli distinti, utilizzati per regolare l’accesso alle funzionalità del sistema.

\textbf{Cittadino}: 
Persona fisica di età pari o superiore a 18 anni, residente nel Comune di Trento o ivi domiciliata per studio o lavoro.  
L’identità e l’eleggibilità al voto sono verificate tramite SPID o CIE.  
La residenza è verificata tramite integrazione con ANPR.

\textbf{Associazione}:  
Entità giuridica registrata che presenta proposte collettive.  
La rappresentanza è certificata tramite atto costitutivo e delega del legale rappresentante, caricati in piattaforma e soggetti a verifica amministrativa.  

\textbf{Moderatore}:  
Utente incaricato dell’attività di verifica e controllo delle proposte e dei contenuti generati dagli utenti.  
L’account è creato dall’amministratore e dispone di permessi di gestione limitati alle funzioni di moderazione.

\textbf{Amministratore}:  
Utente con privilegi completi di gestione della piattaforma.  
È responsabile della creazione degli account speciali (moderatori, associazioni/comitati), della supervisione dei processi di verifica e della configurazione dei parametri operativi del sistema.

\item[Definizione \refstepcounter{definition}\thedefinition\label{def:statoproposta} – Stato della proposta:]
Ogni proposta è associata a uno stato che ne descrive l’avanzamento nel processo amministrativo.  
Gli stati previsti sono: \textit{bozza}, \textit{pubblicata}, \textit{in valutazione}, \textit{accettata}, \textit{rifiutata}, \textit{in attuazione}, \textit{completata}.  

\item[Definizione \refstepcounter{definition}\thedefinition\label{def:proposta-collettiva} – Proposta collettiva:] 
Proposta presentata da un’associazione o comitato riconosciuto.  
Nel processo di classificazione \defref{def:classificazione}, le proposte collettive sono soggette a una sezione dedicata che consente all’amministrazione di applicare criteri o pesi specifici rispetto alle proposte individuali.

  \item[Definizione \refstepcounter{definition}\thedefinition\label{def:endorsement} – Endorsement:] 
  Azione di sostegno digitale espressa da associazioni a favore di una proposta.  
  Il numero e il peso degli endorsement concorrono alla classificazione \defref{def:classificazione}.
 
\end{description}

\section{Vantaggi}

\subsection{Per il Comune}

\begin{itemize}
  \item \textbf{Prioritizzazione data-driven}: il sistema di pubblicazione, voto e ranking aiuta a individuare rapidamente le iniziative più rilevanti per la comunità, riducendo il divario tra bisogni territoriali e agenda amministrativa.

  \item \textbf{Flusso operativo tracciabile}: dalla creazione della proposta all’inoltro all’ufficio competente, con cronologia pubblica e regole chiare; maggiore trasparenza e meno frammentazione procedurale.

  \item \textbf{Allocazione efficiente delle risorse}: per le categorie notevoli il sistema fornisce pre-analisi tecnico-scientifiche, concentrando i tecnici sulle pratiche più impattanti.

  \item \textbf{Canale unico e verificabile}: ogni proposta segue un flusso chiaro con registri pubblici, agevolando audit interni e garantendo trasparenza verso la cittadinanza.

  \item \textbf{Riduzione del carico sugli sportelli}: la centralizzazione delle proposte diminuisce email, PEC e richieste frammentate agli URP, con risparmi operativi e di tempo.
\end{itemize}

\subsection{Per i cittadini}

\begin{itemize}
  \item \textbf{Partecipazione effettiva}: pubblicare, votare e proporre modifiche rende la cittadinanza protagonista nella definizione delle priorità pubbliche.

  \item \textbf{Chiarezza sul percorso}: ogni proposta riporta in modo trasparente il proprio stato
  (\textit{bozza}, \textit{in valutazione}, \textit{accettata}, \textit{rifiutata}),
  in conformità al Requisito Funzionale \ref{sec:stato-tracciabilita}, garantendo tracciabilità e riducendo l’asimmetria informativa.

  \item \textbf{Accesso semplice e universale}: la piattaforma è accessibile via web senza installazioni.

  \item \textbf{Tracciabilità personale}: ogni cittadino può seguire le proprie iniziative, modifiche e voti espressi.
\end{itemize}

\section{Limiti e mitigazioni}

Il progetto \textit{Trento Decide}, pur garantendo un elevato grado di trasparenza e partecipazione, presenta alcuni limiti intrinseci di natura tecnica, organizzativa e sociale.  
La seguente sezione identifica i principali rischi e descrive le strategie di mitigazione previste.

\begin{itemize}

  \item \textbf{Definizione dei criteri di priorità}:  
  Stabilire parametri oggettivi per la valutazione delle proposte è complesso.  
  La formula di ranking richiede un monitoraggio continuo e revisioni periodiche, come previsto dal Requisito Non Funzionale sulla neutralità algoritmica (\ref{ref:neutralita}).

  \item \textbf{Neutralità e bias cognitivi}:  
  Le decisioni automatizzate e i comportamenti collettivi possono introdurre distorsioni involontarie.  
  Si prevede la revisione annuale degli algoritmi e l’audit di neutralità per garantire imparzialità e trasparenza dei processi.

  \item \textbf{Rischio di manipolazione o brigading}:  
  Coordinamenti di gruppo o attività fraudolente possono alterare voti e ranking.  
  Il sistema applica limiti di frequenza, verifica d’identità tramite SPID/CIE e rilevamento automatico di anomalie, integrato con la moderazione (\ref{ref:moderazione}).

  \item \textbf{Digital divide}:  
  Una parte della popolazione può trovarsi in difficoltà nell’uso di strumenti digitali.  
  Sono previsti punti di facilitazione presso biblioteche e sportelli comunali, supporto telefonico e assistenza tramite delega per l’inserimento delle proposte.

  \item \textbf{Competenza multi-ente}:  
  Alcune proposte ricadono su enti diversi dal Comune di Trento (Provincia, Stato, gestori di servizi pubblici).  
  Il sistema deve prevedere un flusso chiaro e tracciabile di reindirizzamento e presa verso l’ente competente.

  \item \textbf{Aspettative giuridiche ambigue}:  
  L’approvazione o la priorità elevata di una proposta non implicano un diritto automatico alla sua attuazione.  
  L’amministrazione pubblica deve fornire motivazioni chiare e documentate per ogni decisione finale.

  \item \textbf{Costi ricorrenti e sostenibilità}:  
  Il mantenimento della piattaforma richiede risorse continuative per moderazione, assistenza, comunicazione, sicurezza e aggiornamenti software.  
  È raccomandata la previsione di un piano di sostenibilità tecnica ed economica pluriennale.
\end{itemize}