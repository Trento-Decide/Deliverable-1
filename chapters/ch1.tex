% !TeX spellcheck = it_IT
\chapter{Il Progetto Trento Decide}

\section{Introduzione}

Il progetto si fonda sull'assunto che il Comune di Trento non possieda - e per fisiologia strutturale di un comune non possiederà mai - una visione a basso livello sufficientemente vasta per poter individuare con assoluta correttezza tutte le possibili migliorie apportabili sul territorio.
\emph{Trento Decide} nasce con l'obiettivo di fornire quanto più possibile questa visione al Comune, avvicinandolo ai cittadini.
La piattaforma si offre come ponte tra la cittadinanza e i processi decisionali interni all'amministrazione comunale, fornendo uno spazio dove ognuno può proporre, discutere e votare iniziative di interesse pubblico.

Nello specifico, ogni cittadino può esporre alla cittadinanza proposte sotto diverse categorie di notevole interesse civico (come urbanistica, ambiente, ecc.); tali proposte sono votabili, esponendole in risalto agli occhi della comunità.

Le proposte vengono periodicamente analizzate dal Comune, che valuta la fattibilità tecnico-scientifica delle proposte scelte, di fatto approvandole o rifiutandole.
L'intero processo di valutazione è reso trasparente attraverso la pubblicazione delle fasi dell'analisi.

Il sistema è implementato come applicazione web accessibile via browser, senza necessità di installazioni; è inoltre progettato per offrire un'esperienza utente intuitiva e sicura, ed è parallelamente conforme alle normative vigenti in materia di protezione dei dati personali.

\emph{Trento Decide} si pone come mezzo di evoluzione in ambito Civic Tech, offrendo alla comunità uno strumento potente e allo stesso tempo totalmente trasparente. 
In questo modo si offre alla comunità uno spazio dove la libertà di espressione e la creatività sono il fondamento per costruire una vita comunale migliore.

\section{Definizioni}

Le definizioni che seguono stabiliscono i riferimenti utilizzati in tutta la documentazione del progetto.

\begin{description}

\item[Definizione \refstepcounter{definition}\thedefinition\label{def:categoria} – Categoria:]
Ambito di notevole impatto per il comune, utilizzato nella classificazione delle proposte in fase di creazione e di visualizzazione.
Trattasi del criterio di differenziazione tra proposte più espressivo.
Nella creazione di una proposta ogni categoria richiede campi specifici da compilare, presenti a seguito di ogni definizione. \\
Le categorie previste sono:

\begin{itemize}
	\item \textbf{Urbanistica}: pianificazione urbana (campi specifici: tipologia intervento, destinazione d’uso).
	\item \textbf{Ambiente}: tutela ambientale e sostenibilità (campi specifici: ambito ambientale, impatto previsto).
	\item \textbf{Sicurezza}: segnalazioni relative a rischi e criticità pubbliche (campi specifici: tipo di criticità, priorità).
	\item \textbf{Cultura}: iniziative di carattere culturale o artistico (campi specifici: ambito culturale, target).
	\item \textbf{Istruzione}: interventi rivolti a scuole e servizi formativi (campi specifici: livello educativo, ambito educativo).
	\item \textbf{Innovazione}: proposte di implementazioni tecnologiche (campi specifici: tecnologia proposta, processo coinvolto).
	\item \textbf{Mobilità e Trasporti}: Mobilità urbana e Trasporti pubblici (campi specifici: ambito trasporti, impatto sulla mobilità).
	\item \textbf{Welfare}: Assistenza fornita (campi specifici: destinatari, impatto sociale).
	\item \textbf{Sport}: Infrastrutture sportive (campi specifici: tipo attività, struttura coinvolta).
\end{itemize}

\item[Definizione \refstepcounter{definition}\thedefinition\label{def:utente} – Utente:]
  Soggetto registrato e autenticato alla piattaforma.
  L'utente rappresenta l'insieme di tutti i ruoli previsti: cittadino, moderatore, amministratore comunale e rappresentante di associazione \defref{def:ruoli}.

\item[Definizione \refstepcounter{definition}\thedefinition\label{def:ruoli} – Ruoli:]
La piattaforma prevede quattro ruoli distinti, qui definiti: 

\textbf{Cittadino}:
Persona fisica di età pari o superiore a 16 anni, residente nel Comune di Trento.  
Le verifiche sulla residenza ed eleggibilità al voto sulla piattaforma sono effettuate attraverso SPID o CIE\@ e ANPR\@.

\textbf{Associazione}:
Entità registrata come associazione nel Comune di Trento.
Le associazioni dispongono di tutte le funzionalità previste per i cittadini, con in aggiunta quelle relative alla creazione delle proposte collettive ed endorsement \defref{def:propostacollettiva}, \defref{def:endorsement}.

\textbf{Moderatore}:
Incaricato nella verifica della conformità dei contenuti pubblici e della gestione delle segnalazioni.

\textbf{Amministratore}:
Utente con privilegi massimi nella piattaforma.
È responsabile della pubblicazione di report e sondaggi, della supervisione dei processi di moderazione, della validazione delle proposte e dell'aggiornamento dei relativi stati di avanzamento.

\item[Definizione \refstepcounter{definition}\thedefinition\label{def:statoproposta} – Stato di una proposta:]
Ogni proposta è associata a uno stato che ne descrive l’avanzamento nel proprio ciclo di vita.
Gli stati previsti sono: \emph{bozza}, \emph{pubblicata}, \emph{in valutazione}, \emph{accettata}, \emph{rifiutata}, \emph{in attuazione}, \emph{completata}.

\item[Definizione \refstepcounter{definition}\thedefinition\label{def:propostacollettiva} – Proposta collettiva:]
Proposta presentata da un’associazione registrata alla piattaforma, distinta dalle proposte individuali in quanto esprime una posizione formalmente attribuibile a un soggetto collettivo.

\item[Definizione \refstepcounter{definition}\thedefinition\label{def:endorsement} – Endorsement:]
Azione con cui un’associazione esprime sostegno a una proposta pubblicata. Gli endorsement sono visibili pubblicamente e possono avere peso aggiuntivo nei processi di valutazione dell'amministrazione.

\end{description}

\section{Vantaggi}

\subsection{Per il Comune}

\begin{itemize}
	
	\item \textbf{Prioritizzazione sugli interventi}: il sistema di voto fornisce al comune una visione precisa sulle priorità della cittadinanza, offrendo all'amministrazione l'opportunità di effettuare aggiustamenti sull'agenda basandosi su quanto fornito dai movimenti della comunità.

  \item \textbf{Riduzione del carico sugli sportelli}: grazie alla centralizzazione del canale per le proposte, si diminuisce il volume di email e richieste rivolte agli Uffici per le Relazioni con il Pubblico, risultando in risparmi temporali e organizzativi.
  
\end{itemize}

\subsection{Per i cittadini}

\begin{itemize}
  
  	\item \textbf{Democrazia diretta}: il sistema permette ai cittadini di esporre idee che hanno la possibilità di concretizzarsi nella vita pubblica, fornendo potere decisionale anche al singolo.
  	
	  \item \textbf{Accesso semplice e universale}: i cittadini possono accedere alla piattaforma da qualsiasi luogo e in qualsiasi momento, senza doversi recare fisicamente agli uffici comunali competenti, o dover ricorrere a mezzi di comunicazione secondari, quali email.
	  
    \item \textbf{Incremento nella partecipazione}: la piattaforma è in primis pensata per avvicinare i cittadini al Comune; attraverso l'espressione volta al miglioramento della vita comunitaria, la popolazione si avvicina certamente alle istituzioni stesse.

  \item \textbf{Maggiore trasparenza}: per ogni proposta analizzata vengono esposti per intero tutti i processi decisionali, garantendo la totale trasparenza riguardo alle scelte fatte.
  
\end{itemize}

\section{Limiti e mitigazioni}

\begin{itemize}

  \item \textbf{Definizione dei criteri di valutazione}:
 Per l'amministrazione, definire dei parametri di scelta per l'attuazione di una proposta privi di condizionamento ideologico è intrinsecamente complesso.

  \item \textbf{Neutralità e bias cognitivi}:
 	I comportamenti collettivi possono introdurre distorsioni involontarie.
 	Si prevede di garantire attraverso uno spazio rispettoso e libero, un ambiente il più privo di condizionamenti possibile (\ref{rnf:etica}). 
 	
  \item \textbf{Rischio di manipolazione di identità}:
  La creazione di identità false può alterare i voti.
  Il sistema verifica l'identità tramite SPID/CIE, che agisce come garante sull'unicità del soggetto.

  \item \textbf{Digital divide}:
  Una porzione della comunità potrebbe trovare difficoltà nell'utilizzo di strumenti digitali.
  Sono previsti presso il Comune degli sportelli per fornire assistenza sull'utilizzo della piattaforma.

  \item \textbf{Competenza multi-ente}:
  Alcune delle proposte potrebbero ricadere su enti differenti dal solo Comune (Provincia, Regione, Nazione).
  L'amministrazione ha eventualmente la possibilità di reindirizzare la proposta verso l'ente competente.

  \item \textbf{Aspettative ambigue}:
  In quanto la piattaforma è un'estensione del diritto sul territorio comunale, l'approvazione e qualsiasi azione svolta su una data proposta non hanno valore giuridico, dunque non genera obblighi.

  \item \textbf{Costi ricorrenti e sostenibilità}:
  La piattaforma per garantire il corretto funzionamento necessita di risorse continuative da spendere negli ambiti di moderazione, assistenza e manutenzione del software.
  
\end{itemize}

\newpage
\section*{Slides Pitch}

\vspace{0.6cm}

% --- SLIDE 1 ---
\noindent
\begin{minipage}[t]{0.48\textwidth} 
    \vspace{0pt} 
    \setlength\fboxsep{0pt}
    \setlength\fboxrule{0.5pt}
    \color{grayline}\fbox{\includegraphics[width=\textwidth]{img/D0/1.png}}
\end{minipage}
\hfill
\begin{minipage}[t]{0.48\textwidth} 
    \vspace{0pt} 
    {\large\textbf{\textcolor{trentogreen}{Slide 1 — Il problema}}}\\[3mm]
    {\small 
        \textbf{Bassa partecipazione attiva}: la comunità non è sufficientemente coinvolta nella vita pubblica.  \\[1mm]
        \textbf{Processi decisionali poco accessibili}: strumenti complessi o poco utilizzati.
    }
\end{minipage}

\vspace{0.6cm}
{\color{grayline}\hrule} 
\vspace{0.6cm}

% --- SLIDE 2 ---
\noindent
\begin{minipage}[t]{0.48\textwidth}
    \vspace{0pt}
    \setlength\fboxsep{0pt}
    \setlength\fboxrule{0.5pt}
    \color{grayline}\fbox{\includegraphics[width=\textwidth]{img/D0/2.png}}
\end{minipage}
\hfill
\begin{minipage}[t]{0.48\textwidth}
    \vspace{0pt}
    {\large\textbf{\textcolor{trentogreen}{Slide 2 — La soluzione}}}\\[3mm]
    {\small 
        \textbf{Trento Decide} avvicina cittadini e Comune attraverso la pubblicazione, discussione e votazione di iniziative.
    }
\end{minipage}

\vspace{0.6cm}
{\color{grayline}\hrule}
\vspace{0.6cm}

% --- SLIDE 3 ---
\noindent
\begin{minipage}[t]{0.48\textwidth}
    \vspace{0pt}
    \setlength\fboxsep{0pt}
    \setlength\fboxrule{0.5pt}
    \color{grayline}\fbox{\includegraphics[width=\textwidth]{img/D0/3.png}}
\end{minipage}
\hfill
\begin{minipage}[t]{0.48\textwidth}
    \vspace{0pt}
    {\large\textbf{\textcolor{trentogreen}{Slide 3 — Vantaggi}}}\\[3mm]
    {\small 
        \textbf{Democrazia diretta}: ogni cittadino ha potere decisionale concreto sul Comune. \\[1mm]
        \textbf{Attenzione alla vita pubblica}: più responsabilità, dunque più coinvolgimento. \\[1mm]
        \textbf{Divario di percezione}: l’amministrazione non vede tutti i problemi quotidiani vissuti dai cittadini.
    }
\end{minipage}

\vspace{0.6cm}
{\color{grayline}\hrule}
\vspace{0.6cm}

% --- SLIDE 4 ---
\noindent
\begin{minipage}[t]{0.48\textwidth}
    \vspace{0pt}
    \setlength\fboxsep{0pt}
    \setlength\fboxrule{0.5pt}
    \color{grayline}\fbox{\includegraphics[width=\textwidth]{img/D0/4.png}}
\end{minipage}
\hfill
\begin{minipage}[t]{0.48\textwidth}
    \vspace{0pt}
    {\large\textbf{\textcolor{trentogreen}{Slide 4 — Che dicono i dati?}}}\\[3mm]
    {\small 
        Tasso di istruzione: \textbf{75,3\%} (+9,8\%).\\[1mm]
        Associazioni per abitante: \textbf{1 ogni 83} (+102,8\%).\\[1mm]
        Partecipazione e volontariato: \textbf{18,4\%} (+12,2\%).\\[1mm]
        Relazioni definite rispetto alla media Nazionale.\\[2mm]
        \textit{\textcolor{gray}{Fonti: Trentino Open Data, Istat}}
    }
\end{minipage}
\vspace{0.8cm}