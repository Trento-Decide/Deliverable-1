\setcounter{secnumdepth}{3}

\chapter{Requisiti Non Funzionali}

\renewcommand{\thesubsection}{RNF\arabic{subsection}}
\setlength{\parskip}{0.4em}
\setlength{\parindent}{0pt}

\section{Introduzione}
I requisiti non funzionali definiscono le proprietà qualitative, operative e normative che il sistema \textit{Trento Decide} deve rispettare per garantire sicurezza, efficienza, accessibilità e affidabilità.  
Ogni requisito è formulato in modo chiaro, verificabile e, ove possibile, riferito a metriche misurabili e standard riconosciuti.

\subsection{Affidabilità}
Il sistema deve garantire una disponibilità minima del 99.72\% (downtime massimo 24 ore/anno).  
Devono essere definiti obiettivi RTO $\le$ 2 ore e RPO $\le$ 12 ore.  
Deve essere previsto un piano di \textit{incident response} e test di resilienza semestrali.

\subsection{Backup e ripristino}
Il sistema deve eseguire backup automatici ogni 12 ore di dati, configurazioni e log applicativi.  
I backup devono essere conservati per almeno 90 giorni e replicati in un data center situato nell’Unione Europea qualificato AgID. 
Devono essere previsti controlli di integrità e procedure di ripristino periodicamente testate.

\subsection{Compatibilità}
Il sistema deve essere conforme agli standard HTML5, CSS3 ed ECMAScript 6, garantendo la piena compatibilità con i principali browser utilizzati nei contesti della Pubblica Amministrazione e dagli utenti finali.  
Il supporto deve coprire almeno le ultime due versioni stabili dei browser Mozilla Firefox (inclusa ESR), Google Chrome/Chromium, Microsoft Edge, Apple Safari e Opera.    
L’interfaccia deve risultare completamente responsive e accessibile da desktop, tablet e smartphone.

\subsection{Etica e imparzialità del sistema} \label{ref:etica}
Il sistema deve promuovere trasparenza, equità e neutralità nei processi di partecipazione, evitando ogni forma di bias politico, ideologico o sociale.  
Tale principio si applica sia ai comportamenti umani sia ai processi automatizzati.

\subsubsection{Neutralità e trasparenza algoritmica} \label{ref:neutralita}
Gli algoritmi di ranking e simulazione devono essere progettati per evitare distorsioni sistematiche o favoritismi impliciti.  
Le formule e i parametri utilizzati devono essere documentati, versionati e sottoposti a revisione almeno annuale.  
Ogni modifica deve essere tracciata in un registro pubblico consultabile dai cittadini.

\subsection{Internazionalizzazione e lingua} \label{ref:i18n}
Il sistema deve supportare almeno cinque lingue: Italiano, Inglese, Tedesco, Arabo (RTL) e Rumeno.  
La lingua dell’interfaccia deve essere selezionabile e persistente a livello di profilo utente.  
I contenuti generati dagli utenti rimangono nella lingua di origine, con possibilità di traduzione automatica informativa chiaramente segnalata.

\subsection{Moderazione e correttezza d’uso} \label{ref:moderazione}
Il sistema deve garantire un ambiente rispettoso mediante meccanismi di moderazione dei contenuti e delle interazioni tra utenti.  
Le segnalazioni devono essere prese in carico entro 24 ore dalla ricezione.  
Tutte le azioni di moderazione devono essere registrate (utente, data/ora, motivazione) con log non alterabile; deve esistere un canale di appello per l’utente.

\subsection{Performance} \label{ref:performance}
Il sistema deve garantire che il 95\% delle operazioni principali (login, consultazione, voto) sia completato entro 1 secondo.  
Tale requisito deve essere mantenuto con un carico simultaneo fino a 20.000 utenti.  
L’interfaccia deve rispettare i target \textit{Core Web Vitals}: LCP $\le$ 2,5 s (p75), INP $\le$ 200 ms (p75), CLS $\le$ 0,1 (p75).

\subsection{Portabilità}
Il sistema deve poter essere distribuito su infrastrutture on-premise o cloud qualificati nell’Unione Europea.  
La distribuzione deve essere containerizzata e replicabile tramite strumenti di \textit{Infrastructure as Code}.  
Compatibilità garantita con database relazionali standard (es. PostgreSQL) e storage a oggetti.

\subsection{Scalabilità}
Il sistema deve supportare la scalabilità orizzontale e verticale senza downtime.  
L’architettura deve permettere l’aggiunta di nuove istanze di servizio e la gestione di carichi concorrenti elevati.

\subsection{Sicurezza}
Il sistema deve rispettare il livello 2 dello standard OWASP ASVS e adottare il principio di \textit{security by design}.  
Le comunicazioni devono essere cifrate, i dati sensibili protetti e devono essere attuate misure contro attacchi comuni (CSRF, XSS, SQLi, brute force).  
È richiesto un audit di sicurezza annuale e la disponibilità di un piano DPIA approvato dal DPO.

\subsection{Usabilità e accessibilità}
Il sistema deve essere utilizzabile senza formazione per gli utenti esterni; la formazione per operatori non deve superare 45 minuti.  
L’interfaccia deve rispettare gli standard WCAG 2.1 livello AA e raggiungere un punteggio minimo SUS $\ge$ 80.