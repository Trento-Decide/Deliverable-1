\setcounter{secnumdepth}{3}

\chapter{Requisiti Non Funzionali}

\renewcommand{\thesubsection}{RNF\arabic{subsection}}
\setlength{\parskip}{0.4em}
\setlength{\parindent}{0pt}

\section{Introduzione}
I requisiti non funzionali definiscono le proprietà qualitative, operative e normative che il sistema \textit{Trento Decide} deve rispettare per garantire sicurezza, efficienza, accessibilità e affidabilità.  
Ogni requisito è formulato in modo chiaro, verificabile e, ove possibile, riferito a metriche misurabili e standard riconosciuti.

\subsection{Affidabilità}
Il sistema deve garantire una disponibilità minima del 99.72\% (downtime massimo 24 ore/anno).  
Devono essere specificati RTO $\le$ 2 ore e RPO $\le$ 12 ore.  
È richiesto un piano di \textit{incident response}, una status page pubblica e test di resilienza semestrali.

\subsection{Backup e ripristino}
Il sistema deve eseguire backup automatici ogni 12 ore di dati, configurazioni e log applicativi. I backup devono essere cifrati con AES-256 e le chiavi gestite tramite KMS con rotazione periodica. I backup devono essere conservati per almeno 90 giorni, replicati off-site in un data center UE qualificato AgID.

\subsection{Compatibilità}
Il sistema deve essere conforme agli standard HTML5, CSS3 ed ECMAScript 6, garantendo la piena compatibilità con i principali browser utilizzati nei contesti della Pubblica Amministrazione e dagli utenti finali.  
Il supporto deve coprire almeno le ultime due versioni stabili dei seguenti browser: Mozilla Firefox (inclusa ESR), Google Chrome/Chromium, Microsoft Edge, Apple Safari e Opera.  
Il sistema deve assicurare la fruibilità completa e uniforme delle interfacce e delle funzionalità indipendentemente dal browser o dal dispositivo impiegato, risultando completamente responsive e accessibile da desktop, tablet e smartphone.

\subsection{Etica e imparzialità del sistema} \label{ref:etica}
Il sistema deve essere progettato per garantire neutralità e imparzialità nelle interazioni tra cittadini e amministrazione, promuovendo trasparenza, equità e rispetto del pluralismo civico. 
Tale principio si applica tanto ai comportamenti umani quanto ai processi automatizzati, promuovendo trasparenza, equità e rispetto del pluralismo civico.  

\subsubsection{Neutralità e trasparenza algoritmica} \label{ref:neutralita}
Come parte integrante del principio etico di imparzialità, gli algoritmi di ranking e simulazione devono essere progettati per evitare distorsioni sistematiche o favoritismi impliciti.  
Le formule, i parametri e le metriche di equità devono essere documentati, versionati e sottoposti a revisione annuale.  
Ogni modifica deve essere tracciata in un registro pubblico e accessibile ai cittadini.

\subsection{Internazionalizzazione e lingua} \label{ref:i18n}
Il sistema deve supportare almeno cinque lingue: Italiano, Inglese, Tedesco, Arabo (RTL) e Rumeno \footnote{La selezione linguistica riflette le principali comunità residenti nel Comune di Trento e le lingue ufficiali della Provincia Autonoma (Italiano e Tedesco), garantendo inclusività per le comunità araba e rumena, tra le più numerose.
}.
La selezione linguistica riflette le principali comunità residenti nel Comune di Trento e le lingue ufficiali della Provincia Autonoma.  
La scelta della lingua deve essere modificabile in qualsiasi momento e persistente a livello di profilo utente.  
I contenuti generati dagli utenti (proposte, commenti, titoli) rimangono nella lingua di origine.  
È tuttavia prevista la possibilità di fornire traduzioni automatiche a scopo informativo, chiaramente marcate come tali, per evitare la creazione di silos linguistici.

\subsection{Moderazione e correttezza d’uso} \label{ref:moderazione}
Il sistema deve garantire un ambiente rispettoso mediante un insieme integrato di meccanismi di moderazione dei contenuti e delle interazioni tra utenti. Le segnalazioni devono essere prese in carico entro 24 ore dalla loro ricezione. I moderatori agiscono anche come risolutori diretti delle problematiche trattate nel \ref{ref:etica}, assicurando coerenza etica e applicazione uniforme delle policy.

\subsection{Performance} \label{ref:performance}
Il sistema deve garantire tempi di risposta brevi, tali che qualsiasi operazione — come login, visualizzazione delle proposte o voto — venga completata entro un massimo di 1 secondo per il 95\% delle richieste.  
Tale requisito deve essere mantenuto anche con un carico simultaneo fino a 20.000 utenti connessi (o carico target equivalente espresso in RPS).  .  
Il valore di riferimento corrisponde al picco stimato durante consultazioni cittadine di grande rilevanza, in linea con dati osservati su piattaforme analoghe (es. Decidim Barcellona e ParteciPa).
Per l’interfaccia utente, devono essere rispettati i target Core Web Vitals (LCP $\le$ 2,5 s p75, INP $\le$ 200 ms p75, CLS $\le$ 0,1 p75) su dispositivi e network rappresentativi.

\subsection{Portabilità}
Il sistema deve poter essere distribuito su infrastrutture on-premise o cloud UE qualificati.  
La distribuzione deve essere containerizzata (Docker/Kubernetes) e replicabile tramite \textit{Infrastructure as Code}.  
Compatibilità garantita con database PostgreSQL e storage a oggetti.

\subsection{Scalabilità}
Il sistema deve supportare scalabilità orizzontale e verticale senza downtime.
Il database deve prevedere read replicas; l’elaborazione asincrona deve essere gestita tramite code/queue con meccanismi di retry con backoff e operazioni idempotenti. I componenti stateless devono essere preferiti per facilitare l’orchestrazione e l’aumento di capacità.

\subsection{Sicurezza}
Il sistema deve rispettare OWASP ASVS livello 2 come baseline e adottare il principio di \textit{security by design}.  
Comunicazioni cifrate tramite TLS 1.3; dati sensibili cifrati con AES-256; password conservate con Argon2 o bcrypt.  
Devono essere implementate protezioni contro CSRF, XSS, SQLi e brute force; auditing completo e pentest annuale.  
È richiesto un piano DPIA e l’approvazione del DPO.

\subsection{Usabilità e accessibilità}
Il sistema deve essere utilizzabile senza formazione per gli utenti esterni; la formazione per operatori non deve superare 45 minuti.  
L’interfaccia deve rispettare gli standard WCAG 2.1 livello AA e ottenere un punteggio SUS $\ge$ 80.  
  
\section{Standard e linee guida di riferimento}
Questa appendice elenca gli standard tecnici, normativi e metodologici cui il sistema \textit{Trento Decide} si conforma o che ne ispirano la progettazione.  
Essi garantiscono interoperabilità, sicurezza, trasparenza e conformità alle normative nazionali ed europee.

\subsection*{Normativa europea e nazionale}
\begin{itemize}
  \item \textbf{Regolamento (UE) 2016/679 – GDPR}  
  Protezione dei dati personali e sicurezza del trattamento (artt. 5, 24, 32, 35).

  \item \textbf{Decreto Legislativo 82/2005 – Codice dell’Amministrazione Digitale (CAD)}  
  Principi e regole tecniche per la digitalizzazione della Pubblica Amministrazione italiana.

  \item \textbf{Linee Guida AgID su sicurezza, accessibilità e interoperabilità (2022–2024)}  
  Requisiti per infrastrutture cloud qualificate, gestione identità, conservazione, design e accessibilità.

  \item \textbf{Regolamento (UE) 910/2014 – eIDAS}  
  Riconoscimento transfrontaliero dei sistemi di identificazione elettronica e dei servizi fiduciari.

  \item \textbf{Linee Guida SPID/CIE – OIDC e SAML 2.0}  
  Specifiche tecniche per l’integrazione dei sistemi di autenticazione federata con la Pubblica Amministrazione italiana.
\end{itemize}

\subsection*{Sicurezza e gestione delle informazioni}
\begin{itemize}
  \item \textbf{ISO/IEC 27001:2022} – Sistemi di gestione della sicurezza delle informazioni (ISMS).
  \item \textbf{ISO/IEC 27701:2019} – Estensione per la gestione della privacy (PIMS).
  \item \textbf{OWASP ASVS – Application Security Verification Standard (Livello 2–3)}  
  Requisiti di sicurezza per applicazioni web, autenticazione e gestione sessioni.
  \item \textbf{TLS 1.3} – Protocollo crittografico per la protezione delle comunicazioni.
  \item \textbf{AES-256} – Standard di cifratura simmetrica per dati a riposo.
  \item \textbf{CSP, HSTS, CSRF/XSS Protection} – Header e pratiche di sicurezza applicativa.
\end{itemize}

\subsection*{Accessibilità, usabilità e design}
\begin{itemize}
  \item \textbf{WCAG 2.1 – Livello AA}  
  Linee guida internazionali per l’accessibilità dei contenuti web (W3C).
  \item \textbf{Linee Guida AgID per il design dei servizi digitali della PA (Design System Italia)}  
  Standard grafici e di interazione per garantire coerenza e accessibilità.
  \item \textbf{System Usability Scale (SUS)} – Metodo di valutazione dell’usabilità con soglia minima SUS $\ge$ 80.
\end{itemize}

\subsection*{Interoperabilità e open data}
\begin{itemize}
  \item \textbf{DCAT-AP_IT} - Profilo italiano di interoperabilità per cataloghi Open Data.
  \item \textbf{Linee Guida AgID Open Data (2022)} - Pubblicazione, licenze e metadati per i dataset pubblici.
  \item \textbf{Licenze CC-BY 4.0 e Italian Open Data License (IODL 2.0)} - Riusabilità dei dati aperti.
  \item \textbf{OpenAPI Specification 3.1} - Standard per la descrizione e documentazione di API REST pubbliche.
\end{itemize}

\subsection*{Tecnologie e compatibilità}
\begin{itemize}
  \item \textbf{HTML5 / CSS3 / ECMAScript 6} – Standard tecnici per la compatibilità tra browser.
  \item \textbf{Docker / Kubernetes} – Containerizzazione e orchestrazione per la portabilità del sistema.
  \item \textbf{PostgreSQL} – Database relazionale di riferimento per l’architettura applicativa.
  \item \textbf{OpenTelemetry} – Standard per l’osservabilità e il monitoraggio distribuito.
\end{itemize}