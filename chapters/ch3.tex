% !TeX spellcheck = it_IT
\setcounter{secnumdepth}{3}

\chapter{Requisiti Non Funzionali}

\setlength{\parskip}{0.4em}
\setlength{\parindent}{0pt}

\section{Introduzione}
I requisiti non funzionali descrivono le condizioni operative e qualitative che il sistema deve rispettare per rimanere sicuro, efficiente e utilizzabile. Ogni requisito è formulato in modo verificabile e, quando possibile, fa riferimento a valori misurabili o a standard riconosciuti.

\begin{rnfscope}

\section{Affidabilità} \label{rnf:affidabilita}
Il sistema deve garantire una disponibilità annuale del 99,91\%, che corrisponde ad un massimo di circa otto ore di inattività all'anno.
In caso di interruzione, il ripristino del servizio deve avvenire entro 2 ore (RTO), mentre la perdita massima di dati consentita non può superare le 12 ore (RPO).
È inoltre necessario prevedere un piano di gestione degli incidenti (incident response) e svolgere test di resilienza ogni sei mesi, includendo simulazioni reali di ripristino.

\section{Anonimato} \label{rnf:anonimato}
Il sistema deve garantire che tutti i dati utilizzati per la generazione dei report amministrativi (\ref{rf:report}) e per l’analisi dell’attività della piattaforma siano trattati in forma anonimizzata.
Nessun log, metadato o aggregazione statistica deve permettere l'identificazione diretta o indiretta dell'identità degli utenti, neppure tramite combinazione di più dataset.
Le procedure di anonimizzazione devono essere applicate prima che i dati siano resi disponibili all’amministrazione o esportabili, assicurando che ogni informazione personale venga rimossa o generalizzata in conformità alle linee guida GDPR\@.

\section{Backup e ripristino} \label{rnf:backup}
Il sistema deve eseguire backup automatici ogni 3 ore di dati, configurazioni applicative e log di sistema. I backup devono essere conservati per almeno 90 giorni e replicati in un data center dell’Unione Europea
qualificato AgID e geograficamente separato dalla sede principale.
Devono essere previsti controlli di integrità dei backup, verificando la coerenza dei dati e la loro ripristinabilità, e devono essere svolti test semestrali delle procedure di ripristino.

\section{Compatibilità} \label{rnf:compatibilita}
Il sistema deve rispettare gli standard HTML5, CSS3 ed ECMAScript 6 e funzionare correttamente sui browser maggiormente diffusi nella Pubblica Amministrazione e tra gli utenti.
È richiesta la compatibilità minima con le seguenti versioni: Firefox 52 ESR, Chromium/Chrome 52, Opera 40, Safari 10 e Microsoft Edge 79 o successivi.
L’interfaccia utente deve essere \emph{responsive}, adattandosi a desktop, tablet e smartphone.

\section{Etica e imparzialità del sistema} \label{rnf:etica}
Il sistema deve garantire trasparenza, equità e neutralità in tutti i processi.
È necessario prevenire l’introduzione di bias politici, ideologici o sociali, che possono derivare dalla logica applicativa o da un uso scorretto degli strumenti da parte degli utenti (cfr.~\ref{rf:moderazioneutenti}). Tali principi riguardano esclusivamente il funzionamento della piattaforma e non limitano l’espressione degli utenti.

\section{Internazionalizzazione e lingua} \label{rnf:lingua}
Il sistema deve supportare almeno le seguenti lingue: Italiano, Inglese, Tedesco, Arabo (RTL) e Rumeno.
La lingua dell’interfaccia deve essere selezionabile e persistente a livello di profilo utente (\ref{rf:cambiolingua}).
I contenuti generati dagli utenti devono rimanere nella lingua di origine, con la possibilità di tradurre automaticamente il contenuto a solo scopo informativo, quest'ultima deve essere chiaramente segnalata tramite interfaccia utente.

\section{Moderazione e correttezza d’uso} \label{rnf:moderazione}
Il sistema deve garantire un ambiente rispettoso tramite strumenti di moderazione dei contenuti e delle interazioni tra utenti.
Le segnalazioni devono essere prese in carico entro 24 ore dalla loro ricezione.
Ogni intervento di moderazione deve essere tracciato (utente, data/ora, motivazione) in un log non alterabile.

\section{Moderazione automatica} \label{rnf:moderazioneautomatica}
La moderazione automatica (si veda~\ref{rf:moderazioneautomatica}) deve limitarsi all’oscuramento temporaneo e alla segnalazione, senza rimozioni definitive.  I criteri d'intervento devono essere imparziali, scattando esclusivamente a tutela degli utenti contro contenuti gravemente offensivi, senza intaccare la libera espressione del cittadino.

\section{Performance} \label{rnf:performance}
Il sistema deve garantire che il 95\% delle operazioni principali (autenticazione, consultazione dei contenuti, espressione del voto) sia completato entro 1 secondo.
Tale requisito deve essere mantenuto con un carico simultaneo fino a 20.000 utenti attivi.
L’interfaccia deve rispettare i target \emph{Core Web Vitals}: LCP $\leq$ 2{,}5 s (p75), INP $\leq$ 200 ms (p75), CLS $\leq$ 0{,}1 (p75),
misurati su dispositivi reali e non mediante test sintetici locali.

\section{Portabilità} \label{rnf:portabilita}
Il sistema deve essere deployabile sia su infrastrutture on-premise che su cloud qualificati nell’Unione Europea (conformi AgID).
Deve essere garantita la compatibilità con database relazionali standard (come PostgreSQL).

\section{Scalabilità} \label{rnf:scalabilita}
Il sistema deve supportare la scalabilità orizzontale e verticale senza downtime.
L’architettura deve permettere l’aggiunta di nuove istanze di servizio e la gestione di carichi concorrenti elevati.

\section{Sicurezza} \label{rnf:sicurezza}
Il sistema deve seguire il paradigma \emph{security by design}, implementando i meccanismi necessari per proteggere i dati
e le funzionalità del sistema contro minacce comuni e accessi non autorizzati.
Tutte le comunicazioni devono essere cifrate tramite protocolli sicuri (es. TLS 1.2 o superiore) e i dati sensibili
devono essere protetti tramite tecniche di hashing o cifratura a riposo.
Devono essere previste contromisure contro attacchi applicativi frequenti, come XSS, CSRF, SQL injection e tentativi
di bruteforce (OWASP Top 10).
È inoltre richiesto un aggiornamento periodico delle componenti software e una revisione di sicurezza annuale, nel rispetto delle normative vigenti e del GDPR\@.

\section{Usabilità e accessibilità} \label{rnf:useacc}
Il sistema deve essere utilizzabile senza alcun tipo di formazione per gli utenti esterni; il training per operatori interni non deve eccedere i 45 minuti.
L’interfaccia deve rispettare gli standard WCAG 2.1 livello AA e raggiungere un punteggio minimo SUS $\ge$ 80.

\end{rnfscope}